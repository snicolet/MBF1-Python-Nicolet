% Book of exercises for Python programming
%
%	Formatting:
%	
%	%\begin{myboxi}[Remarque]
%	Rappel: x = x+1
%	est équivalent à:
%	x += 1
%	\end{myboxi}
%	
%	Labelling of the exercises: keyword_n (with n positive integer)
%	
%	\begin{exercice} \label{ex:keyword_n} 
%	Title
%	
%	Text of the exercise with some inline code \inlinecode{x = 3} and with some maths $3 + 4 = 7$
%	
%	Listing with python formatting
%	\begin{lstlisting}
%	multi
%	line 
%	code
%	\end{lstlisting}
%
%	Verbatim listing (e.g. console)
%	\begin{lstlisting}[style=verbatim]
%	> > > 3+2
%	5
%	\end{lstlisting}
%	\end{exercice}
%
%
%	Fill out on this line: \dotfill
%
%
%	Links: \url{http://a.b/}
%
%	Wrapped picture:
%	\begin{wrapfigure}{r}{0.2\textwidth}
%	  \begin{center}
%	    \includegraphics[width=0.18\linewidth]{graphics/tp1_idle_icon.png}
%	  \end{center}
%	\end{wrapfigure}
%
%
%
%	Full width picture
%	\includegraphics[width=\linewidth]{graphics/tp1_1.png} 


\documentclass[notitlepage,a4paper,10pt,twoside]{article}

% Packages:
\usepackage[T1]{fontenc}
\usepackage[utf8]{inputenc}
\usepackage{lmodern,amsmath,amsthm,tabularx}
\usepackage[english,french]{babel}
\usepackage{courier,enumitem}
\usepackage{textcomp,fancyhdr}
\usepackage{wrapfig}
\usepackage{multicol}
\usepackage[inner=3cm,outer=3cm,top=3cm,bottom=3cm]{geometry}
\usepackage{changepage,graphicx,epstopdf,emptypage}
\epstopdfsetup{outdir=./graphics/}
\usepackage{refcount}
\usepackage[yyyymmdd,hhmmss]{datetime}
\usepackage{environ,filecontents}
\usepackage{catchfile}
\usepackage{xcolor}
% \usepackage{hyperref}

\usepackage[yyyymmdd,hhmmss]{datetime}
\renewcommand{\dateseparator}{}


% Hyperrefs
\usepackage[hypcap=false]{caption}
\usepackage{amssymb}            % Fancy math symbols.
\usepackage{euscript}           % Nice script font.
\usepackage[hidelinks]{hyperref}
%%%%%%%%

%
%% source code specific stuff
%\usepackage{fancyvrb}
%\usepackage{listings}
%\fvset{frame=leftline,framerule=1mm,framesep=2mm,numbers=left,numbersep=2mm}
%\lstset{
%  language=Python,
%  basicstyle=\linespread{1.2}\ttfamily
%}




% Set margins for lists:
\setlist[itemize]{leftmargin=*}
\setlist[enumerate]{labelindent=0px}
\setlist[description]{leftmargin=*}
%%%%%%%%

% Exercise and solution environments:
\newenvironment{solution}[1][\unskip]{%
\setcounterref{exercice}{#1}
\par
\noindent
\textbf{Exercice \ref{#1}.} \newline
\noindent
}{
\newpage
}


\setcounter{secnumdepth}{0}


\def\appendsolution{\immediate\write18{cat temp.tex >> sol.tex}}


%\NewEnviron{solution}[1][\unskip]{
%\setcounterref{exercice}{#1}
%\par
%\noindent
%\textbf{Exercice \ref{#1}.} \newline
%\noindent
%\BODY
%}{}


%\NewEnviron{solution}[1][\unskip]{%
%\global\let\putsectionA\BODY}



%\usepackage{fancyvrb}
%\newenvironment{tobiwrite}[1]
%  {\typeout{Writing file #1}\VerbatimOut{#1}}
%  {\endVerbatimOut}



%\newwrite\appendwrite
%\newcommand*\appendBtoA[2]{%
%    \begingroup
%    \IfFileExists{#1}%
%      {\CatchFileDef{\filecontentA}{#1}{\endlinechar=`^^J\catcode\endlinechar=12\relax}}% keep existing end-of-lines
%      {\let\filecontentA\empty}%
%    \IfFileExists{#2}%
%      {\CatchFileDef{\filecontentB}{#2}{\endlinechar=`^^J\catcode\endlinechar=12\relax}}% keep existing end-of-lines
%      {\let\filecontentB\empty}%
%
%    \immediate\openout\appendwrite=#1\relax
%    %\immediate\write\appendwrite{\filecontentA \filecontentB}%
%    \immediate\write\appendwrite{\unexpanded\expandafter{\filecontentA}\unexpanded\expandafter{\filecontentB}}%
%    \immediate\closeout\appendwrite
%    \endgroup
%}



%\newread\in
%\newwrite\out
%
%\newcommand*\appendBtoA[2]{%
%\immediate\openin\in=#1
%\immediate\openout\out=#2
%\endlinechar-1
%\loop \unless\ifeof\in
%        \readline\in to\l
%        \immediate\write\out{\l}
%\repeat
%\immediate\closeout\out
%\closein\in
%\end






\newtheoremstyle{break}% name
  {}%         Space above, empty = `usual value'
  {}%         Space below
  {}% Body font
  {}%         Indent amount (empty = no indent, \parindent = para indent)
  {\bfseries}% Thm head font
  {.}%        Punctuation after thm head
  {\newline}% Space after thm head: \newline = linebreak
  {}%         Thm head spec

\theoremstyle{break}
\newtheorem{exercice}{Exercice}
%%%%%%%%

% set counters for lists inside exercise environments
\setlist[enumerate,1]{label=\textbf{\arabic{exercice}.\arabic{enumi}}.}
%%%%%%%%

% Remark boxes:
\usepackage{xcolor}
\definecolor{bgblue}{RGB}{245,243,253}
\definecolor{ttblue}{RGB}{91,194,224}
\usepackage[many]{tcolorbox}
\newtcolorbox{myboxi}[1][]{
  breakable,
  title=#1,
  colback=white,
  colbacktitle=white,
  coltitle=black,
  fonttitle=\bfseries,
  bottomrule=0pt,
  toprule=0pt,
  leftrule=3pt,
  rightrule=3pt,
  titlerule=0pt,
  arc=0pt,
  outer arc=0pt,
  colframe=black,
}
%%%%%%%

% padding in tables
\renewcommand{\arraystretch}{1.3}
%%%%%%%%



% Listings in the code
\usepackage{listings}
\usepackage{color}
\definecolor{mygreen}{rgb}{0,0.6,0}
\definecolor{mygray}{rgb}{0.5,0.5,0.5}
\definecolor{mymauve}{rgb}{0.58,0,0.82}
\definecolor{listingbg}{gray}{0.95}
\definecolor{verbatimbg}{rgb}{0.95,0.95,0.85}


\lstset{ %
  backgroundcolor=\color{listingbg},   % choose the background color; you must add \usepackage{color} or \usepackage{xcolor}
  basicstyle=\small\ttfamily,        % the size of the fonts that are used for the code
  breakatwhitespace=false,         % sets if automatic breaks should only happen at whitespace
  breaklines=true,                 % sets automatic line breaking
  captionpos=b,                    % sets the caption-position to bottom
  commentstyle=\color{mygreen},    % comment style
  deletekeywords={...},            % if you want to delete keywords from the given language
  escapeinside={\%*}{*)},          % if you want to add LaTeX within your code
  extendedchars=true,              % lets you use non-ASCII characters; for 8-bits encodings only, does not work with UTF-8
  frame=tb,                    % adds a frame around the code
  keepspaces=true,                 % keeps spaces in text, useful for keeping indentation of code (possibly needs columns=flexible)
  keywordstyle=\color{blue},       % keyword style
  language=Python,                 % the language of the code
  morekeywords={*,...},            % if you want to add more keywords to the set
  numbers=left,                    % where to put the line-numbers; possible values are (none, left, right)
  numbersep=5pt,                   % how far the line-numbers are from the code
  numberstyle=\tiny\color{mygray}, % the style that is used for the line-numbers
  rulecolor=\color{black},         % if not set, the frame-color may be changed on line-breaks within not-black text (e.g. comments (green here))
  showspaces=false,                % show spaces everywhere adding particular underscores; it overrides 'showstringspaces'
  showstringspaces=false,          % underline spaces within strings only
  showtabs=false,                  % show tabs within strings adding particular underscores
  stepnumber=1,                    % the step between two line-numbers. If it's 1, each line will be numbered
  stringstyle=\color{mymauve},     % string literal style
  tabsize=2,                       % sets default tabsize to 2 spaces
  title=\lstname,                 % show the filename of files included with \lstinputlisting; also try caption instead of title
  belowskip=-1.5em,
  aboveskip=1em,
  resetmargins=false
}

\lstdefinestyle{verbatim}
{language=,numbers=none,backgroundcolor=\color{verbatimbg}}  

%\lstset{inputencoding=utf8/latin1}

\lstset{
     literate=%
         {á}{{\'a}}1
         {í}{{\'i}}1
         {é}{{\'e}}1
		 {è}{{\`e}}1  
		 {ê}{{\^e}}1       
         {ç}{{\c{c}}}1
         {ú}{{\'u}}1
         {ô}{{\^o}}1
         {ê}{{\^{e}}}1
         {à}{{\`a}}1
         {î}{{\^{i}}}1
         {â}{{\^{a}}}1
         {ž}{{\v{z}}}1
         {ď}{{\v{d}}}1
         {ť}{{\v{t}}}1
         {ň}{{\v{n}}}1                
         {ů}{{\r{u}}}1
         {Á}{{\'A}}1
         {Í}{{\'I}}1
         {É}{{\'E}}1
         {Ý}{{\'Y}}1
         {Ú}{{\'U}}1
         {Ó}{{\'O}}1
         {Ě}{{\v{E}}}1
         {Š}{{\v{S}}}1
         {Č}{{\v{C}}}1
         {Ř}{{\v{R}}}1
         {Ž}{{\v{Z}}}1
         {Ď}{{\v{D}}}1
         {Ť}{{\v{T}}}1
         {Ň}{{\v{N}}}1                
         {Ů}{{\r{U}}}1    
}

% redefining lstinline with a different name because Texmaker behaves strangely...
\newcommand{\inlinecode}{\lstinline}
%\end{lstlisting} 
%%%%%%%%



% Maths stuff:
\newcommand{\numberset}{\mathbb}
\newcommand{\N}{\numberset{N}}
\newcommand{\Z}{\numberset{Z}}
\newcommand{\Q}{\numberset{Q}}
\newcommand{\R}{\numberset{R}}
\newcommand{\C}{\numberset{C}}
\newcommand{\boldP}{\numberset{P}}
\newcommand{\boldF}{\numberset{F}}
\renewcommand{\epsilon}{\varepsilon}
\renewcommand{\theta}{\vartheta}
% \renewcommand{\rho}{\rho}
\renewcommand{\phi}{\varphi}
\newcommand{\blank}{{-}}
\newcommand{\norm}[1]{\left\lVert#1\right\rVert}
\newcommand{\abs}[1]{\left\lvert#1\right\rvert}
%%%%%%%%%%%%



%\usepackage{answers}
%\Newassociation{solution}{sol}{ans}
%\Newassociation{solution}{solformatted}{ans}
%\newcommand{\solformattedparams}{}

\title{\bf Exercices de Python} %  \\ \large Cahier d'exercices pour les cours Informatique I et Informatique II}

\author{{\Large St\'ephane {\sc Nicolet}}\\
Universit\'e Paris 2 -- Panth\'eon-Assas\\
stephane.nicolet@assas-universite.fr
}
%\date{\today-v0.9}
\date{}

\pagestyle{plain}
\setlength{\headheight}{14.6pt}


\begin{document}
% \selectlanguage{french}


\maketitle  
\tableofcontents

\newpage


\pagestyle{fancy}
\fancyhead{}
\fancyfoot{}
\renewcommand{\headrulewidth}{0.3pt}
\renewcommand{\sectionmark}[1]{\markright{#1}}

\fancyhead[LE]{\bfseries \thepage}
\fancyhead[RE]{TD de Python}
\fancyhead[LO]{\nouppercase\rightmark}
\fancyhead[RO]{\bfseries \thepage}



%\part{Travaux dirig\'es}\
\section{Chapitre 1: Variables, types de base, fonctions d'entr\'ee et de sortie}

\setcounter{exercice}{0}

\bigskip

%%%%%%%%
%%%%%%%%
%%%%%%%%



\begin{exercice}[les nombres: types int et float] \label{ex1a}
\

\noindent Initialisez trois variables \inlinecode{x}, \inlinecode{y} et \inlinecode{z} \`a des nombre entiers ou flottants, puis familiarisez-vous avec les op\'erateurs arithm\'etiques de base en les appliquant sur ces variables. On rappelle que:
\begin{itemize}
\item[$\bullet$] l'addition se note \inlinecode{+}
\item[$\bullet$] la soustraction se note \inlinecode{-}
\item[$\bullet$] la multiplication se note \inlinecode{*}
\item[$\bullet$] l'exponentiation se note \inlinecode{**}
\item[$\bullet$] le modulo se note \%
\item[$\bullet$] la division se note \inlinecode{/}
\end{itemize}
En Python 2 (contrairement \`a Python 3), la division par d\'efaut est la division enti\`ere: cela signifie que si les deux nombres sont des entiers, c'est la division enti\`ere qui est appliqu\'ee. Si au moins un des deux nombres est r\'eel, c'est la division r\'eelle qui est appliqu\'ee.
\end{exercice}


\begin{filecontents*}{temp.tex}
\begin{solution}[ex1a]
\begin{lstlisting}[style=verbatim]
...
\end{lstlisting}
\end{solution}
\newpage
\end{filecontents*}
\appendsolution



\medskip
\begin{exercice}[les cha\^ines de caract\`eres: type string] \label{ex1b}
\

\noindent Initialisez deux variables \inlinecode{s} et \inlinecode{t} \`a des cha\^ines de caract\`eres, puis familiarisez-vous avec les op\'erateurs de base sur les cha\^ines en les appliquant sur ces variables. On rappelle que:
\begin{itemize}
\item[$\bullet$] la concat\'enation de cha\^ines de caract\`eres s'obtient par l'op\'erateur \inlinecode{+}
\item[$\bullet$] la r\'ep\'etition de cha\^ines de caract\`eres s'obtient par l'op\'erateur \inlinecode{*}
\item[$\bullet$] l'extraction de l'\'el\'ement d'indice \inlinecode{n} d'une cha\^ine \inlinecode{s} s'obtient par la syntaxe \inlinecode{s[n]} (les indices commencent \`a $0$)
\item[$\bullet$] l'extraction de l'\'el\'ement d'indice \inlinecode{n}, mais en partant de la fin de la cha\^ine \inlinecode{s}, s'obtient par la syntaxe \inlinecode{s[-n]} (dans ce cas, les indices commencent \`a $-1$)
\item[$\bullet$] l'extraction de la sous-cha\^ine de \inlinecode{s} d'indices \inlinecode{m} compris jusqu'\`a l'indice \inlinecode{n} non compris s'obtient par la syntaxe \inlinecode{s[m:n]}
\item[$\bullet$] la longueur d'une cha\^ine \inlinecode{s} s'obtient par la syntaxe \inlinecode{len(s)}
\end{itemize}
Pour la syntaxe \inlinecode{s[m:n]}, si l'indice \inlinecode{m} n'est pas sp\'ecifi\'e, il vaut $0$ par d\'efaut; si l'indice \inlinecode{n} n'est pas sp\'ecifi\'e, il vaut \inlinecode{len(n)} par d\'efaut.
\end{exercice}


\begin{filecontents*}{temp.tex}
\begin{solution}[ex1b]
\begin{lstlisting}[style=verbatim]
...
\end{lstlisting}
\end{solution}
\newpage
\end{filecontents*}
\appendsolution




\medskip
\begin{exercice}[les Bool\'eens: type bool] \label{ex1c}
\

\noindent En Python, les valeurs bool\'eennes se disent \inlinecode{True} et \inlinecode{False}. Les op\'erateurs logiques sont les suivants:
\begin{itemize}
\item[$\bullet$] la n\'egation se note \inlinecode{not}
\item[$\bullet$] le ``et'' logique, appel\'e aussi disjonction, se note \inlinecode{and}
\item[$\bullet$] le ``ou'' logique, appel\'e aussi conjonction, se note \inlinecode{or}
\end{itemize}
Initialisez quelques variables aux valeurs Bool\'eennes \inlinecode{True} et \inlinecode{False} et testez les op\'erateurs logiques sur vos variables en tapant quelques expressions logiques de votre choix.

\smallskip

\noindent De plus, les diff\'erents op\'erateurs de comparaison entre nombres (ou autre types \'egalement) se note \inlinecode{<}, \inlinecode{<=}, \inlinecode{>}, \inlinecode{>=}, \inlinecode{==}, \inlinecode{!=}. Les r\'esultat des \'evaluations d'expressions contenant ces op\'erateurs sont des valeurs Bool\'eennes, \inlinecode{True} ou \inlinecode{False}. Par exemple, l'expression \inlinecode{2 < 3} est \'evalu\'ee en \inlinecode{True} et l'expression \inlinecode{2 == 3} est \'evalu\'ee en \inlinecode{False}. Tapez quelques expressions de votre choix contentant ces op\'erateurs de comparaison.

\begin{myboxi}[Point important] Fa\^ites attention de ne pas confondre le symbole d'\'egalit\'e ``\inlinecode{=}'', qui sert \`a instancier une variable (par exemple \inlinecode{x = 2}), de celui de double \'egalit\'e ``\inlinecode{==}'', qui sert \`a tester l'\'egalit\'e entre deux op\'erands (par exemple \inlinecode{2 == 3}). \end{myboxi}
\end{exercice}


\begin{filecontents*}{temp.tex}
\begin{solution}[ex1c]
\begin{lstlisting}[style=verbatim]
...
\end{lstlisting}
\end{solution}
\newpage
\end{filecontents*}
\appendsolution




\medskip
\begin{exercice}[Cha\^ines de caract\`eres] \label{ex2}
\
\begin{enumerate}
\item Initialisez deux variables appel\'ees \inlinecode{prenom} et \inlinecode{nom} contenant votre pr\'enom et votre nom, respectivement.
\item En utilisant l'op\'erateur de concat\'enation sur les cha\^ines de caract\`eres (\inlinecode{+}), initialisez une variable \inlinecode{nom_complet} contenant votre pr\'enom et votre nom s\'epar\'es par un espace.
\item En utilisant l'op\'erateur de multiplication sur les cha\^ines de caract\`eres (\inlinecode{*}) et la fonction \inlinecode{print}, imprimez vos nom et pr\'enom une centaine fois \`a l'\'ecran s\'epar\'es par ``\verb# * #'' (espace - \'etoile - espace).
\item En utilisant les op\'erateurs d'extraction de sous-cha\^ines, initialisez \`a partir de la variable \inlinecode{nom_complet} une autre variable \inlinecode{initiales} contenant vos initiales s\'epar\'ees par un espace.
\item Imprimez vos initiales une centaine fois \`a l'\'ecran s\'epar\'es par ``\verb# * #'' (espace - \'etoile - espace).
\end{enumerate}
\end{exercice}


\begin{filecontents*}{temp.tex}
\begin{solution}[ex2]
\begin{lstlisting}[style=verbatim]
...
\end{lstlisting}
\end{solution}
\newpage
\end{filecontents*}
\appendsolution



\medskip
\begin{exercice}[Cha\^ines de caract\`eres] \label{ex3}
\

\noindent En utilisant la fonction \inlinecode{input()}, \'ecrivez un script qui demande \`a l'utilisateur d'entrer son pr\'enom, puis d'entrer son nom, puis affiche automatiquement le pr\'enom et le nom, les initiales et la premi\`ere lette du nom de famille, comme illustr\'e ci-dessous:
\begin{lstlisting}[style=verbatim]
>>> Quel est votre prenom? Astor
>>> Quel est votre nom? Piazzolla
Astor Piazzolla
A. P.
Premiere lettre du nom de famille: P
\end{lstlisting}
\end{exercice}


\begin{filecontents*}{temp.tex}
\begin{solution}[ex3]
\begin{lstlisting}[style=verbatim]
...
\end{lstlisting}
\end{solution}
\newpage
\end{filecontents*}
\appendsolution



\medskip
\begin{exercice}[int et float] \label{ex4}
\

\noindent Cr\'eez un script permettant de convertir des temp\'eratures de degr\'es Celsius en degr\'es Fahrenheit et vice-versa. La formule de conversion est la suivante:
$$
F = \frac{9}{5} \cdot C + 32
$$
Votre script devra demander \`a l'utilisateur la temp\'erature et afficher les deux conversions comme suit:
\begin{lstlisting}[style=verbatim]
Quelle est la temperature? 20.2
20.2 C = 68.36 F
20.2 F = -6.555555555 C
\end{lstlisting}
\end{exercice}


\begin{filecontents*}{temp.tex}
\begin{solution}[ex4]
\begin{lstlisting}[style=verbatim]
...
\end{lstlisting}
\end{solution}
\newpage
\end{filecontents*}
\appendsolution




\medskip
\begin{exercice}[Bool\'eens] \label{ex5}
\

\noindent Cr\'eez un programme qui, premi\`erement, demande \`a l'utilisateur d'entrer des valeurs de v\'erit\'e (\inlinecode{True} ou \inlinecode{False}) pour trois variables propositionnelle $A$, $B$ et $C$, puis, deuxi\`emement, demande \`a l'utilisateur d'entrer ``en fran\c{c}ais'' une expression de logique Bool\'eenne sur ces trois variables propositionnelles, et enfin, affiche l'\'evaluation de cette expression en fonction des valeurs de v\'erit\'e de $A$, $B$ et $C$. Le d\'eroulement du programme devra ressembler \`a l'ex\'ecution ci-dessous.

\begin{lstlisting}[style=verbatim]
Entrer une valeur de verite pour A: True
Entrer une valeur de verite pour B: False
Entrer une valeur de verite pour C: True
Entrer une expression Boolenne (exprimee avec des "non", "ou", "et" et des parentheses): 
non (A et B) ou (C et non B)
L'evaluation de votre expression Booleenne est: True
\end{lstlisting}
\end{exercice}

\begin{myboxi}[Indication] Si \inlinecode{s} est une cha\^ine de caract\`eres, la fonction \inlinecode{s.replace(old, new)} retourne une autre cha\^ine de caract\`eres o\`u toutes les occurrences de la sous-cha\^ine \inlinecode{old} dans \inlinecode{s} sont remplac\'ees par la la sous-cha\^ine \inlinecode{new}. Ainsi, il vous faudra remplacer les sous-cha\^ines \inlinecode{"non"}, \inlinecode{"et"} et \inlinecode{"ou"} par \inlinecode{"not"}, \inlinecode{"and"} et \inlinecode{"or"}, puis \'evaluer votre expression Bool\'eenne \`a l'aide de la fonction \inlinecode{bool()} pour obtenir le r\'esultat d\'esir\'e.\end{myboxi}


\begin{filecontents*}{temp.tex}
\begin{solution}[ex5]
\begin{lstlisting}[style=verbatim]
...
\end{lstlisting}
\end{solution}
\newpage
\end{filecontents*}
\appendsolution







%\begin{filecontents*}{temp.tex}
%
%\begin{solution}[ex:]
%
%\end{solution}
%
%
%\end{filecontents*}
%\appendsolution


     % variables
\clearpage
\section{Chapitre 2: Types de donn\'ees}

\setcounter{exercice}{0}

\bigskip


%%%%%%%%
%%%%%%%%
%%%%%%%%


% *** CHAINES *** %
\begin{myboxi}[Rappel sur les cha\^ines de caract\`eres] 
Les cha\^ines de caract\`eres sont donn\'ees entre guillemets. Par exemple: \inlinecode{s = "bonjour"}.

\medskip

Comme d\'ej\`a mentionn\'e, l'acc\`es aux \'el\'ements et aux sous-cha\^ines d'une cha\^ine \inlinecode{s} s'effectue via les instructions \inlinecode{s[n]}, \inlinecode{s[-n]}, \inlinecode{s[m:n]}, \inlinecode{s[m:]} et \inlinecode{s[:n]}.

\medskip

Les cha\^ines de caract\`eres sont des structures de donn\'ees non modifiables. Cela signifie que nous ne pouvons pas proc\'eder \`a des r\'eaffectation de leurs \'el\'ements:
\begin{lstlisting}
>>> s =  "abba"
>>> s[2] = "z"
Traceback (most recent call last):
  File "<stdin>", line 1, in <module>
TypeError: 'str' object does not support item assignment
\end{lstlisting}

\medskip

Le parcours d'une cha\^ine de caract\`eres s'effectue via les instructions:
\begin{lstlisting}
for variable in chaine:
	instructions...
\end{lstlisting}

\medskip

Il existe diverses m\'ethodes sur les cha\^ines de caract\`eres, nous en aborderons quelques unes.
\end{myboxi}


\medskip
\begin{myboxi}[Rappel sur les tests conditionnels \inlinecode{if... else...}]
En Python, un tel test conditionnel \inlinecode{if... else...} est obtenu via la syntaxe suivante (les blocs \inlinecode{elif} et \inlinecode{else} sont facultatifs):
\begin{lstlisting}
if condition:
	instructions...
elif:
	instructions...
...
else:
	instructions...
\end{lstlisting}
\end{myboxi}





\medskip
\begin{exercice}[cha\^ines de caract\`eres] \label{TD2ex1}
\

\vspace{-1.5ex}
\begin{enumerate}
\item Initialisez une variables \inlinecode{c} contenant la cha\^ine de caract\`eres contenant des chiffres et des lettres, du type \inlinecode{"X44bf38j23jdjgfjh737nei47"}.
\item \'Ecrivez un programme qui construit deux cha\^ines de caract\`eres  \inlinecode{c_alpha} et \inlinecode{c_num} telles que \inlinecode{c_alpha} repr\'esente la suite des lettres de \inlinecode{c} et \inlinecode{c_num} repr\'esente la suite des chiffres de \inlinecode{c}. Effectuez un parcours de votre cha\^ine et utilisez un test conditionnel ainsi que les m\'ethodes \inlinecode{str.isalpha()} et \inlinecode{str.isdigit()} qui permettent de d\'eterminer si la cha\^ine \inlinecode{str} est compos\'ee de caract\`eres alphab\'etiques ou num\'eriques.
\item \'Ecrivez un programme qui d\'etermine si la sous-cha\^ine \inlinecode{j23} appara\^it dans votre cha\^ine, et, si c'est le cas, qui la remplace par \inlinecode{j24}. La cha\^ine de d\'epart \inlinecode{c} devra donc \^etre modifi\'ee dans ce cas. Utilisez un test conditionnel ainsi que les m\'ethodes \inlinecode{str.find()} et \inlinecode{str.replace()} qui permettent de retrouver et remplacer des sous-cha\^ines dans la cha\^ine \inlinecode{str}.
\item \'Ecrivez un programme qui d\'etermine si la sous-cha\^ine \inlinecode{f37} appara\^it dans votre cha\^ine, mais pas forc\'ement de mani\`ere cons\'ecutive. En utilisant la m\'ethode \inlinecode{str.find()}, recherchez le premier indice d'apparition du caract\`ere \inlinecode{f}, puis celui de \inlinecode{3}, puis celui de \inlinecode{7} et v\'erifiez que ces indices forment bien une suite croissante.
\end{enumerate}
\end{exercice}




\medskip
\begin{exercice}[cha\^ines de caract\`eres] \label{TD2ex2}
\

\vspace{-1.5ex}
\begin{enumerate}
\item Initialisez une variables \inlinecode{texte} contenant la cha\^ine de caract\`eres suivante:\\
\inlinecode{"We introduce here the Python language"}
\begin{enumerate}
\item Cr\'eez un script qui compte le nombre de caract\`eres de cette cha\^ine. Vous initialiserez une variable ``compteur'' \`a $0$, puis, \`a chaque caract\`ere parcouru, incr\'ementerez cette variable. V\'erifiez que votre r\'esultat corresponde bien \`a celui de l'instruction \inlinecode{len(texte)}.
\item Cr\'eez un autre script qui compte le nombre de caract\`eres de cette cha\^ine qui ne sont pas des espaces. Pensez \`a utiliser un test conditionnel dont la syntaxe est d\'ecrite ci-dessus.
\item Sachant que dans cette cha\^ine de caract\`eres, les mots sont s\'epar\'es par des espaces, cr\'eez un script qui compte le nombre de mots de cette cha\^ine. Pensez encore une fois \`a utiliser un test conditionnel.
\end{enumerate}
\item Initialisez une variables \inlinecode{texte2} contenant le texte suivant:
\begin{center}
\inlinecode{"We introduce here the Python language. To learn more about the language, consider going through the excellent tutorial https://docs.python.org/tutorial. Dedicated books are also available, such as http://www.diveintopython.net/."}\\ 
\end{center}
Dans cette cha\^ine de caract\`eres, les mots ne sont plus uniquement s\'epar\'es par des espaces, mais \'egalement par des symboles de ponctuation. Testez votre script qui compte le nombre de mots sur cette cha\^ine de caract\`eres. Votre script est-il toujours valable? Si oui, pourquoi? Sinon, comment devez-vous le modifier? On aimerait que chaque adresse web ne soit compt\'ee que comme un seul mot.
\end{enumerate}
\end{exercice}


\begin{filecontents*}{temp.tex}
\begin{solution}[TD2ex2]
\begin{lstlisting}[style=verbatim]
...
\end{lstlisting}
\end{solution}
\newpage
\end{filecontents*}
\appendsolution



\clearpage
% *** LISTES *** %
\begin{myboxi}[Rappel sur les listes] 
Les listes sont donn\'ees entre crochets, les \'el\'ements \'etant s\'epar\'es par des virgules. Par exemple: \inlinecode{ma_liste = [1, 2, 3]}

\medskip

L'acc\`es aux \'el\'ements et aux sous-listes d'une liste \inlinecode{l} s'effectue par la m\^eme syntaxe que pour les cha\^ines: on utilise les instructions  \inlinecode{l[n]}, \inlinecode{l[-n]}, \inlinecode{l[m:n]}, \inlinecode{l[m:]} et \inlinecode{l[:n]}.

\medskip

Contrairement aux cha\^ines, les listes sont des structures de donn\'ees modifiables. Cela signifie que nous pouvons proc\'eder \`a des r\'eaffectation de leurs \'el\'ements:
\begin{lstlisting}
>>> l =  [2, "abba", 3.7, True]
>>> l[2] = 101
>>> print l
[2, 'abba', 101, True]
\end{lstlisting}

\medskip

Le parcours d'une liste s'effectue via les instructions:
\begin{lstlisting}
for variable in liste:
	instructions...
\end{lstlisting}

\medskip

Il existe diverses m\'ethodes sur les listes, nous en aborderons quelques unes.
\end{myboxi}




\medskip
\begin{exercice}[Listes] \label{TD2ex3}
\

\begin{enumerate}
\item Cr\'eez un programme qui demande \`a l'utilisateur d'entrer trois mots \`a la suite, puis renvoie les trois mots tri\'es par ordre alphab\'etique. Utilisez une liste pour stocker les trois mots. Construisez et triez la liste gr\^ace aux m\'ethodes \inlinecode{append()} et \inlinecode{sort()}, respectivement.
\item Modifiez votre programme de mani\`ere \`a ce que l'utilisateur puisse entrer autant de mots qu'il le souhaite. Le processus de saisie s'arr\^ete lorsque l'utilisateur entre le mot ``FIN''. Utilisez une boucle \inlinecode{while}.
\end{enumerate}
\end{exercice}


\begin{filecontents*}{temp.tex}
\begin{solution}[TD2ex3]
\begin{lstlisting}[style=verbatim]
...
\end{lstlisting}
\end{solution}
\newpage
\end{filecontents*}
\appendsolution






\medskip
\begin{exercice}[Listes] \label{TD2ex4}
\

\noindent On consid\`ere les deux listes suivantes:
\begin{flushleft}
\inlinecode{couleurs = ["Pique", "Trefle", "Carreaux", "Coeur"]} \\
\inlinecode{valeurs = [2, 3, 4, 5, 6, 7, 8, 9, 10, "valet", "dame", "roi", "as"]} 
\end{flushleft}
\begin{enumerate}
\item \`A partir de ces deux listes, g\'en\'erer une liste \inlinecode{cartes} contenant toutes les 52 cartes sous forme de cha\^ines de caract\`eres. Utilisez un double parcours de ces listes, c'est-\`a-dire une double boucle for, ainsi que la m\'ethode \inlinecode{list.append()}.
\item Importez la fonction \inlinecode{shuffle} de la librairie \inlinecode{random} gr\^ace \`a l'instructions \inlinecode{from random import shuffle}. Cette fonction \inlinecode{shuffle()} permet de m\'elanger une liste. M\'elangez alors la liste de vos cartes.
\item Cr\'eez ensuite quatre listes \inlinecode{joueur1}, \inlinecode{joueur2}, \inlinecode{joueur3} et \inlinecode{joueur4} qui correspondent \`a la distribution votre jeu de cartes m\'elang\'e \`a quatre joueurs diff\'erents. Les cartes doivent \^etre distribu\'ees une par une et \`a tour de r\^ole. Utilisez un compteur qui compte modulo $4$ pour simuler la distribution aux joueurs \`a tour de r\^ole. Utilisez un test conditionnel avec des conditions ``elif''.
\end{enumerate}
\end{exercice}


\begin{filecontents*}{temp.tex}
\begin{solution}[TD2ex4]
\begin{lstlisting}[style=verbatim]
...
\end{lstlisting}
\end{solution}
\newpage
\end{filecontents*}
\appendsolution




\medskip
\begin{exercice}[Listes, fichiers et listes de listes] \label{TD2ex5}
\

\noindent Le fichier \inlinecode{diamonds.csv} contient des donn\'ees d'environ $54000$ diamants. Nous allons r\'ecolter ces donn\'ees sous forme de listes afin de pouvoir les manipuler. 
\begin{enumerate}
\item Copiez le fichier \inlinecode{diamonds.csv} dans le r\'epertoire dans lequel se trouve votre script actuel. En Python, il est possible de lire ce fichier et de cr\'eer une liste des lignes de ce fichier. Pour cela, utilisez instructions suivantes:
\begin{lstlisting}
with open("diamonds.csv", "r") as f:
	diamants = f.readlines()
\end{lstlisting}
La liste cr\'e\'ee s'appelle \inlinecode{diamants}. Les \'el\'ements de cette liste sont des cha\^ines de caract\`eres qui correspondent aux lignes du fichier. Quelle est la longueur de cette liste? Afficher les premier, deuxi\`eme et troisi\`eme \'el\'ements de cette liste. Remarquer que le premier \'el\'ement de la liste correspond aux variables mesur\'ees sur les diamants. Les autres lignes correspondent aux donn\'ees proprement dites.
\item On aimerait maintenant que chaque \'el\'ement de la liste soit une liste plut\^ot qu'une cha\^ine de caract\`eres. Pour cela, on utilise la m\'ethode \inlinecode{split()} qui permet de couper une cha\^ine en une liste d'\'el\'ements. Cette m\'ethode est illustr\'ee ci-dessous:
\begin{lstlisting}[style=verbatim]
>>> diamants[2]								# chaine de caracteres
'0.21,"Premium","E",59.8,326,3.89,3.84,2.31\n'
>>> diamants[2].split(",")		# chaine transformee en liste
['0.21', '"Premium"', '"E"', '59.8', '326', '3.89', '3.84', '2.31\n']
\end{lstlisting}
Testez cette commande. \'Ecrivez un programme qui applique la m\'ethode \inlinecode{split()} \`a touts les \'el\'ements de votre liste \inlinecode{diamants}. Utilisez un parcours de la liste \inlinecode{diamants} et un compteur qui rerp\'esente le num\'ero de ligne courant.
\item Cr\'eez une liste \inlinecode{diamants_100} qui contient les $100$ \'el\'ements de la liste \inlinecode{diamants} qui succ\`edent au tout premier \'el\'ement. Affichez les $20$ premiers \'el\'ements de \inlinecode{diamants_100}.
\item Cr\'eez une liste \inlinecode{diamants_prix} qui contient les prix des $54000$ diamants convertis en nombres r\'eels (type \inlinecode{float}). Utilisez un parcours de la liste \inlinecode{diamants_prix}. Affichez les $20$ premiers \'el\'ements de \inlinecode{diamants_100}.
\end{enumerate}.
\end{exercice}


\begin{filecontents*}{temp.tex}
\begin{solution}[TD2ex5]
\begin{lstlisting}[style=verbatim]
...
\end{lstlisting}
\end{solution}
\newpage
\end{filecontents*}
\appendsolution





\clearpage
% *** TUPLES *** %
\begin{myboxi}[Rappel sur les tuples] 
Les tuples sont donn\'ees entre parenth\`eses, les \'el\'ements \'etant s\'epar\'es par des virgules. Par exemple: \inlinecode{ma_tuple = (1, 2, 3)}

\medskip

L'acc\`es aux \'el\'ements et aux sous-tuples d'un tuple \inlinecode{t} s'effectue par la m\^eme syntaxe que pour les listes: on utilise les instructions  \inlinecode{t[n]}, \inlinecode{t[-n]}, \inlinecode{t[m:n]}, \inlinecode{t[m:]} et \inlinecode{t[:n]}.

\medskip

Contrairement aux listes, les tuples sont des structures de donn\'ees non modifiables. Cela signifie que nous ne pouvons pas proc\'eder \`a des r\'eaffectation de leurs \'el\'ements, ni modifiez leur logueur, etc.:
\begin{lstlisting}
>>> t = (1, 2, 3, "good", True)
>>> t[3] = "bad"
Traceback (most recent call last):
  File "<stdin>", line 1, in <module>
TypeError: 'tuple' object does not support item assignment
>>> t.append(28.45)
Traceback (most recent call last):
  File "<stdin>", line 1, in <module>
AttributeError: 'tuple' object has no attribute 'append'
\end{lstlisting}

\medskip

Le parcours d'un tuple s'effectue via les instructions:
\begin{lstlisting}
for variable in tuple:
	instructions...
\end{lstlisting}
\end{myboxi}



\medskip
\begin{exercice}[Listes de tuples] \label{TD2ex6}
\

\begin{enumerate}
\item Cr\'eez un programme qui demande \`a l'utilisateur d'entrer le pr\'enom, le nom et le num\'ero de matricule d'un \'etudiant, puis stocke ces informations dans un tuple \`a trois \'el\'ements. Affichez ce tuple \`a l'\'ecran. L'int\'er\^et d'utiliser un tuple dans ce cas r\'eside dans le fait de ne pas pouvoir modifier les informations d'un \'etudiant, ce qui est plus s\'ecuris\'e.
\item Modifiez votre programme de mani\`ere \`a ce que l'utilisateur puisse entrer autant de saisies qu'il le souhaite. Le processus de saisie s'arr\^ete lorsque l'utilisateur entre le mot ``FIN''. La liste des \'etudiants saisis sera stock\'ee dans une liste de tuples. Utilisez une boucle \inlinecode{while}.
\item Affichez de mani\`ere conviviale tous les \'etudiants de votre liste construite au point pr\'ec\'edent. Utilisez un parcours de liste.
\end{enumerate}
\end{exercice}


\begin{filecontents*}{temp.tex}
\begin{solution}[TD2ex6]
\begin{lstlisting}[style=verbatim]
...
\end{lstlisting}
\end{solution}
\newpage
\end{filecontents*}
\appendsolution






\clearpage
% *** DICTIONNAIRES *** %
\begin{myboxi}[Rappel sur les dictionnaires] 
Les dictionnaires sont des couples d'\'el\'ements ``cl\'e-valeur'' donn\'es entre accolades; les cl\'es sont s\'epar\'ees de leurs valeurs correspondantes par des double points, et les couples ``cl\'e-valeur'' sont s\'epar\'es entre eux par des virgules. Par exemple:
\begin{lstlisting}
mon_dico = {"FR" : 643801, "DE" : 357168, "GB" : 229848}
\end{lstlisting}

\medskip

L'acc\`es \`a la valeur correspondante \`a la cl\'e \inlinecode{key} d'un dictionnaire \inlinecode{d} s'effectue via l'instruction \inlinecode{d[key]}. L'affectation d'une valeur \inlinecode{value}  \`a une cl\'e \inlinecode{key} d'un dictionnaire \inlinecode{d} s'effectue via l'instruction \inlinecode{d[key] = value}.

\medskip

Tout comme les listes, les dictionnaires sont des structures de donn\'ees modifiables. Cela signifie que nous pouvons proc\'eder \`a des r\'eaffectation de leurs \'el\'ements:
\begin{lstlisting}
>>> mon_dico = {"FR" : 643801, "DE" : 357168, "GB" : 229848}
>>> mon_dico["DE"] = 257200
>>> mon_dico
{'DE': 257200, 'GB': 229848, 'FR': 643801}
>>> mon_dico["BE"] = 30528
>>> mon_dico
{'DE': 257200, 'GB': 229848, 'BE': 30528, 'FR': 643801}
>>> del mon_dico["FR"]
>>> mon_dico
{'DE': 257200, 'GB': 229848, 'BE': 30528}
\end{lstlisting}

\medskip

Le parcours des cl\'es, des valeurs, ou des couples ``cl\'e-valeur'' d'un dictionnaire s'effectue via les quatre types d'instructions suivantes:
\begin{lstlisting}
# par defaut, le parcours d'un dico correspond parcours de ses cles
# dans le cas ci-dessous, "variable" prend les valeurs de cles du dico
for variable in dico:
	instructions...
# la syntaxe suivante est plus precise mais equivalente
for variable in dico.keys():
	instructions...
# pour le parcours des valeurs d'un dico
for variable in dico.values():
	instructions...
# pour le parcours des couples "cle-valeur" d'un dico
for variable_cle, variable_valeur in dico.items():
	instructions...
\end{lstlisting}

\medskip

Il existe diverses m\'ethodes sur les dictionnaires, nous en aborderons quelques unes.
\end{myboxi}






\medskip
\begin{exercice}[Dictionnaires] \label{TD2ex7}
\

\noindent Cet exercice correspond \`a une g\'en\'eralisation de l'exercice $6$ dans le cas des dictionnaires.
\begin{enumerate}
\item Cr\'eez un dictionnaire qui contient comme cl\'es un certain nombres de mots en fran\c{c}ais et comme valeurs la traduction de ces mots en anglais. Ce dictionnaire fait donc office de traducteur fran\c{c}ais-anglais.
\item Ajouter \`a votre dictionnaire le couple cl\'e-valeur \inlinecode{"cerveau" : "brain"}.
\item Recherchez si votre dictionnaire contient la traduction du mot ``cerveau'', et si tel est, le cas, affichez sa traduction anglaise. Effectuez un parcours par cl\'es de votre dictionnaire.
\item Cr\'eez un nouveau dictionnaire dont les cl\'es et valeurs correspondent aux valeurs et aux cl\'es de votre dictionnaire pr\'ec\'edent. Ainsi, votre nouveau dictionnaire fera office de traducteur anglais-fran\c{c}ais au lieu de fran\c{c}ais-anglais. Pour construire ce nouveau dictionnaire, utilisez un parcours cl\'e-valeur de votre dictionnaire initial.
\item Recherchez si votre dictionnaire contient la traduction du mot ``brain''. 
\item Recherchez si votre dictionnaire contient la valeur ``cerveau'' et si tel est le cas, afficher sa cl\'e correspondante. Effectuez un parcours cl\'es-valeurs de votre dictionnaire.
\item Modifiez votre dictionnaire de d\'epart de sorte que les valeurs ne soient plus des simples mots anglais, mais des listes de mots correspondant \`a autant de traductions possibles de vos cl\'es.
\item Ajouter \`a votre nouveaux dictionnaire le couple cl\'e-valeur \inlinecode{"chemin" : ["path", "way"]}.
\item Recherchez la deuxi\`eme traduction du mot ``chemin''.
\item Effacez la cl\'e ``chemin'' et sa valeur correspondante de votre dictionnaire (fonction \inlinecode{del}).
\end{enumerate}
\end{exercice}


\begin{filecontents*}{temp.tex}
\begin{solution}[TD2ex7]
\begin{lstlisting}[style=verbatim]
...
\end{lstlisting}
\end{solution}
\newpage
\end{filecontents*}
\appendsolution






\medskip
\begin{exercice}[Dictionnaire de tuples] \label{TD2ex7}
\

\noindent Cet exercice correspond \`a une g\'en\'eralisation de l'exercice $6$ dans le cas des dictionnaires.
\begin{enumerate}
\item Cr\'eez un programme qui demande \`a l'utilisateur d'entrer le pr\'enom, le nom et le num\'ro de matricule d'un \'etudiant, puis stocke ces informations dans un dictionnaire. Les cl\'es du dictionnaires correspondront aux noms des \'etudiants et ses valeurs seront des tuples \`a trois \'el\'ements (pr\'enom, nom, matricule). Affichez ce dictionnaire \`a l'\'ecran. L'int\'er\^et d'utiliser un dictionnaire dans ce cas r\'eside dans le fait de pouvoir acc\'eder aux informations des \'etudiants \`a leur noms.
\item Modifiez votre programme de mani\`ere \`a ce que l'utilisateur puisse entrer autant de saisies qu'il le souhaite. Le processus de saisie s'arr\^ete lorsque l'utilisateur entre le mot ``FIN''. Le dictionnaire des \'etudiants saisis sera stock\'ee dans une liste de tuples. Utilisez une boucle \inlinecode{while}.
\item Affichez de mani\`ere conviviale tous les \'etudiants de votre dictionnaire construit au point pr\'ec\'edent. Utilisez un parcours par cl\'es de votre dictionnaire.
\item En utilisant la syntaxe \inlinecode{in} ou la m\'ethode \inlinecode{has_key}\footnote{Cette m\'ethode ne fonctionne plus en Python 3.}, d\'eterminer si un \'etudiant du nom de ``Obama'' appartient \`a votre dictionnaire, et, si tel est le cas, renvoyez les informations cet l\'etudiant.
\item Cr\'eez un script qui d\'etermine si le num\'ero de matricule $12345678$ existe dans votre dictionnaire, et si tel est le cas, renvoyez les informations de l\'etudiant correspondant. Utilisez un parcours ``cl\'e-valeur'' de votre dictionnaire.
\item Question subsidiaire. Dans l'\'etat actuel de votre programme, il est impossible d'enregistrer deux \'etudiants qui portent le m\^eme nom. Pourquoi cela? Modifier votre programme du point $2$ de mani\`ere \`a pouvoir enregistrer des \'etudiants portant le m\^eme nom. Pour cela, les cl\'es de votre dico ne devront plus correspondre uniquement au nom des \'etudiants: mais attention, les cl\'es ne peuvent pas \^etre des listes, tuples, etc.
\end{enumerate}
\end{exercice}


\begin{filecontents*}{temp.tex}
\begin{solution}[TD2ex7]
\begin{lstlisting}[style=verbatim]
...
\end{lstlisting}
\end{solution}
\newpage
\end{filecontents*}
\appendsolution






%\begin{filecontents*}{temp.tex}
%
%\begin{solution}[ex:]
%
%\end{solution}
%
%
%\end{filecontents*}
%\appendsolution


     % types de données
\clearpage
\section{Chapitre 3: Expressions arithmétiques, entrées/sorties}

\setcounter{exercice}{0}

\bigskip

%%%%%%%%
%%%%%%%%
%%%%%%%%



\medskip
\begin{myboxi}[Le concile de Nicée]
La définition de la date de Pâques a été
 fixée en 325 lors du concile de Nicée :
P\^aques est le dimanche qui suit le quatorzième jour de la lune qui atteint cet âge au 21 mars ou immédiatement après.

Autrement dit, c'est le premier dimanche qui suit ou qui coïncide avec la première pleine lune apres le 21 mars (marquant le début du printemps).
\end{myboxi}


\medskip
\begin{myboxi}[Algorithme de Oudin pour calculer la date de Pâques]
Calculer la date de Pâques est loin d'être une chose facile. On connait plusieurs méthodes, 
dont la suivante due à Oudin qui a l'avantage de
demander peu d'opérations. Quoiqu'il existe une version généraliste de l'agorithme de Oudin
(sans limite de siècle), nous présentons ici une forme simplifiée uniquement valable pour le
calendrier grégorien, donc pour {\textbf{toute année postérieure à 1583}}.


\medskip
\medskip

Dans l'algorithme ci-dessous, chaque ligne introduit une nouvelle variable qui dépend 
des précédentes. Les divisions doivent toujours être entières. L'exemple en bleu indique les calculs pour l'année 2007.
\medskip

\begin{enumerate}[label=\arabic*.]
\item G représente l'écart d'or diminué de 1 : diviser l'année par 19, en prendre le reste ;  
\color{blue} (2007/19 = 105, or 105 × 19 = 1995 et il nous faut 2007, donc l'écart vaut G = 12) \color{black}
\item C et C4 permettent le suivi des années bissextiles : diviser l'année par 100 puis encore par 4 ;  
\color{blue} (2007/100 = C = 20 et 20/4 = C4 = 5) \color{black}
\item E : diviser 8×C +13 par 25 sans les décimales;  
\color{blue} (8 × 20 + 13 = 173/25 = E = 6) \color{black}
\item H qui dépend de l'épacte : diviser 19 × G + C - C4 - E + 15 par 30, en prendre le reste ; 
\color{blue} (on prend le reste d'une division selon le même principe que pour G : 252/30 = 8, or 8 × 30 = 240 et il nous faut 252, donc l'écart vaut H = 12) \color{black}
\item K : diviser H par 28 ; 
\color{blue} (12/28 = K = 0) \color{black}
\item P : diviser 29 par H + 1 ; 
\color{blue} (29/13 = P = 2) \color{black}
\item Q : diviser 21-G par 11; 
\color{blue} ( 21 - 12 = 9/11 = Q = 0) \color{black}
\item I représente le nombre de jours entre la pleine lune pascale et le 21 mars : (-1 + K × P × Q) × K + H ; 
\color{blue} ( 0 × 2  × 0 - 1 = -1  × -0 = 0 + 12 = I = 12) \color{black}
\item B : diviser l'année par 4 et enlever les décimales, y ajouter l'année; 
\color{blue} (2007/4 = 501 + 2007 = 2508) \color{black}
\item J1 : Additionner B + I + 2 + C4 et retrancher C; 
  \color{blue} (J1 =2507) \color{black}
\item J2 calcule le jour de la lune pascale (0=dimanche 1=lundi...6=samedi) : diviser J1 par 7 et en prendre le reste ; 
\color{blue} (on calcule toujours le reste d'une division selon le même principe qu'avec G et H, le résultat est J2 = 1) \color{black}
\item R = 28 + I - J2. C'est le résultat final, enfin !
\color{blue} (R = 39)  \color{black}
\item R représente la date de Pâques dans le mois de mars, s'il dépasse 31 cela signifie que l'on déborde sur avril (... 30 correspond au 30 mars, 31 au 31 mars, 32 au 1er avril, 33 au 2 avril, ...). Retrancher 31 le cas échéant pour obtenir la date d'avril. 
\color{blue} (Pâques 2007 tombe donc le 8 avril) \color{black}
\end{enumerate}
\end{myboxi}

\newpage

\medskip
\begin{exercice}[Date de Pâques]\label{TD3_ex1}
\

\begin{enumerate}
\item Calculez la date de Pâques pour l'année 2007 en suivant la méthode de Oudin.
\item Debugguez au fur et à mesure les valeurs des variables avec la fonction 
\inlinecode{print} pour vérifier votre implémentation de l'algorithme à l'aide de l'exemple bleu.
\
\begin{lstlisting}
 G = ...
 C = ...
 C4 = ...
 print( f"G = {G}" )
 print( f"C = {C}" )
 print( f"C4 = {C4}" )
\end{lstlisting}
\end{enumerate}

\end{exercice}

\medskip
\begin{exercice}[Utilisation de fonctions et de fichiers]\label{TD3_ex2}
\

\begin{enumerate}
\item Modifiez le script pour mettre le calcul de Pâques
dans une fonction \inlinecode{date_de_paques(N)}. Votre fonction doit renvoyer une
chaine de caractères, par exemple "\inlinecode{Dimanche 23 mars 2008}"
\item Réalisez un programme qui demande une année à l'utilisateur dans le programme
principal, puis affiche la date de Pâques pour cette année.
\item Quelle sera la prochaine année (à partir de 2025) où Pâques sera fêté un 1er avril ?
\item Réalisez un programme qui affiche toutes les années du XXIe siècle durant lesquelles
Pâques est fêté une 1er avril.
\item Déterminez les années du XXIe siècle pour lesquelles Pâques est fêté le plus tard.
\item \'Ecrivez toutes les dates de Pâques du XXIe siècle dans un fichier texte.
\
\begin{lstlisting}
 with open("toto.txt", "w") as f:
     f.write("I have written something in the file !!!")
\end{lstlisting}

\item Déterminez le jour de votre anniversaire (lundi, mardi, mercredi, etc) pour une année N
donnée (indication : Pâques est un dimanche).
\end{enumerate}

\end{exercice}




%\begin{filecontents*}{temp.tex}
%
%\begin{solution}[ex:]
%
%\end{solution}
%
%
%\end{filecontents*}
%\appendsolution


     % boucles, tests
\clearpage
\section{Chapitre 4: Boucles et tests conditionnels}

\setcounter{exercice}{0}

\bigskip

%%%%%%%%
%%%%%%%%
%%%%%%%%

\medskip
\begin{myboxi}[Rappel sur les boucles \inlinecode{for}]
La fonction \inlinecode{range(start,stop,[step])} produit un objet qui retourne une liste de nombres entiers telle que: la premi\`ere valeur est \inlinecode{start} (\inlinecode{start} vaut $0$ par d\'efaut); les valeurs suivantes sont incr\'ement\'ees par pas de \inlinecode{step} (\inlinecode{step} vaut $1$ par d\'efaut); et ce jusqu'\`a atteindre la valeur \inlinecode{stop} non incluse.

\smallskip

Une boucle \inlinecode{for} signifie ``pour \inlinecode{var} allant de \inlinecode{start} \`a \inlinecode{stop} par pas de \inlinecode{step}, effectuer les instructions...''. La syntaxe d'une boucle \inlinecode{for} utilise la fonction \inlinecode{range()} et se pr\'esente comme suit:
\begin{lstlisting}
for var in range(start,stop[,step]):
	instructions...
\end{lstlisting}
On peut forcer l'interruption d'une boucle \inlinecode{for} avec l'instruction \inlinecode{break}.

\smallskip

On rappelle \'egalement la syntaxe pour parcourir les cha\^ines de caract\`eres, les listes, les tuples et les dictionnaires \`a l'aide de boucles \inlinecode{for}.
\begin{lstlisting}
# parcours de chaines, listes ou tuples
for variable in chaine_liste_ou_tuple:
	instructions...
# parcours des cles d'un dico
for variable in dico.keys(): # (ou for variable in dico:)
	instructions...
# parcours des valeurs d'un dico
for variable in dico.values():
	instructions...
# parcours des couples "cle-valeur" d'un dico
for variable_cle, variable_valeur in dico.items():
	instructions...
\end{lstlisting}
\end{myboxi}


\medskip
\begin{myboxi}[Rappel sur les boucles \inlinecode{while}]
Une boucle \inlinecode{while} signifie ``tant que la condition est vraie, effectuer les instructions...''. La syntaxe d'une boucle \inlinecode{while} se pr\'esente comme suit:
\begin{lstlisting}
while condition:
	instructions...
\end{lstlisting}
On peut forcer l'interruption d'une boucle \inlinecode{while} avec l'instruction \inlinecode{break}.
\end{myboxi}


\medskip
\begin{myboxi}[Rappel sur les tests conditionnels \inlinecode{if... else...}]
On rappelle la syntaxe d'un test conditionnel \inlinecode{if... else...} (les blocs \inlinecode{elif} et \inlinecode{else} sont facultatifs):
\begin{lstlisting}
if condition:
	instructions...
elif:
	instructions...
...
else:
	instructions...
\end{lstlisting}
\end{myboxi}



\medskip
\begin{exercice}[Boucles ``for'']\label{TD3_ex1}
\

\begin{enumerate}
\item \'Ecrivez un script qui demande \`a l'utilisateur d'entrer un entier $N$ et affiche ensuite la table de multiplication de $N$.
\item Modifiez votre script de sorte que la table de multiplication soit affich\'ee sur une seule ligne. Il suffit d'ajouter une virgule apr\`es votre instruction \inlinecode{print}.
\item Modifiez votre script de telle mani\`ere qu'il affiche, line par ligne, les tables de multiplications de tous les entiers plus petit ou \'egaux \`a $N$. Utilisez une double boucle \inlinecode{for}.
\item \'Ecrivez un script qui demande \`a l'utilisateur d'entrer un entier $N$ et affiche ensuite un petit dessin comme ci-dessous de hauteur $N$.
\begin{lstlisting}[style = verbatim]
# exemple pour N = 4
*
**
***
****
\end{lstlisting}
\item Essayez de modifier votre script de telle sorte que le dessin affich\'e soit comme ci-dessous. D\'eterminez r\`egle qui donne le nombre d'espaces et d'\'etoiles de chaque ligne.
\begin{lstlisting}[style = verbatim]
# exemple pour N = 5
    *
   * *
  * * *
 * * * *
* * * * *
\end{lstlisting}
\end{enumerate}
\end{exercice}

\begin{filecontents*}{temp.tex}
\newpage
\begin{solution}[TD3_ex1]
\begin{lstlisting}
...
\end{lstlisting}
\end{solution}
\newpage
\end{filecontents*}
\appendsolution





\medskip
\begin{exercice}[Boucles ``for'']\label{TD3_ex2}
\

\noindent On consid\`ere les deux listes suivantes:
\begin{flushleft}
\inlinecode{jours = [31, 28, 31, 30, 31, 30, 31, 31, 30, 31, 30, 31]} \\
\inlinecode{mois = ["January", "February", "March", "April", "May", "June", "July", "August", "September", "October", "November", "December"]} 
\end{flushleft}
\begin{enumerate}
\item \'Ecrivez un script qui cr\'ee la liste de $12$ tuples suivante:
\begin{flushleft}
\inlinecode{mj = [("January", 31), ("February", 28), ("March", 31), ("April", 30), ("May", 31), ("June", 30), ("July", 31), ("August", 31), ("September", 30), ("October", 31), ("November", 30), ("December", 31)]} 
\end{flushleft}
\item \'Ecrivez un script qui cr\'ee une liste de $365$ cha\^ines de caract\`eres correspondant aux $365$ jours de l'ann\'ee, comme ci-dessous. Utilisez une double boucle \inlinecode{for} sur votre liste \inlinecode{mj}.
\begin{flushleft}
\inlinecode{annee = ["1 January", "2 January",..., "1 February", "2 February",...]} 
\end{flushleft}
\item Initialisez la liste suivante:
\begin{flushleft}
\inlinecode{jours_semaine = ["Monday", "Tuesday", "Wednesday", "Thursday", "Friday", "Saturday", "Sunday"]} \\
\end{flushleft}
Supposons que le premier janvier soit un lundi. \'Ecrivez un script qui cr\'ee une deuxi\`eme liste des $365$ jours de l'ann\'es comprenant les jours de la semaine en plus, comme ci-dessous. Effectuez une double boucle de taille $365$, utilisez les listes \inlinecode{annee} et \inlinecode{jours_semaine} et pensez \`a utiliser l'op\'erateur modulo sur les indices de la liste \inlinecode{jours_semaine}. 
\begin{flushleft}
\inlinecode{annee2 = ["Monday 1 January", "Tuesday 2 January",...]} 
\end{flushleft}
\item \`A partir de vos listes \inlinecode{annee} et \inlinecode{jours_semaine} , cr\'eez un dictionnaire dont les cl\'es sont les \'el\'ements de \inlinecode{annee} et les valeurs sont les jours de la semaine correspondants, comme ci dessous.
\begin{flushleft}
\inlinecode{dico_annee = ["1 January" : "Monday", "2 January" : "Tuesday",...]} 
\end{flushleft}
\item \`A quel jour de la semaine correspond au $28$ octobre?
\end{enumerate}
\end{exercice}

\begin{filecontents*}{temp.tex}
\newpage
\begin{solution}[TD3_ex2]
\begin{lstlisting}
...
\end{lstlisting}
\end{solution}
\newpage
\end{filecontents*}
\appendsolution



\medskip
\begin{exercice}[Boucles ``while''] \label{TD3ex3}
\

\noindent 
\begin{enumerate}
\item \'Ecrivez un programme qui demande \`a l'utilisateur d'entrer $3$ notes entre $0$ et $20$. Le programme affichera ensuite le minimum, le maximum et la moyenne des notes entr\'ees. Utilisez une boucle \inlinecode{while}. Les notes entr\'ees seront stock\'ees progressivement dans une liste. Utilisez un parcours de votre liste de notes pour calculer la moyenne de celles-ci.
\item Modifiez le programme de sorte que celui-ci commence par demander \`a l'utilisateur le nombre de notes $N$ qu'il d\'esire entrer, puis proc\`ede comme pr\'ec\'edemment.
\item Modifiez le programme de sorte que l'utilisateur puisse entrer autant de notes qu'il le d\'esire et termine sa saisie par une certaine instruction, comme ``fin'' par exemple.
\end{enumerate}


\end{exercice}


\begin{filecontents*}{temp.tex}
\begin{solution}[TD3ex3]
\begin{lstlisting}[style=verbatim]
...
\end{lstlisting}
\end{solution}
\newpage
\end{filecontents*}
\appendsolution





\medskip
\begin{exercice}[Boucles ``while''] \label{TD3ex4}
\

\noindent \'Ecrivez un programme qui choisit un nombre entier au hasard entre $1$ et $100$, puis demande \`a l'utilisateur de le deviner. Si l'utilisateur entre un nombre trop petit ou trop grand, le programme devra afficher ``plus haut'' ou ``plus bas'', respectivement. Le programme s'arr\^ete lorsque le nombre a \'et\'e trouv\'e. Pour g\'en\'erer un nombre entier au hasard, importez la librairie \inlinecode{random} et utilisez la fonction \inlinecode{randint()}.
\end{exercice}


\begin{filecontents*}{temp.tex}
\begin{solution}[TD3ex4]
\begin{lstlisting}[style=verbatim]
...
\end{lstlisting}
\end{solution}
\newpage
\end{filecontents*}
\appendsolution





%\begin{filecontents*}{temp.tex}
%
%\begin{solution}[ex:]
%
%\end{solution}
%
%
%\end{filecontents*}
%\appendsolution


     % fonctions
\clearpage
\section{Chapitre 5: Fonctions }

\setcounter{exercice}{0}

\bigskip

%%%%%%%%
%%%%%%%%
%%%%%%%%

\medskip
\begin{myboxi}[Rappel sur les fonctions et les proc\'edures]
La syntaxe pour d\'efinir une fonction:
\begin{lstlisting}
def nom_fonction(arg_1,...,arg_N):
	instructions...
	return valeur   # dans le cas d'une procedure, simplement return
\end{lstlisting}
\end{myboxi}

\medskip
\begin{myboxi}[Paramètres par défaut dans les fonctions et les procédures]
La syntaxe pour d\'efinir une fonction avec des arguments qui poss\`edent des valeurs par d\'efaut est la suivante:
\begin{lstlisting}
def nom_fonction(arg_1 = val_1,...,arg_N = val_N):
	instructions...
	return valeur # dans le cas d'une procedure, simplement return
\end{lstlisting}
\end{myboxi}

\medskip
\begin{exercice}[Calcul d'intégrale par la méthode des rectangles]\label{TD5_ex1}
\
\begin{enumerate}
\item Montrez que l'équation $x^2 + y^2 = 1$ est l'équation du cercle de centre $(0,0)$ et de rayon 1.
En déduire l'équation développée du cercle de centre $(1,0)$ et de rayon 1 en faisant le changement de variable $x'=x-1$.
\item Soit $f$ la fonction $f(x) = \sqrt{2x - x^2}$. Quels sont l'ensemble de définition et le graphe de $f$ ? 
Vérifiez vos résultats en utilisant la calculatrice graphique du site \url{http://desmos.com}
\item Soit $g$ la fonction définie par $g(x) = f(|x|)$. Quel est l'ensemble de définition de $g$ ? Tracez son graphe
sur \url{http://desmos.com}
\item Expliquez pourquoi $$\int_{-2}^{2}g(x)dx = \pi$$
\item Programmez en Python les deux fonctions $f(x)$ et $g(x)$.
\item Définissez une fonction \inlinecode{integrale(phi, a, b, n)} qui renvoie une approximation de
l'intégrale de la fonction $\phi$ sur $[a,b]$ par la méthode des rectangles avec $n$ subdivisions.
On pourra consulter \url{https://fr.wikipedia.org/wiki/Somme_de_Riemann} et utiliser la méthode du point médian.
\item Appliquez la question précédente à la fonction $g$ sur $[-2,2]$ et en déduire une approximation numérique de $\pi$.
\end{enumerate}
\end{exercice}


\begin{filecontents*}{temp.tex}
\begin{solution}[TD3ex4]
\begin{lstlisting}[style=verbatim]
...
\end{lstlisting}
\end{solution}
\newpage
\end{filecontents*}
\appendsolution


\medskip
\begin{exercice}[fonctions, paramètres par défaut]\label{TD5_ex2}
\
\begin{enumerate}
\item D\'efinissez une fonction \inlinecode{surf_cercle()} qui calcule la surface d'un cercle dont le rayon est donn\'e en argument. Pour avoir acc\`es \`a la valeur de $\pi$, ajouter l'instruction \inlinecode{from math import pi}.
\item Modifiez votre fonction \inlinecode{surf_cercle} de sorte que votre argument poss\`ede une valeur par d\'efaut de $1$.
\item D\'efinissez une fonction \inlinecode{vol_boite()} qui renvoie le volume d'une bo\^ite parall\'el\'epip\'edique dont on fournit les trois dimensions en arguments.
\item Modifiez votre fonction de telle sorte que si un seul argument est fourni, la bo\^ite est consid\'er\'ee comme cubique (l'argument \'etant l'ar\^ete de ce cube); si deux arguments sont fournis, la bo\^ite est consid\'er\'ee comme un parall\'el\'epip\`ede \`a base carr\'ee (le premier argument est le c\^ot\'e du carr\'e, et le second la hauteur du parall\'el\'epip\`ede); si trois arguments sont fournis, la bo\^ite est consid\'er\'ee comme un parall\'el\'epip\`ede g\'en\'eral. Pour cela, donnez des valeurs par d\'efaut \`a vos arguments et utiliser un test conditionnel pour savoir si tel ou tel param\`etre poss\`ede telle ou telle valeur par d\'efaut...
\end{enumerate}
\end{exercice}

\medskip
\begin{exercice}[fonctions sur les chaines de caractères]\label{TD5_ex2}
\
\begin{enumerate}
\item D\'efinissez une fonction \inlinecode{remplacement()} qui prend trois arguments \inlinecode{c1}, \inlinecode{c2} et \inlinecode{ch} et qui remplace tous les caract\`eres \inlinecode{c1} par des caract\`eres \inlinecode{c2} dans la cha\^ine caract\`eres \inlinecode{ch}.
\item Modifiez votre fonction de telle sorte que si les arguments \inlinecode{c1}, \inlinecode{c2} et \inlinecode{ch} ne sont pas sp\'ecifi\'es, alors ils prendront les valeurs \inlinecode{" "}, \inlinecode{"*"} et \inlinecode{""}, respectivement.
\end{enumerate}
\end{exercice}

\begin{filecontents*}{temp.tex}
\newpage
\begin{solution}[TD5_ex2]
\begin{lstlisting}
...
\end{lstlisting}
\end{solution}
\newpage
\end{filecontents*}
\appendsolution



\medskip
\begin{exercice}[comptage de lexèmes]\label{TD4_ex2}
\

\begin{enumerate}
\item En vous inspirant de l'exercice $2$ du TD $2$, cr\'eez deux fonctions \inlinecode{nb_car(chaine)} et \inlinecode{nb_mots(chaine)} qui comptent le nombre de caract\`eres et le nombre de mots d'une cha\^ine de caract\`eres \inlinecode{chaine} donn\'ee en argument.
\item Comparez votre solution avec l'usage de la fonction \inlinecode{str.split()} disponible dans la librairie
standard de Python; dont la documentation est par exemple dans \url{https://docs.python.org/3/library/stdtypes.html}.
\end{enumerate}
\end{exercice}

\begin{filecontents*}{temp.tex}
\newpage
\begin{solution}[TD4_ex2]
\begin{lstlisting}
...
\end{lstlisting}
\end{solution}
\newpage
\end{filecontents*}
\appendsolution



%\begin{filecontents*}{temp.tex}
%
%\begin{solution}[ex:]
%
%\end{solution}
%
%
%\end{filecontents*}
%\appendsolution


     % programmation structurée
\clearpage
\section{Chapitre 6: Programmation structurée}

\setcounter{exercice}{0}

\bigskip


%%%%%%%%
%%%%%%%%
%%%%%%%%




\medskip
\begin{exercice}[table de multiplication]\label{TD4_ex3}
\
\begin{enumerate}
\item D\'efinissez une fonction \inlinecode{TableMult(n, debut = 1, fin = 10)} qui imprime la table de multiplication de \inlinecode{n} entre les multiplicateurs \inlinecode{debut} et \inlinecode{fin}.
\item Am\'eliorez votre fonction de telle sorte que si l'argument \inlinecode{debut} et plus grand que \inlinecode{fin}, la table de multiplication soit affich\'ee correctement, dans l'ordre d\'ecroissant.
\item Cr\'eez une fonction similaire o\`u les param\`etres \inlinecode{n}\inlinecode{debut} et \inlinecode{fin} sont demand\'es \`a l'utilisateur plut\^ot que pass\'es en argument.
\end{enumerate}
\end{exercice}

\begin{filecontents*}{temp.tex}
\newpage
\begin{solution}[TD4_ex3]
\begin{lstlisting}
...
\end{lstlisting}
\end{solution}
\newpage
\end{filecontents*}
\appendsolution



\medskip
\begin{exercice}[tour de magie]\label{TD4_ex3}
\

\noindent Le but de cet exercice est d'implémenter un tour de magie sur un jeu de 52 cartes.
\medskip
\begin{enumerate}
\item Parmi les structures de données : entier, flottant, couple, chaine de caractères, liste et dictionnaire,
quelle est celle qui vous parait la plus adaptée pour représenter l'ensemble des couleurs d'un jeu de cartes 
dans l'ordre Trèfle, Carreau, Coeur, Pique ? Même question pour les hauteurs des cartes de l'as au roi.
\item Après avoir défini les ensembles de hauteurs et de couleurs en Python, définissez les deux fonctions 
\inlinecode{hauteur(carte)} et \inlinecode{couleur(carte)} qui renvoient respectivement l'index de la hauteur et de la couleur d'une carte dans vos ensembles. Faites attention que les index commencent à zéro en Python. Par exemple, on a:
\begin{lstlisting}[style = verbatim]
hauteur("valet de carreau") = 10
couleur("valet de carreau") = 1
\end{lstlisting}
\item Définissez la fonction \inlinecode{numero(carte)} qui renvoie le numéro d'une carte (un entier entre 0 et 51) dans l'ordre canonique : tous les Trèfles, tous les Carreaux, tous les Coeurs puis tous les Piques.
\begin{lstlisting}[style = verbatim]
numero("valet de carreau") = 23
\end{lstlisting}
\item Définir la bijection réciproque \inlinecode{carte(numero)} qui prend un numéro en argument et renvoie la carte correspondante. Testez votre implémentation en vérifiant que l'on a bien la relation de composition \inlinecode{n = numero(carte(n))} pour tout $n$.
\begin{lstlisting}[style = verbatim]
carte(23) = "le valet de carreau"
\end{lstlisting}
\item Programmez la fonction entree(prompt) qui demande dans la console une carte à l'utilisateur et renvoie la carte tapée par l'utilisateur, sous forme de chaîne de caractères. Votre fonction pourra être robuste et redemander la carte si l'ordinateur n'a pas compris.
\item Implémentez le tour de magie "The Fitch Cheney Five-Card Trick"
\begin{lstlisting}[style = verbatim]
saisir la premiere carte : le 10 de pique
saisir la deuxieme carte : le 4 de coeur
saisir la troisieme carte : le 10 de carreau
saisir la quatrieme carte : le roi de trefle

... hmm, je pense que la cinquieme carte est le 3 de pique
\end{lstlisting}
\end{enumerate}
\end{exercice}

\newpage


\medskip
\begin{exercice}[le jeu ``Motus'']\label{TD4_ex4}
\

\noindent Soit la liste de mots de cinq lettres suivantes:
\begin{flushleft}
\inlinecode{mots = ["arbre", "grave", "piece", "nuage", "crane", "sonne", "table", "herbe", "ecrou", "mulet"]} 
\end{flushleft}
Cr\'ez un programme qui impl\'emente le jeu ``Motus'' avec cette liste de mots. Plus pr\'ecis\'ement, au d\'epart, l'ordinateur choisit un mot au hasard dans cette liste (utilisez la fonction \inlinecode{randint()} de la librairie \inlinecode{random}). Ensuite, l'utilisateur dispose de $10$ essais pour deviner le mot. Pour cela, il proposera des lettres tour \`a tour. \`A chaque fois, l'ordinateur l'informera de si cette lettre est pr\'esente ou non dans le mot, et, si tel est le cas, mettra \`a jour les lettres d\'ej\`a devin\'ees. L'ex\'ecution de votre programme devra ressembler \`a quelque chose comme ci-dessous:
\begin{lstlisting}[style = verbatim]
Essai 6
Entrer une lettre: t

...et

Essai 7
Entrer une lettre: m

m..et
\end{lstlisting}
\begin{enumerate}
\item Commencez par d\'efinir une fonction \inlinecode{remplace(i, c, ch)} qui renvoie une cha\^ine dans laquelle le \inlinecode{i}-\`eme caract\`ere de la cha\^ine \inlinecode{ch} est remplac\'e par \inlinecode{c}.
\item Utilisez ensuite cette fonction dans votre programme. \`A chaque lettre propos\'ee par l'utilisateur, effectuez un parcours du mot \`a deviner, et, parall\`element, construisez un nouveau mot dans lequel les lettres non encore devin\'ees apparaissent comme des points alors que celles d\'ej\`a devin\'ees apparaissent clairement.
\item Modifiez votre programme de telle sorte que votre jeu soit impl\'ement\'e dans une proc\'edure \inlinecode{motus()} \`a un argument; cet argument devra correspondre au nombre d'essais maximum auquel l'utilisateur a le droit et aura une valeur par d\'efaut de $10$.
\end{enumerate}
\end{exercice}


\begin{filecontents*}{temp.tex}
\newpage
\begin{solution}[TD4_ex4]
\begin{lstlisting}
...
\end{lstlisting}
\end{solution}
\newpage
\end{filecontents*}
\appendsolution




\medskip
\begin{exercice}[le jeu ``Puissance $4$'' (long et plus difficile; tr\`es bon exercice r\'ecapitulatif)]\label{TD4_ex5}
\

\noindent Programmez un jeu ``Puissance $4$'' sur une grille de taille $N$. Le jeu devra avoir l'apparence comme montr\'e ci-dessous:
\begin{lstlisting}[style = verbatim]
. 	. 	. 	. 	. 	. 	. 	. 	

. 	. 	. 	. 	X 	. 	. 	. 	

. 	. 	. 	O 	. 	. 	. 	. 	

O 	. 	. 	. 	. 	. 	X 	. 	

. 	. 	. 	. 	. 	. 	. 	. 	

. 	. 	. 	O 	. 	. 	. 	. 	

. 	. 	. 	. 	. 	X 	. 	. 	

. 	. 	. 	. 	. 	. 	. 	. 	

Joueur 1: quelle position (ligne, colonne)?
\end{lstlisting}
\begin{enumerate}
\item Programmez une fonction \inlinecode{generer_grille(N = 4)} qui g\'en\`ere une grille d'une certaine taille \inlinecode{N} valant $4$ par d\'efaut. Chaque ligne de la grille est une liste de taille \inlinecode{N} dont les \'el\'ements sont des cha\^ines \inlinecode{"."}. La grille est alors repr\'esent\'ee comme la liste de ses lignes, comme illustr\'e ci-dessous.
\begin{lstlisting}[style = verbatim]
# exemple d'une grille de taille 3 x 3
# on a 3 listes de taille 3 remplies avec des "."
[[".", ".", "."], [".", ".", "."], [".", ".", "."]]
\end{lstlisting}
\item Programmez une fonction \inlinecode{placer_pion(joueur, grille, position)} qui permet de placer le pion du joueur \inlinecode{joueur} \`a la position \inlinecode{position} dans la grille \inlinecode{grille}. L'argument \inlinecode{joueur} vaut \inlinecode{1} ou \inlinecode{2}. S'il vaut, \inlinecode{1}, le pion plac\'e sera un rond (O majuscule) \inlinecode{"O"}; s'il vaut, \inlinecode{2}, le pion plac\'e sera une croix (X majuscule) \inlinecode{"X"}. L'argument \inlinecode{grille} est une liste de listes. L'argument \inlinecode{position} est une liste ou un tuple de la forme \inlinecode{(i,j)}.
\item Programmez une fonction \inlinecode{chercher_alignement(joueur, liste, n)} qui cherche si un alignement de \inlinecode{n} pions du joueur \inlinecode{joueur} existe dans la liste \inlinecode{liste} (cette liste repr\'esentera plus tard, une ligne, une colonne ou une diagonale). Suivant que l'argument \inlinecode{joueur} vaut \inlinecode{1} ou \inlinecode{2}, on cherchera un alignement de \inlinecode{n} \inlinecode{"O"} ou \inlinecode{n} \inlinecode{"X"}, respectivement.
\item Programmez une fonction \inlinecode{generer_lignes(grille)} qui retourne la liste de toutes les lignes de la grille \inlinecode{grille}. Remarquez que dans ce cas, il suffit de retourner la grille elle m\^eme, puisque celle-ci est donn\'ee comme la liste de ces lignes.
\item Programmez une fonction \inlinecode{generer_colonnes(grille)} qui retourne la liste de toutes les colonnes de la grille \inlinecode{grille}. L'id\'ee est de transposer la matrice \inlinecode{grille}...
\item Programmez une fonction \inlinecode{generer_diagonales1(grille)} qui retourne la liste de toutes les diagonales ``montantes'' de la grille \inlinecode{grille}. Pour cela, remarquez que chaque diagonale de ce type est constitu\'ee des \'el\'ements de la grille dont les indices donnent une m\^eme somme. Par exemple, la diagonale \inlinecode{[(3,1), (2,2), (1,3)]} est form\'ee de tous les \'el\'ements dont les indices ont une somme de $4$.
\item Programmez une fonction \inlinecode{generer_diagonales2(grille)} qui retourne la liste de toutes les diagonales ``descendantes'' de la grille \inlinecode{grille}. Pour cela, remarquez que chaque diagonale de ce type est constitu\'ee des \'el\'ements de la grille dont les indices donnent une m\^eme diff\'erence. Par exemple, la diagonale \inlinecode{[(1,3), (2,4), (3,5)]} d'une grille de taille $5$ est form\'ee de tous les \'el\'ements dont les indices ont une diff\'erence de $-2$.
\item Programmez une fonction \inlinecode{imprimer_grille(grille)} qui imprime la grille \inlinecode{grille} dans votre terminal de mani\`ere conviviale, comme illustr\'e ci-dessus. On rappelle que les cha\^ines \inlinecode{"\t"}, \inlinecode{"\n"} codent un espace de tabulation et un retour de ligne, respectivement.
\item En utilisant toutes les fonctions que vous avez programm\'ees jusque l\`a, d\'efinissez une fonction \inlinecode{jouer(n)} qui permet de jouer \`a Puissance $4$ sur une grille de taille \inlinecode{n}. Utilisez une grande boucle \inlinecode{while} qui demande \`a chaque joueur de placer un pion \`a tour de r\^ole, jusqu'\`a ce qu'un alignement de $4$ pions soit trouv\'e dans une ligne, une colonne ou une diagonale. Tout le programme peut se faire en $250$ lignes de code environ.
\end{enumerate}
\end{exercice}

\begin{filecontents*}{temp.tex}
\newpage
\begin{solution}[TD4_ex5]
\begin{lstlisting}
...
\end{lstlisting}
\end{solution}
\newpage
\end{filecontents*}
\appendsolution





%\begin{filecontents*}{temp.tex}
%
%\begin{solution}[ex:]
%
%\end{solution}
%
%
%\end{filecontents*}
%\appendsolution     % fichiers
\clearpage
\section{Chapitre 7: Graphisme (MatPlotLib) - simulations de Monte-Carlo}

\setcounter{exercice}{0}

\bigskip

%%%%%%%%
%%%%%%%%
%%%%%%%%

\begin{exercice}[Tracer des courbes avec MatPlotLib] \label{ex:fichiers_1} \
\
\begin{enumerate}
\item Implémenter deux fonctions réelles $f$ et $g$ de votre choix.
\item Définir trois listes X, Y et Z, où la liste X contient les valeurs de $x$ dans $[-5,5]$ par pas de $0.05$, la liste Y les valeurs de $y=f(x)$ et Z les valeurs de $z=g(x)$.
\item Vérifier que les trois listes X, Y et Z ont la même longueur.
\item Tracer les courbes $Y=f(X)$ et $Z=g(X)$ sur le même graphique à l'aide de la librairie MatPlotLib.
\end{enumerate}
\end{exercice}

%%%%%%%%
%%%%%%%%
%%%%%%%%
\medskip



\begin{exercice}[Simulation de Monte-Carlo] \label{ex:fichiers_1} \
\

\noindent Le but de cet exercice est d'utiliser la méthode de Monte-Carlo pour obtenir une approximation statistique de l'aire de la surface à l'intérieur d'une courbe fermée. On obtiendra en particulier une valeur approchée de $\pi$.
\medskip
\begin{enumerate}

\item Utiliser la fonction \inlinecode{eval()} de Python pour évaluer une chaîne de caractères en remplaçant
les variables par leur valeur dans l'environnement actuel.
\begin{lstlisting}
x = 0.7
y = -0.8
formule = " x**2 + y**2 <= 1.0 "
test = eval(formule)
print(test)
\end{lstlisting}
\item On veut tirer $N$ point aléatoires dans le carré $[-1,1]\times[-1,1]$. A l'aide de la librairie NumPy, générer un couple $(X,Y)$, où $X$ et $Y$ sont deux listes de longueur $N$
de tirages de variables aléatoires uniformes réelles dans l'intervalle $[-1,1]$.
\item Si je tire N points aléatoires uniformément dans le carré $[-1,1]\times[-1,1]$, quelle est la probabilité que chaque point soit à l'intérieur du disque de centre $(0,0)$ et de rayon 1 ?
\item Implémenter la fonction \inlinecode{montecarlo(listeX, listeY, formule)} qui prend en argument une formule et les coordonnées de $N$ points aléatoires. La fonction tracera en rose le nuage des points qui valident la formule, et en vert le nuage des points qui invalident la formule.
\begin{myboxi}[Indication] Pour tracer un nuage de points avec MatPlotLib, utiliser \inlinecode{plt.scatter()}.
\end{myboxi}
\item Changer le titre du graphique tous les mille points pour afficher l'évolution de la valeur approchée de la surface à l'intérieur de la courbe (utiliser la fonction \inlinecode{plt.title()} de MatPlotLib).
\item Pour avoir une animation plus fluide, ralentir l'affichage en faisant une pause de 0.2 secondes tous les mille points affichés à l'aide de la fonction \inlinecode{plt.pause()}.
\item Afficher la simulation de Monte-Carlo avec 10000 points, et le calcul approché des aires,
pour les trois courbes suivantes :
$$ x^2 + y^2 < 1 $$
$$ \sqrt{|x|} + \sqrt{|y|} < 1 $$
$$ x^2 + (0.3 + 1.5y - |y|^{0.6})^2 < 1 $$
\end{enumerate}
\end{exercice}

\begin{filecontents*}{temp.tex}
\begin{solution}[ex:resultats_3]
\begin{lstlisting}
...
\end{lstlisting}
\end{solution}
\end{filecontents*}
\appendsolution





     % Pandas
\clearpage
\section{Chapitre 8: Graphisme (MatPlotLib) - mouvement brownien}

\setcounter{exercice}{0}

\bigskip

%%%%%%%%
%%%%%%%%
%%%%%%%%

\begin{exercice}[Simulation de mouvements browniens] \label{ex:fichiers_1} \

\begin{enumerate}
\item To be written... (voir le fichier /td/ito.py)
\end{enumerate}

\end{exercice}


%%%%%%%%
%%%%%%%%
%%%%%%%%

\begin{exercice}[Attracteur de Lorentz] \label{ex:fichiers_1} \

\begin{enumerate}
\item To be written... (voir le fichier /td/lorentz.py)
\end{enumerate}

\end{exercice}



\begin{filecontents*}{temp.tex}
\begin{solution}[ex:resultats_3]
\begin{lstlisting}
...
\end{lstlisting}
\end{solution}
\end{filecontents*}
\appendsolution





     % Pandas
\clearpage
\input{chapters/TD_9.tex}     % Pandas
\clearpage
\input{chapters/TD_10.tex}     % Pandas

	
%\setcounter{part}{1}
%\part{Programmation orientee objet}
%\setcounter{exercice}{80} % to continue the numbering from Info-I
%\input{info2/info2.tex}


% uncomment the following lines if solutions are wanted
%\clearpage
%\part{Corriges}
%
\begin{solution}[ex:affectation_1]
Le premier affiche \verb$100$ et le deuxième \verb$25$.
\begin{lstlisting}[style=verbatim]
>>> largeur = 5
>>> hauteur = 10
>>> largeur = hauteur
>>> print largeur, hauteur
10 10
>>>
>>> largeur = 5
>>> hauteur = 10
>>> hauteur = largeur
>>> print largeur, hauteur
5 5
\end{lstlisting}

L'affectation est toujours de droite à gauche.
\end{solution}


\begin{solution}[ex:affectation_2]
\begin{lstlisting}[style=verbatim]
>>> a, b, c = 3, 5, 7
>>> print a - b / c
3
>>> print a - float(b) // c
3.0
>>> print a - float(b) / c
2.28571428571

\end{lstlisting}
L'opérateur \verb$/$ fait la division entière, si les deux nombres sont de type \texttt{int}.
\end{solution}



\begin{solution}[ex:types_1]
\begin{lstlisting}[style=verbatim]
>>> r, pi = 12, 3.14159
>>> s = pi * r ** 2
>>> print s
452.38896
>>> print type(r), type(pi), type(s)
<type 'int'> <type 'float'> <type 'float'>
>>>
>>>
>>> print type("Salut")
<type 'str'>
>>>
>>> print type([1,2,3])
<type 'list'>
\end{lstlisting}

On verra les types \texttt{str} et \texttt{list} dans les prochains chapitres.

\end{solution}



\begin{solution}[ex:premier_programme_1]

\begin{lstlisting}[style=verbatim]
>>> # Tout d'abord, on assigne des valeurs aux
>>> # variables a, b et c
>>> a = 2
>>> b = 4
>>> c = 1
>>> # Comme on utilise deux fois le discriminant (delta),
>>> # on va stocker sa valeur dans une variable :
>>>
>>> delta = b**2 - 4*a*c
>>> delta # Ceci nous affiche ce qui est stocké
8
>>> # Si delta < 0, alors le polynôme n'a pas de solutions réelles.
>>> # on calcule les solutions qu'on stocke dans deux variables x1 et x2
>>> x1 = (-b + delta**0.5)/(2*a)
>>> x2 = (-b - delta**0.5)/(2*a)
>>> # pour composer une phrase, il faut concaténer les variables.
>>> # pour pouvoir faire la concaténation, il faut que les variables
>>> # soit de type chaîne de caractères (string)
>>> print('les solutions sont : ' + str( x1 ) + ', ' + str (x2) )
les solutions sont -0.29289321881345243, -1.7071067811865475
\end{lstlisting}

\end{solution}


\begin{solution}[ex:second_programme_1]
\begin{enumerate}
\item
\begin{lstlisting}[style=verbatim]
>>> nom = "Steve Jobs"
>>> # On stocke la chaîne de caractère dans la variable nom
>>> # On concatène ("colle") l'étoile à "Steve Jobs", puis on
>>> # concatène 100 fois cette nouvelle chaîne.
>>>
>>> (nom + " * ") * 100
Steve Jobs * Steve Jobs * Steve Jobs * Steve Jobs * Steve Jobs * Steve
Jobs * Steve Jobs * . . . Steve Jobs * Steve Jobs *
\end{lstlisting}

\item
\begin{lstlisting}[style=verbatim]
>>> initiales = nom[0] + nom[6] # On concatène la "S" avec "J"
>>> (initiales + " * ") * 100
SJ * SJ * SJ * SJ * SJ * SJ * SJ * SJ * SJ * SJ * SJ * SJ . . . .
* SJ * SJ * SJ * SJ * SJ * SJ *
\end{lstlisting}
\end{enumerate}
\end{solution}


\begin{solution}[ex:interagir_1]
\begin{enumerate}
\item
\begin{lstlisting}
# on stocke le texte entré par l'utilisateur dans trois variables
# pour une chaîne de caractères on utilise raw_input()
nom = raw_input("Quel est votre nom? ")
# pour un nombre on utilise input()
age = input("Quel est votre age? ")
taille = input("Quel est votre taille en mètres? ")
# On calcule la taille en centimètres et on la convertit en
# entier pour enlever la partie décimale
taille_cm = int(taille*100)
# De nouveau, on est obligé de convertir toutes
# les variables en type chaîne de caractères pour composer une phrase
print( "Bonjour " + nom + ", vous êtes né en " + str( 2015 - age ) + " et
faites " + str(taille) + " m ou " + str(taille_cm) + " cm" )
\end{lstlisting}

\item
\begin{lstlisting}
a = input("Quel est votre coefficient a? ")
b = input("Quel est votre coefficient b? ")
c = input("Quel est votre coefficient c? ")
delta = b**2 - 4*a*c
print( "delta vaut " + str(delta) )
x1 = (-b + delta**0.5) / (2*a)
x2 = (-b - delta**0.5) / (2*a)
print( "x1 vaut " + str(x1) )
print( "x2 vaut " + str(x2) )
\end{lstlisting}
\end{enumerate}
\end{solution}

\begin{solution}[ex:conversion_1]
\begin{enumerate}
\item
\begin{lstlisting}
# on demande une chaine sous format HH:MM:SS
heure = raw_input("Entrez l'heure a convertir: ")
hs = heure[0:2] # heures -- deux premiers caractères
mins = heure[3:5] # minutes
secs = heure[6:8] # seconds
# toutes les trois variables sont des chaines
# on ne peut pas les utiliser dans les calculs
# il faut d'abord les convertir en nombres
hint = int(hs)
minint = int(mins)
secint = int(secs)
# on calcule maintenant le nombre de secondes
seconds = hint*3600 + minint*60 + secint
# et on l'affiche à l'écran
print "Il est " + heure + ". \n Il s'est écoulé " + str( seconds ) + \
" secondes depuis minuit."
\end{lstlisting}

\item

\begin{lstlisting}
# on demande une chaine sous format HH:MM:SS.SSS
heure = raw_input("Entrez l'heure a convertir: ")
hs = heure[0:2] # heures -- deux premiers caractères
mins = heure[3:5] # minutes
secs = heure[6:] # secondes du 6ème indice jusqu'à la fin
# toutes les trois variables sont des chaines
# on ne peut pas les utiliser dans les calculs
# il faut d'abord les convertir en nombres
hint = int(hs)
minint = int(mins)
secint = float(secs) # cette fois-ci - un nombre réel
# on calcule maintenant le nombre de seconds
seconds = hint*3600 + minint*60 + secint
# et on l'affiche à l'écran
print "Il est " + heure + ". \n Il s'est écoulé " + str( seconds ) + \
      " secondes depuis minuit."
\end{lstlisting}

\end{enumerate}
\end{solution}
\begin{solution}[ex:diagrammes_1]
\begin{enumerate}
\item Seul (a) n'est pas valide.
\item Les dwits valides sont: (c), (e), (g), (h).
\item Les chaînes de caractères valides sont: (a), (e), (f).
\end{enumerate}
\end{solution}
\begin{solution}[ex:operateurs_1]
\

\noindent\begin{tabularx}{\linewidth}{|c|X|c|}
\hline
\# & Expression          & La valeur de \texttt{x}  \\
\hline
1 & \texttt{y = 5; x = y+1;}     &  6 \\
\hline
2 & \texttt{x = 0; x += 1 ; x += 1; x += x;} &  4 \\
\hline
3 & \texttt{x = "Hello"; y = 'Toto'; x = x+y;}&  "HelloToto" \\
\hline
4 & \texttt{x = 3.0; x = x/3;}    & 1.0 \\
\hline
5 & \texttt{x = 10; x = x/3;}     & 3 \\
\hline
6 & \texttt{x = 10; x = x\%3;}    & 1 \\
\hline
7 & \texttt{x = 7; x /= 2;}     & 3 \\
\hline
8 & \texttt{x = 5**2;}      & 25 \\
\hline
9 & \texttt{x = 3; x **=3;}     & 27 \\
\hline
10 & \texttt{x = 2**1/2;}      & 1 \\
\hline
\end{tabularx}
\end{solution}
\begin{solution}[ex:operateurs_2]
\noindent\begin{tabularx}{\linewidth}{|c|X|X|}
\hline
\# & Expression    & Expression Unaire  \\
\hline
1 & \texttt{x = 1+x}    & \verb$x += 1$ \\
\hline
2 &\texttt{x = x*10}    & \verb$x *= 10$ \\
\hline
3 &\texttt{x = x-1}    & \verb$x -= 1$ \\
\hline
4 & \texttt{x = -2}    & Pas possible \\
\hline
5 & \texttt{x = 10 / x}    & Pas possible \\
\hline
6 & \texttt{x = x / 10}    & \verb$x /= 10$ \\
\hline
7 & \texttt{x = x + "titi"}    & \verb$x += "titi"$ \\
\hline
8 & \texttt{x = "titi" + x}    & Pas possible \\
\hline
9 & \texttt{x = x + 15}    & \verb$x += 15$ \\
\hline
10 & \texttt{x = 15 + x}    & \verb$x += 15$ \\
\hline
\end{tabularx}
\end{solution}
\begin{solution}[ex:operateurs_3]
\noindent\begin{tabularx}{\linewidth}{|c|l|X|}
\hline
\# & Formule & Expression Python \\
\hline
1 & $x = 5x^3+4x^2+2x-9$   & \verb$x = 5*pow(x,3) + 4*pow(x,2) + 2*x -9$   \\
\hline
2 & $\Delta=b^2-4ac$   & \verb$delta = pow(b,2) – 4*a*c$   \\
\hline
3 & $x = \sqrt{2+x}$    & \verb$x = sqrt(2+x)$   \\
\hline
4 & $x = \sqrt{|x| + 5x^3}$   & \verb$x = sqrt( abs(x)+ 5*pow(x,3) )$   \\
\hline
5 & $x=-\sqrt{-7}$    & \verb$x = -sqrt(-7)$ (renvoie un message
d'erreur selon les modules importés)   \\
\hline
6 & $x=\sqrt[3]{x^2}$    &  \verb$x = pow(x,(2/3.0))$  \\
\hline
7 & $\frac{\pi(a+b)}{4K\left(\frac{a-b}{a+b}\right)}$    & \verb$pi*(a+b) / (4*K*( (a-b) / (a+b) ) )$   \\
\hline
\end{tabularx}
\end{solution}
\begin{solution}[ex:conversion_2]
\begin{enumerate}
\item Chaque doigt de la main peut représenter 2 états : levé (=1) ou baissé (=0). Il y a donc
deux possibilités pour chaque doigt. Avec une main normale, il y aura donc 5 doigts, ce
qui représente $2^5$ possibilités, ou un nombre binaire à 5 chiffres. Le plus grand nombre que
l'on puisse obtenir avec une main sera donc : $[11111]_\text{base2}$ . En chiffres décimaux, cela
correspondra à $[31]_\text{base10}$ ($1*2^4+1*2^3+1*2^2+1*2^1+1*2^0$). Notons que le plus grand chiffre
obtenu ne sera pas $2^5$ mais $2^5-1$ puisque le chiffre 0 doit également être défini.
\item En généralisant, on voit qu'un nombre binaire composé de $n$ chiffres aura pour valeur
maximale une suite de $n$ fois le nombre $[11 \dots 11]_\text{base2}$. Cela correspondra à la valeur $2^n-1$
en base décimale.
En utilisant les $2$ mains, on aura donc : $n=10$, $2^{10} -1 = 1023$
En utilisant les $2$ mains et les $2$ pieds : $n=20$, $2^{20} -1 = 1048575$.
\end{enumerate}
\end{solution}
\begin{solution}[ex:conversion_3]
\begin{lstlisting}
t = input('Quelle est la température ? ')
F = t*9./5. + 32
C = (t-32)/(9./5.)
# Attention à ne pas utiliser la variable F mais bien la
# variable t dans la deuxième conversion : on travaille avec
# la valeur entrée par l'utilisateur qu'on veut transformer et
# non pas celle calculée précédemment.
print str(t) + ' C = ' + str(F) + ' F'
print str(t) + ' F = ' + str(C) + ' C'
\end{lstlisting}
\end{solution}
\begin{solution}[ex:conditionnels_1]

\begin{lstlisting}
mot = raw_input('Entrez le 1er mot: ')
phrase = mot
print phrase
mot = raw_input('Entrez le 2eme mot: ')
# Utilisation de l'operateur conditionnel pour tester
# si la première lettre du mot est une voyelle
if (mot[0]=='a' or mot[0]=='e' or mot[0]=='i'
 or mot[0]=='o' or mot[0]=='u' or mot[0]=='y'):
 phrase = mot+' '+phrase
else :
 phrase = phrase+' '+mot
print phrase
mot = raw_input("Entrez le 3eme mot: ")
if (mot[0]=='a' or mot[0]=='e' or mot[0]=='i'
 or mot[0]=='o' or mot[0]=='u' or mot[0]=='y'):
 phrase = mot+' '+phrase
else :
 phrase = phrase+' '+mot
print phrase
\end{lstlisting}

\end{solution}

\begin{solution}[ex:conversion_1b]
\begin{lstlisting}
nsec = int(raw_input("Nombre de secondes? "))

secPerAnn=60*60*24*365
nann, rem = nsec/secPerAnn, nsec%secPerAnn

secPerMoi=60*60*24*30
nmoi, rem = rem/secPerMoi, rem%secPerMoi

secPerJou=60*60*24
njou, rem = rem/secPerJou, rem%secPerJou

secPerHeu=60*60
nheu, rem = rem/secPerHeu, rem%secPerHeu

secPerMin=60
nmin, rem = rem/secPerMin, rem%secPerMin

print str(nann) + " annees, " + str(nmoi) + " mois, " + str(njou) + " jours, "
    + str(nheu) + " heures, " + str(nmin) + " minutes, " + str(rem) + " secondes."
\end{lstlisting}
\end{solution}

\begin{solution}[ex:verite_1]
\begin{lstlisting}[style=verbatim]
A = (not P) or (Q and K)

B = (not P) or ((not Q) and (P or K))

C = ((P or Q) and (not K)) or (P and Q)

D = ((P and Q) or (P and (not Q))) or ((P and Q) and K) or (((not P) and (not Q)) and K)
\end{lstlisting}

Tableau de vérité:


\noindent\begin{tabularx}{\linewidth}{|c||c|c|c||X|X|X|X|}
\hline
\# & P &  Q  &K   & A & B & C & D \\
\hline
1& False& False& False& True & True & False & False \\
\hline
2& False& False& True& True & True & False &True  \\
\hline
3& False& True& False&True  &True  &True  & False \\
\hline
4& False& True& True&True  &True  &  False&  False\\
\hline
5& True &False& False& False &True  & True & True \\
\hline
6& True& False& True& False &True  &  False&True  \\
\hline
7& True& True& False& False &  False&True  & True \\
\hline
8& True& True& True& True &  False& True & True \\
\hline
\end{tabularx}
\end{solution}
\begin{solution}[ex:priorite_1]
\centerline{
\noindent\begin{tabularx}{1.05\linewidth}{|c|X|c|}
\hline
\# & Expression             & Résultat \\
\hline
1 & \verb$2 + 2 * 2 == 8 and True and i ** j / k * 2 ** 2$ & \verb$False$ \\
\hline
2 & \verb$5 ** 2 / 3 == 5 ** 2 / 3.0$ & \verb$False$ \\
\hline
3 & \verb$- 5 ** 2.0 / 3 == 5 ** 2 / 3.0$ & \verb$False$ \\
\hline
4 & \verb$False or False and True or True$ & \verb$True$ \\
\hline
5 & \verb$False != 1 or 10 / 2 < 10$ & \verb$True$ \\
\hline
6 & \verb$k < j or i + 2 == k - 2$ & \verb$False$ \\
\hline
7 & \verb$i != 2 and j / i ** 0 < k$ & \verb$False$ \\
\hline
8 & \verb$j / (i - 2) < k and i != 0$ & Erreur \\
\hline
9 & \verb$k <= j or i+2 == k-1$ & \verb$False$ \\
\hline
10 & \verb$2 + i == j + 2 or i < j != True and False == not True$ & Erreur \\
\hline
11 & \verb$2 + i == j + 2 or i < j != True and not False == True$ & \verb$True$ \\
\hline
12 & \verb$k > j > i$ & \verb$True$ \\
\hline
13 & \verb$j in word$ & Erreur \\
\hline
14 & \verb$"j" in word and "" in word$ & \verb$True$ \\
\hline
15 & \verb$word [5:4] * 3 == "" and word [0] + word [1] in word$ & \verb$True$ \\
\hline
16 & \verb$str(j) in word * k + str(j) and word [2] > word [1]$ & \verb$False$ \\
\hline
\end{tabularx}
}
\end{solution}
\begin{solution}[ex:booleens_1]
\noindent\begin{tabularx}{\linewidth}{|c|X|c|}
\hline
\# & Expression             & La valeur de b \\
\hline
1 & \verb!b = False or True!          & \verb$True$   \\
\hline
2 & \verb!b = not False and True!         & \verb$True$   \\
\hline
3 & \verb!b = True or (10/1)<10!         & \verb$True$   \\
\hline
4 & \verb!a = False; b = not a!          & \verb$True$   \\
\hline
5 & \verb!i=1; j=2; k=3; b = (k<=j) or (i+1==k-1)!     & \verb$True$   \\
\hline
6 & \verb!i=1; j=2; k=3; b = (k<=j) and (i+1==k-1)!     & \verb$False$   \\
\hline
7 & \verb$i=0; j=4; k=9; b = (i!=0) and ((j/i)<k)$     & \verb$False$   \\
\hline
8 & \verb!i=1; j=2; b=i<j!           & \verb$True$   \\
\hline
9 & \verb!i=2; i+=1; b = (i>2.5)!         & \verb$True$   \\
\hline
10 & \verb!i=1; j=2; a=False; c=True!  \newline
   \verb$b = (1+i==j+1) and ((a!=(i<j)) and (c==(not a)))$   & \verb$False$    \\
\hline
11 & \verb!b = True and not not True!        & \verb$True$   \\
\hline
12 & \verb!b = (not True == False);! \newline
   \verb$b = b and (not True != (not False))$      & \verb$True$   \\
\hline
13 & \verb!i=1; j=2; k=3; b = (k<=j) and (i+1==k-1)!     & \verb$False$   \\
\hline
14 & \verb!a = True; c = False; d = (a == (not c));! \newline
   \verb!b = not(d and ((a and c) or ((not a)|(not c))))!   & \verb$False$   \\
\hline
\end{tabularx}
\end{solution}
\begin{solution}[ex:plusgrand_2]
\begin{lstlisting}
n1 = int(raw_input("Entrez un nombre : "))
n2 = int(raw_input("Entrez un autre nombre : "))

if n1 >= n2:
 print "Le premier nombre (" + str(n1) + ")",
else:
 print "Le deuxième nombre (" + str(n2) + ")",

print "est le plus grand parmi les deux."

\end{lstlisting}
\end{solution}
\begin{solution}[ex:pairs_1]
\begin{lstlisting}
import random

n = random.randint(-100, 100)

print "Nombre =", n

if n % 2 == 0:
 print "Le nombre est pair."
else:
 print "Le nombre est impair."

\end{lstlisting}
\end{solution}
\begin{solution}[ex:pairs_1]
\begin{lstlisting}
# -*- coding: UTF-8 -*-

# ex. 26
# Solution d'une equation du second degré - version améliorée de l'exercice 4


a = input("Coefficient a : ")
b = input("Coefficient b : ")
c = input("Coefficient c : ")

print "Equation :", a, "x**2 +", b, "x +", c, " = 0"


delta = b**2 - 4*a*c

print "Le delta est", delta

if delta > 0:
    x1 = (-b - delta**0.5)/(2*a)
    x2 = (-b + delta**0.5)/(2*a)
    print "Les solutions sont :"
    print "x1 =", x1
    print "x2 =", x2
elif delta == 0:
    x = -b / (2.0*a)
    print "Une seule solution :"
    print "x =", x
else:
    print "Pas de solutions."
\end{lstlisting}
\end{solution}
\begin{solution}[ex:horaires_1]
insert
\end{solution}
\begin{solution}[ex:motdepasse_1]

\begin{lstlisting}
    # -*- coding: utf-8 -*-

motdepass = "python"

compteur = 0

while compteur < 3:
    test = raw_input("Entrer le mot de pass : ")
    if test == motdepass:
        print "Mot de pass correct !"
        break
    else:
        print "Le mot de pass n'est pas correct."
    if compteur == 2:
        print "Accès interdit !"
    compteur += 1

    \end{lstlisting}

\end{solution}
\begin{solution}[ex:chaines_4]
\begin{lstlisting}
maChaine, out = "zorglub", ""
for i in range(len(maChaine)):
    out = maChaine[i] + out
\end{lstlisting}
\end{solution}

\begin{solution}[ex:somme_1]
\begin{lstlisting}
a = int(raw_input("Saisir la borne a (entier): "))

b = int(raw_input("Saisir la borne b (entier): "))

s = 0


for i in range(a,b+1):
    if i%3 == 0 or i%5 == 0:
        print "Somme le nombre", i, "..."
        s+=i


print "La somme est :", s
\end{lstlisting}
\end{solution}

\begin{solution}[ex:etoiles_1]
\begin{lstlisting}
nbr_lignes = int(raw_input("Nombre de lignes ? "))
compteur = 0
while compteur <= nbr_lignes:
    compteur += 1
    print compteur * "*"
    # n*string = string + string + ... + string
\end{lstlisting}
\end{solution}


\begin{solution}[ex:chainesboucles_1]
\begin{lstlisting}
maChaine, out = "gaston", ""
for i in range(len(maChaine)):  # pour chaque caractère
    out = out + maChaine[i] + "*" # on l'ajoute avec l'étoile
out = out[:-1]      # on supprime dernière étoile
print out
\end{lstlisting}

\end{solution}
\begin{solution}[ex:boucles_3]
\begin{lstlisting}
nombre = 1
for i in range(12):     # 12 fois
    print nombre      # on l'imprime
    nombre *= 3      # et on le multiplie
\end{lstlisting}
\end{solution}
\begin{solution}[ex:fizzbuzz]
\begin{lstlisting}
for i in range(1,101):
    if (i % 3 == 0):
        if (i % 5 == 0):
            print "FizzBuzz"
        else:
            print "Fizz"
    else:
        if (i % 5 == 0):
            print "Buzz"
        else:
            print i
\end{lstlisting}

Solution alternative:
\begin{lstlisting}
for i in range(1,101):
    s = ""
    if i % 3 == 0: s += "Fizz"
    if i % 5 == 0: s += "Buzz"
    if s == "": s = str(i)
    print s
\end{lstlisting}
\end{solution}

\begin{solution}[ex:boucles_1]

\begin{lstlisting}
nombre, compteur = 7, 0
while compteur < 20:
    print nombre
    compteur += 1
    nombre += 7
\end{lstlisting}

\end{solution}





\begin{solution}[ex:boucles_2]

\begin{lstlisting}
count = 1
rate = 1.65
while count <= 16384:
    print str(count) + " euro(s) = " + str(count*rate) + " dollar(s)"
    count=count*2
\end{lstlisting}
\end{solution}



\begin{solution}[ex:boucles_3b]
\begin{lstlisting}
n,count=1,1
while count<=12 :
    print n
    n,count = n*3,count+1
\end{lstlisting}
\end{solution}



\begin{solution}[ex:boucles_4b]
\begin{lstlisting}
compteur = 0
while compteur < 20:
    compteur += 1
    if compteur % 3 == 0:
        print 7*compteur, "*"
    else:
        print 7*compteur
\end{lstlisting}
\end{solution}


\begin{solution}[ex:boucles_5b]
\begin{lstlisting}
count=1
while count<=50:
    if count%7==0:
        print count*13
    count=count+1
\end{lstlisting}
\end{solution}



\begin{solution}[ex:riz_1]
\begin{lstlisting}
count,n = 0,1
while count <= 64:
    print str(count) + ":\t" + str(n) + "\t" + str(float(n))
    n,count = n*2,count+1
\end{lstlisting}
\end{solution}



\begin{solution}[ex:notes_1]
\begin{lstlisting}
listeNotes = []

while True:
    a = float(raw_input("Saisir une note d'élève: "))
    if a<0:
        break
    listeNotes.append(a)

print "Nombre de notes entrées:", len(listeNotes)
print "Note max:", max(listeNotes)
print "Note min:", min(listeNotes)
print "Moyenne:", sum(listeNotes)/len(listeNotes)
\end{lstlisting}
\end{solution}



\begin{solution}[ex:listes_1]
\begin{lstlisting}
n = int(raw_input("n = ? "))

liste = [0] * n
print("Liste de zéros", liste)

indice = 0
while indice < n:
    liste[indice] = 10 + indice
    indice += 1
print("Liste de nombres", liste)

if n <= 2:
    print("n doit être au moins 3")
else:
    indice = 2
    liste[0] = 0
    liste[1] = 1
    while indice < n:
        liste[indice] = liste[indice-2] + liste[indice-1]
        indice += 1
    print("Suite de Fibonacci", liste)
\end{lstlisting}
\end{solution}





\begin{solution}[ex:listes_2b]
\begin{lstlisting}
serie=[]
while True:
    a = input("Veuillez entrer une valeur: ")
    if a == "":
        break
    serie.append(float(a))
print(serie)
\end{lstlisting}
\end{solution}

\begin{solution}[ex:boucles_6]

\begin{lstlisting}[style=verbatim]
condition vraie
-----------------------
[0, 1]
-----------------------
o l l e h
-----------------------
['Lorem', '-', 'ip-sum', '-', 'dolor']
-----------------------
52c
\end{lstlisting}

\end{solution}

\newpage
\begin{solution}[ex:fichiers_1]



Soit un fichier physique qui s'appelle ``\verb#test.txt#''. Écrivez une
instruction qui permettra:
\begin{itemize}
\item d'ouvrir le fichier en lecture:
\begin{lstlisting}
f = open( "test.txt", "r" )
\end{lstlisting}

\item d'ouvrir le fichier en écriture (si le fichier existe déjà, on doit l'écraser):
\begin{lstlisting}
f = open( "test.txt", "w" )
\end{lstlisting}

\item d'ouvrir le fichier en écriture, pour ajouter des données à la fin:
\begin{lstlisting}
f = open( "test.txt", "a" )
\end{lstlisting}

\end{itemize}
Soit un ficher logique f qui contient plusieurs lignes de texte et qui est déjà ouvert
en lecture avec la commande \inlinecode{f = open("test.txt","r")}. Écrivez des
instructions qui permettront:
\begin{itemize}
\item de récupérer toutes les lignes du ficher f sous la forme d'une chaîne:
\begin{lstlisting}
f.read()
\end{lstlisting}

\item de récupérer toutes les lignes du fichier f sous la forme d'une liste:
\begin{lstlisting}
f.readlines()
\end{lstlisting}

\item d'imprimer la deuxième ligne du fichier f:
\begin{lstlisting}
print f.readlines()[1]
\end{lstlisting}

\end{itemize}

Soit un ficher logique f qui contient plusieurs lignes de texte et qui est ouvert en
lecture et écriture avec la commande \inlinecode{f = open("test.txt", "a+")}. Écrivez des
instructions qui permettront:
\begin{itemize}
\item d'écrire dans le fichier f deux lignes (p.ex. "ligne 1" et "ligne 2"):
\begin{lstlisting}
f.write("ligne 1\nligne2\n")
\end{lstlisting}

\item d'écrire dans le fichier f deux nombres séparés par une virgule:
\begin{lstlisting}
a, b = 5, 10
f.write(str(a) + "," + str(b) + "\n")
\end{lstlisting}


\item d'imprimer la deuxième ligne du fichier f:
\begin{lstlisting}
f.seek(0)
lignes = f.read().split("\n")
print lignes[1]
\end{lstlisting}
\end{itemize}
\end{solution}
\newpage


\newpage
\begin{solution}[ex:modes_1]

Les modes : \texttt{ouvrir svp}, \texttt{t}, \texttt{R+}, \texttt{rw+} ne sont pas valides.

Le modes valides dans la liste de l'exercice sont les suivants :

\begin{description}
\item[\texttt{r}] lecture, accès séquentiel (pas écriture!).
\item[\texttt{w+b}] écriture (et lecture) à accès direct binaire: le fichier sera créé. Un fichier existant avec le même nom est écrasé.
\item[\texttt{w}] uniquement écriture, accès séquentiel. Le fichier sera créé. Un fichier existant avec le même nom est écrasé.
\item[\texttt{a+}] lecture et écriture à accès direct. Le fichier sera créé si pas existante. La tête de lecture/écriture est positionnée à la fin du fichier (``append'').
\item[\texttt{r+}] lecture et écriture à accès direct. Le fichier doit exister déjà. La tête de lecture/écriture est positionnée au début du fichier.
\end{description}

\end{solution}
\newpage


\newpage
\begin{solution}[ex:lecture_1]


Écrivez les instructions qui permettront :
\begin{enumerate}
\item De récupérer toutes les lignes sous forme de liste de chaines de caractères :
\begin{lstlisting}
lignes = fp.readlines()
\end{lstlisting}

\item D'imprimer à l'écran la deuxième ligne du fichier :
\begin{lstlisting}
print fp.readlines()[1]
\end{lstlisting}

\item D'imprimer à l'écran l'avant dernière ligne du fichier :
\begin{lstlisting}
print fp.readlines()[-2]
\end{lstlisting}
\end{enumerate}



\end{solution}
\newpage

\begin{solution}[ex:resultats_1]
\begin{lstlisting}
# -*- coding: utf-8 -*-


def resultatsTest(nom_fichier):
    donnees = {}
    with open(nom_fichier, 'r') as f:
        for ligne in f:
            ligne = ligne.rstrip("\n")

            donnees_etudiant = ligne.split(",")
            prenom = donnees_etudiant[0]
            nom = donnees_etudiant[1]

            donnees[prenom + " " + nom] = donnees_etudiant

    return donnees


def affichageEtudiant(l):
    prenom = l[0]
    nom = l[1]
    faculte = l[2]
    note = l[3]
    bonus = l[4]

    print "L'étudiant", prenom, nom, "de la faculté", faculte, \
          "a obtenu la note :", note, "avec un bonus de :", bonus, "."


def affichageTest(donnees, note_min = 3.5):
    for e in donnees.keys():
        note = float(donnees[e][3])
        if note > note_min:
            affichageEtudiant(donnees[e])



print "------- .1 -------"

donnees_test = resultatsTest("etudiants.csv")

print donnees_test

print "------- .2 -------"

print "Affichage des données du premier étudiant dans le fichier :"

affichageEtudiant(donnees_test["Bart Simpson"])


print "------- .3 -------"

print "Avec note minimale = 2.5"
affichageTest(donnees_test, 2.5)

print "--------------------"

print "Avec note minimale par défault (3.5)"
affichageTest(donnees_test)
\end{lstlisting}
\end{solution}
\begin{solution}[ex:resultats_2]

Code à ajouter après la solution de l'exercice \ref{ex:resultats_1}.

\begin{lstlisting}
donnees_test = resultatsTest("etudiants.csv")

while True:
    nom = raw_input("Entrez le nom et le prénom d'un étudiant séparés par espace : ")
    if nom == "":
        print "Fin."
        break
    if nom not in donnees_test.keys():
        print "Etudiant non trouvé."
        continue
    affichageEtudiant(donnees_test[nom])
\end{lstlisting}
\end{solution}
\begin{solution}[ex:resultats_3]
\begin{lstlisting}

def enregistreDonnees(donnees, fichier, faculte = ""):
    with open(fichier, 'w') as fh:
        compteur = 0
        for e in donnees.keys():
            if faculte != "": # si la faculté a été spécifiée
                if donnees[e][2] != faculte:
                    # si la faculté n'est pas celle spécifiée,
                    # on passe à l'étudiant suivant.
                    continue
            fh.write(",".join(donnees[e])+"\n")
            compteur += 1
    print compteur, "lignes ont été écrites dans le fichier", fichier


enregistreDonnees(donnees_test, "test-all.csv")
enregistreDonnees(donnees_test, "test-esc.csv", "ESC")
\end{lstlisting}
\end{solution}
\begin{solution}[ex:triangle_1]

\begin{lstlisting}
def triangle(longueur):
    for i in range(3):
        #On avance
        forward(longueur)
        #On tourne
        left(120)
\end{lstlisting}
\end{solution}
\begin{solution}[ex:triangle_2]

\begin{lstlisting}
def triangle2(longueur, rempli):
    #Si rempli vaut True, on commence a remplir le triangle
    fill(rempli)
    for i in range(3):
        forward(longueur)
        left(120)
    #Et finalement on arrete de remplir le triangle puisqu'il est fini
    fill(False)
\end{lstlisting}

\end{solution}
\begin{solution}[ex:triangle_3]

\begin{lstlisting}
def troisTriangles(longueur):
    #On dessine un premier triangle
    triangle(longueur)
    #On se place pour dessiner le suivant
    forward(longueur)
    #On dessine le 2e triangle
    triangle(longueur)
    #On revient au debut
    backward(longueur)
    #On se met en place pour le 3e
    left(60)
    forward(longueur)
    right(60)
    #Et on dessine le dernier triangle
    triangle(longueur)
    #Et on remet la tortue au debut (inutile pour cet ex mais servira apres)
    left(60)
    backward(longueur)
    right(60)
\end{lstlisting}

\end{solution}
\begin{solution}[ex:triangle_4]

\begin{lstlisting}
def trianglesAlea():
    for i in range(12):
        #tirage aleatoire de la longueur
        longueur = randint(50,200)
        #tirage aleatoire de la position
        x, y = randint(-300,300), randint(-300,300)
        #tirage aleatoire de la couleur
        couleur = choice(["black", "red", "blue", "green", "yellow", "purple"])
        #tirage aleatoire de l'orientation
        orientation = randint(0,360)
        #On se rend a la position tiree
        #Mais on oublie pas de lever le stylo pour faire ceci
        up()
        goto(x,y)
        down()
        #On prend une orientation de depart
        right(orientation)
        #Et on dessine le triangle
        color(couleur)
        fill(True)
        for i in range(3):
            forward(longueur)
            left(120)
        fill(False)
\end{lstlisting}

\end{solution}
\begin{solution}[ex:triangle_5]

\begin{lstlisting}
speed(11)

def triangle(cote):   # triangle
 down()
 for i in range(3):
  forward(cote)
  left(120)
 up()

def sierpinski(cote):
 if (cote > 16):
  for i in range(3):
   sierpinski(cote/2) # le petit
   forward(cote)  # on avance par cote
   left(120)   # on tourne
 else:
  triangle(cote)


up()
back(256) # pour le centre
sierpinski(512)

done()
\end{lstlisting}

\end{solution}
\begin{solution}[ex:triangle_6]
\begin{lstlisting}
from turtle import*
import math
import random

speed(11)

def sierpinski(a): # dessin le triangle par un seul trait
 left(120);
 for i in range(3):
  forward(a/2) # on avance par a/2
  if(a>30): # on lance la recursion
   sierpinski(a/2)
  forward(a/2) # on avance par a/2
  right(120) # on tourne
 right(120);


begin_fill() # grand triangle noir
fillcolor("black");
backward(256)
for i in range(3):
 forward(512)
 left(120)
forward(256)
end_fill()

begin_fill()
fillcolor("white"); # puis le triangle de serpinski blanc
sierpinski(256)
end_fill()

done()
\end{lstlisting}
\end{solution}

\begin{solution}[ex:random_race_1]
\
\begin{itemize}
\item

\begin{lstlisting}
def randomRace(turtles,steps,step_length=20):
    for count in range(steps):
        for turtle in turtles:
            turtle.forward(step_length)
            #random turn
            turtle.left(choice([-90,90]))
\end{lstlisting}
\item

\begin{lstlisting}
from turtle import *
from random import *

clearscreen() # we clear the screen from any turtle

red_turtle = Turtle()
red_turtle.color("red")
blue_turtle = Turtle()
blue_turtle.color("blue")

randomRace([red_turtle,blue_turtle],50)

def distanceOrigin(turtle):
    return (turtle.xcor()**2 + turtle.ycor()**2)**0.5

if distanceOrigin(red_turtle) >= distanceOrigin(blue_turtle):
    print "The red turtle won the race!"
else:
    print "The blue turtle won the race!"
\end{lstlisting}

\end{itemize}
\end{solution}
\begin{solution}[ex:classes_1]
\begin{lstlisting}
from turtle import *
from random import *

t1 = Turtle()
t2 = Turtle()
t3 = Turtle()
t2.color("red")
t3.color("blue")

def triangle(t, cote):
    t.up()
    t.goto(randint(-200,200),randint(-200,200))
    t.down()
    for i in range(3):
        t.forward(cote)
        t.left(120)
    t.up()
\end{lstlisting}

\begin{lstlisting}
def infoTortue(t):
    print "Position de la tortue :", t.position()
    print "Couleur :", t.color()
    print "Angle :", t.heading()

t1 = Turtle()
t1.color("red")

t1.forward(100)
t1.left(50)


infoTortue(t1)
\end{lstlisting}
\end{solution}
\begin{solution}[ex:polygones_1]
\begin{lstlisting}
from turtle import *
from random import randint

def polygone(longueur,nombre) :
        angle = 360.0 / nombre;
        for i in range(0,nombre) :
            forward(longueur)
            right(angle)

def exercice_1() :
    # Appel de la fonction avec les parametres voulus
    # i.e. longueur d'un cote et nombre de cotes
    polygone(70,9)

def exercice_2() :
    for i in range(0,12) :
        # orientation et positionnement choisis au hasard
        orientation = randint(0,360)
        x = randint(-200,200)
        y = randint(-200,200)
        # deplacement de la tortue avec ces valeurs
        up()
        goto(x,y)
        right(orientation)
        down()

        # nombre de cot?s et taille d'un cote choisis au hasard
        longueur = randint(0,200)
        # un polygone a au moins 3 cotes (triangle)
        nombre = randint(3,10)
        # Dessinons maintenant notre polygone !
        polygone(longueur,nombre)
\end{lstlisting}
\end{solution}
\begin{solution}[ex:classes_4]
Exemples de attributs: \verb$couleur$ (string), \verb$matériau$ (string), \verb$poids$ (float), \verb$pour_exterieur$ (bool), \verb$avec_roulette$ (bool)
\end{solution}
\begin{solution}[ex:classes_2]
\includegraphics[width=\textwidth]{graphics/classes_d2.pdf}
\end{solution}
\begin{solution}[ex:classes_3]
\includegraphics[width=\textwidth]{graphics/classes_d3.pdf}

\end{solution}

\begin{solution}[ex:create_class_1]
\begin{lstlisting}
class Domino():
    def __init__(self, Avalue = 0, Bvalue = 0):
        self.Avalue = Avalue
        self.Bvalue = Bvalue

    def fixed_points(self):
        print "A value: {} B value: {}".format (self.Avalue,self.Bvalue)

    def sum_points(self):
        plus = self.Avalue + self.Bvalue
        return plus

d1 = Domino(5,6)
d2 = Domino(2,9)

d1.fixed_points()
print d2.sum_points()
\end{lstlisting}

\end{solution}

\begin{solution}[ex:create_class_2]
\begin{lstlisting}
# -*- coding: utf-8 -*-

from random import *

class JeuDeCartes():

    def __init__(self): #constructeur
        self.cartes = []

        for color in range(4):
            for number in range(2,15):
                self.cartes.append( (number, color) )

    def nom_carte(self, a, b):
        self.number = ['','','Deux','Trois','Quatre','Cinq','Six','Sept','Huit','Neuf','Dix','Valet','Dame','Roi','As']
        self.color = ['Coeur','Carreau','Trefle','Pique']
        return self.number[a] + " de " + self.color[b]

    def battre(self):
        cartes_copy = self.cartes[:]
        self.cartes = []
        while len(cartes_copy) > 0:
            self.cartes.append(cartes_copy.pop(randint(0,len(cartes_copy)-1)))

    def tirer(self):
        if len(self.cartes) == 0:
            return None
        else:
            return self.cartes.pop(0)

    def affiche_cartes(self):
        for c in self.cartes:
            print self.nom_carte(c[0],c[1])


#starts the game
A = JeuDeCartes()
B = JeuDeCartes()
A.battre()
B.battre()

a = 0
b = 0
for n in range(10):

    (a1, a2) = A.tirer()
    (b1, b2) = B.tirer()
    print "La carte tirée par A est", A.nom_carte(a1,a2)
    print "La carte tirée par B est", B.nom_carte(b1,b2)
    if a1 > b1:
        a = a + 1
        print("Un point pour A")
    elif a1 < b1:
        b = b + 1
        print("Un point pour B")

if a < b:
    print 'Le joueur B gagne'
else:
    print 'Le joueur A gagne'

\end{lstlisting}

\end{solution}
\begin{solution}[ex:create_class_3]
\begin{lstlisting}


from random import *

# the class Person
class Person:
    def __init__(self, name, age = 0):
        self.name = name
        self.age = age

    def get_name(self):
        print "The name is " + str(self.name) #gives the name

    def get_age(self):
        print "the age is " + str(self.age) #gives age


# a class with similar characteristic to class Person
class Student(Person):
    def __init__(self, name, age, mark = 0):
        # appel à la méthode d'initialisation de la classe mère pour créer les attributs name et age
        Person.__init__(self, name, age)
        # la note
        self.mark = mark

    #method
    def get_mark(self):
        print "The student's mark is " + str(self.mark)



names = ["Juan","Luis","Pepe","Moncho"]
age = randint(25,50)

student1 = Student(choice(names),age,3)
student1.get_name()
student1.get_age()
student1.get_mark()
print isinstance(student1, Student)

person1 = Person(choice(names),53)
person1.get_name()
person1.get_age()
print isinstance(person1, Student)

# the next instruction will not work, because person1 is not a Student,
# so it doesn't have the method "get_mark()"
# person1.get_mark()
\end{lstlisting}

\end{solution}

\begin{solution}[ex:create_class_4]
\begin{lstlisting}
from math import sqrt
from random import randint

class Rocket():
    # Rocket simulates a rocket ship for a game,
    #  or a physics simulation.

    def __init__(self, x=0, y=0):
        # Each rocket has an (x,y) position.
        self.x = x
        self.y = y

    def move_rocket(self, x_increment=0, y_increment=1):
        # Move the rocket according to the paremeters given.
        #  Default behavior is to move the rocket up one unit.
        self.x += x_increment
        self.y += y_increment

    def get_distance(self, other_rocket):
        # Calculates the distance from this rocket to another rocket,
        #  and returns that value.
        distance = sqrt((self.x-other_rocket.x)**2+(self.y-other_rocket.y)**2)
        return distance

class Shuttle(Rocket):
    # Shuttle simulates a space shuttle, which is really
    #  just a reusable rocket.

    def __init__(self, x=0, y=0, flights_completed=0):
        Rocket.__init__(self, x, y)
        self.flights_completed = flights_completed


# Create several shuttles and rockets, with random positions.
#  Shuttles have a random number of flights completed.
shuttles = []
for x in range(0,3):
    x = randint(0,100)
    y = randint(1,100)
    flights_completed = randint(0,10)
    shuttles.append(Shuttle(x, y, flights_completed))

rockets = []
for x in range(0,3):
    x = randint(0,100)
    y = randint(1,100)
    rockets.append(Rocket(x, y))

# Show the number of flights completed for each shuttle.
for index, shuttle in enumerate(shuttles):
    print("Shuttle %d has completed %d flights." % (index, shuttle.flights_completed))

print("\n")
# Show the distance from the first shuttle to all other shuttles.
first_shuttle = shuttles[0]
for index, shuttle in enumerate(shuttles):
    distance = first_shuttle.get_distance(shuttle)
    print("The first shuttle is %f units away from shuttle %d." % (distance, index))

print("\n")
# Show the distance from the first shuttle to all other rockets.
for index, rocket in enumerate(rockets):
    distance = first_shuttle.get_distance(rocket)
    print("The first shuttle is %f units away from rocket %d." % (distance, index))


\end{lstlisting}

\end{solution}



\begin{solution}[ex:fractions_1]
\begin{lstlisting}
# -*- coding: utf-8 -*-


class Fraction():
    # initialisation et attributs
    def __init__(self, num = 0, den = 1):
        self.num = num
        if den == 0:
            print "Dénominateur nul"
            return None
        self.den = den

        # simplification
        self.simplifier()

    def plus(self, f):
        resultat = Fraction(self.num * f.den + self.den * f.num, self.den * f.den)
        return resultat

    def moins(self, f):
        resultat = Fraction(self.num * f.den - self.den * f.num, self.den * f.den)
        return resultat

    def fois(self, f):
        resultat = Fraction(self.num * f.num, self.den * f.den)
        return resultat

    def div(self, f):
        resultat = Fraction(self.num * f.den, self.den * f.num)
        return resultat

    def simplifier(self):
        '''calcul du plus grand commun diviseur - PREMIERE VERSION'''

        pgcd = 1
        # calculer le pgcd
        i = min(abs(self.num), abs(self.den))
        while i >= 2:
            if self.num % i == 0 and self.den % i == 0:
                 pgcd = i
                 break
            i=i-1
        self.num /= pgcd
        self.den /= pgcd

        if self.den < 0:
            self.num *= -1
            self.den *= -1

    def display(self):
        '''affichage d'une fraction'''
        print self.num, "/" , self.den

    def __str__(self):
        '''répresentation comme texte'''
        return str(self.num) + " / " + str(self.den)

    def __eq__(self, other):
        '''opérateur de comparaison'''
        if not isinstance(other, Fraction):
            return False
        self.simplifier()
        other.simplifier()
        return self.num == other.num and self.den == other.den


a = Fraction(1,4)
a.display()

b = Fraction(2,3)
b.display()


p = a.plus(b)
p.display()


m = a.moins(b)
# avec le 'print', on appelle la méthode __str__ de la fraction
print m

f = a.fois(b)
print f

d = a.div(b)
print f

\end{lstlisting}
\end{solution}





\begin{solution}[ex:diag_seq_1]
\
\begin{itemize}

\item
\begin{center}
\includegraphics[width=0.5\textwidth]{graphics/diag_seq3.pdf}
\end{center}

\item
\begin{center}
\includegraphics[width=0.5\textwidth]{graphics/diag_seq4.pdf}
\end{center}

\item
\begin{center}
\includegraphics[width=0.5\textwidth]{graphics/diag_seq5.pdf}
\end{center}

\end{itemize}

\end{solution}




\begin{solution}[ex:classe_2]

\textbf{Solution des exercices \ref{ex:classe_2}, \ref{ex:heritage_1}, \ref{ex:classe_3}.}

\verb$academics.py$ :

\begin{lstlisting}
#!/usr/bin/python
# -*- coding: utf-8 -*-

class Personne:
    # EXERCICE 86
    def __init__(self,name):
        self.name = name

    def toString(self):
        print "Personne : " + str(self.name)

    def __str__(self):
        return "Personne : " + str(self.name)

# EXERCICE 87
class Etudiant(Personne):
    def __init__(self,name,noEtudiant):
        Personne.__init__(self,name)
        self.noEtudiant = noEtudiant
    def __str__(self):
        return  Personne.__str__(self) +" , no etudiant : " + str(self.noEtudiant)

class Professeur(Personne):
    def __init__(self,name,faculte):
        Personne.__init__(self,name)
        self.faculte = faculte

    def changerFaculte(self,faculte):
        self.faculte = faculte

    def __str__(self):
         return "Prof : " + str(self.name) + " , faculte : " + str(self.faculte)
# EXERCICE 88
count = 1
class Cours:
    def __init__(self,title):
        self.title = title
        self.listeEtudiant = dict()
        global count
        self.noSalle = count
        count += 1
        self.horaire = ""
        self.prof = ""
    def __str__(self):
        return "Cours : " + str(self.title) + " , no Salle : " + str(self.noSalle) + " , horaire : " + str(self.horaire) + " , prof : " + str(self.prof)

    def assignerHoraire(self,horaire):
        self.horaire = horaire

    def inscrireEtudiant(self,etudiant):
        self.listeEtudiant[etudiant.noEtudiant] = etudiant

    def assignerProf(self,prof):
        self.prof = prof
\end{lstlisting}
\end{solution}



\begin{solution}[ex:modules_1]
\verb$test.py$ :
\begin{lstlisting}
#!/usr/bin/python
# -*- coding: utf-8 -*-

#import academics as ac
from academics import *

def main():

    # exercice 86
    pers = Personne("Pepe")
    print pers
    pers.toString()
    #exercice 86
    a = Etudiant("Renaud",1)
    b = Etudiant("Henry Hankford",2)
    c = Professeur("Hugo Bosse","Lettres Modernes - ")
    d = Professeur("Paul Johnson","Criminologie")
    listePersonne = (a,b,c,d)
    for person in listePersonne:
        print person

    # exercice 87
    c1 = Cours("Economie 402")
    c1.inscrireEtudiant(a)
    c1.inscrireEtudiant(b)
    c1.assignerProf(c)
    c1.assignerHoraire("Merc. 16h 18h")
    print c1
    for v in c1.listeEtudiant.values() :
        print v

if __name__ == '__main__':
    main()
\end{lstlisting}
\end{solution}



\begin{solution}[ex:dessin_modele_1]

\includegraphics[width=\textwidth]{graphics/tp5_3.pdf}

\end{solution}



\begin{solution}[ex:diagrammes_etats_1]

\begin{enumerate}
\item Le réveil a deux états distincts Désarmé (alarme = off) ou Armé (alarme=on). Une action de l'utilisateur permet de passer d'un état à un autre. On suppose que le réveil est bien désarmé au départ.
Si l'on considère les phrases 2 et 3, le fait de sonner constitue un nouvel état pour le réveil. Il s'agit bien d'une période de temps durant laquelle le réveil effectue une certaine activité (sonner) qui dure jusqu'à ce qu'un événement vienne l'interrompre.
Le passage de l’état Armé à l'état Sonnerie est déclenché par une transition due à un changement interne représenté par le mot clé `When'. En revanche, d'après l'énoncé, le retour de l'état Sonnerie à l'état armé ne s'effectue que sur un évènement utilisateur.

\includegraphics[width=0.5\textwidth]{graphics/et_1.png}


\item
Dans ce cas il suffit d'ajouter une activité Sonner à l'état Sonnerie et une transition automatique en sortie de cet état.

\includegraphics[width=0.5\textwidth]{graphics/et_2.png}


\end{enumerate}
\end{solution}




\begin{solution}[ex:academics_implement_1]
\textbf{Solution des exercices \ref{ex:academics_implement_1}, \ref{ex:classes_examen_1}.}

\verb$academics_new.py$ :

\begin{lstlisting}
#-*- encoding : utf-8 -*-


from academics import *
#veuillez mettre le fichier academics.py
#dans le meme dossier que ce script


from random import *

class Examinateur(Professeur):
    def __init__(self,nom,faculte):
        Professeur.__init__(self,nom,faculte)

    def evaluerEtudiant(self,etudiant):
        st = etudiant.noEtudiant
        last = st % 10
        return (last / 2) + 1


class ExaminateurMechant(Examinateur):
    def __init__(self,nom,faculte):
        Examinateur.__init__(self,nom,faculte)

    def evaluerEtudiant(self,etudiant):
        note = Examinateur.evaluerEtudiant(self,etudiant)
        if (note > 2 ):
            note -= 1
        return note
    def __str__(self):
        return "Prof. \" Mechant \" " + self.name


class CoursAvecExamen(Cours):
    def __init__(self,title):
        Cours.__init__(self,title)
        self.notes = dict()

    def menerExamen(self):
        for e,v in self.listeEtudiant.items():
            self.notes[e] = self.prof.evaluerEtudiant(self.listeEtudiant[e])

    def procesVerbal(self):
        #print "*********** Proces Verbal ***********"
        print "Cours : " + self.title + " \n " + str(self.prof) + " .\n"
        print "Liste des etudiants : \n "
        for no,etu in self.listeEtudiant.items():
            print "etudiant no : " + str(etu.noEtudiant)
            if(self.notes[etu.noEtudiant] != None):
                print  "    grade : " + str(self.notes[etu.noEtudiant])


    def meanGrade(self):
        grade = 0
        for e in self.listeEtudiant:
            grade += self.notes[e]
        if(len(self.notes)>0):
            return grade/len(self.notes)
        else:
            print "aucuns etudiants inscrit"

    def resultatsAgreges(self):
        #print " ********  Resultats agreges **********"
        print " Cours : " + self.title
        print "Moyenne du cours : " + str(self.meanGrade())
        nb=0
        for no in self.listeEtudiant:
            if(self.notes[no] > 3.5):
                nb+=1
        print "Nombre d'etudiant avec une note > 3.5 : " + str(nb)


def main():
    e1 = Etudiant("Barrington Levy",7)
    p1 = ExaminateurMechant("Dr pepper","Green Scientific House")
    #print "note de e1 = " + str(p1.evaluerEtudiant(e1))
    print p1
    c1 = CoursAvecExamen("Histoire")
    c1.assignerProf(p1)
    c1.inscrireEtudiant(e1)
    c1.menerExamen()
    print "etudiant got : " + str(c1.notes[e1.noEtudiant])

    # EXERCICE 3
    # 3.a
    c1 = CoursAvecExamen("Geographie")
    c2 = CoursAvecExamen("Etude de l'Etre et le Neant - Sartre")
    c3 = CoursAvecExamen("Study of SMP algorithm for extended multiprocessor")
    cours = (c1,c2,c3)
    # 3.b
    e1 = Examinateur("Mr. Patate", "Patatologie")
    e2 = ExaminateurMechant("Mr. Robert", "Roblogie")
    c1.assignerProf(e1)
    c2.assignerProf(e2)
    c3.assignerProf(e2)
    # 3.c
    allowedName = ("Donald","Cendrillon","Homer","Lisa","Bender", "Kenny")
    alphabet = "abcdefghijklmnopqrstuvwxyz"
    etId = 2012001
    listEtudiant = {}
    for i in range(0,30):
        name = choice(allowedName)
        lettres = sample(alphabet, 5) # cinq lettres aleatoires
        etName = name + "".join(lettres).capitalize()
        listEtudiant[i] = Etudiant(etName,etId)
        etId += 1

    # 3.d
    #listing des etudiants
    for i in range(1,30,3):
        #listing des cours
        for noCours in range(len(cours)):
            # tirage au sort ( 30 % de chance de tomber sur 1 => inscription)
            no = random()
            if(no <0.3):
                cours[noCours].inscrireEtudiant(listEtudiant[i])

    # 3.e
    for c in cours:
        c.menerExamen()
        c.procesVerbal()
        c.resultatsAgreges()


if __name__ == '__main__':
    main()
\end{lstlisting}
\end{solution}



\begin{solution}[ex:revision_classes_1]

\begin{lstlisting}
# EX2
#2.1)
class Livre:    # une classe - Livre
 def __init__(self, t, p=20):
  self.titre = t  # avec un titre
  self.prix  = p  # et un prix

 def __str__(self):  # l'affichage
  return 'Livre: '+self.titre + ' - ' + ("%.2f" % self.prix) + ' chf'

 def reduirePrix(self, pourcentage):
  self.prix *=((100-pourcentage)/100.0)

#end class

#2.2)
l1 = Livre("Python", 20)
print l1
l1.reduirePrix(10)
print l1

#2.3)
import random as rnd;

class Bibliotheque:
 def __init__(self):
  self.livres = []
 def reduirePrix(self, pourcentage):
  for i in self.livres :
   i.reduirePrix(pourcentage)
 def __str__(self):
  biblio = "Liste des livres : \n"
  for i in self.livres:
   biblio += (str(i)+"\n")
  return biblio
 def ajouterLivres(self,n):
  for p in range(10,10*(n+1),10):
   titre = "".join(rnd.sample("abcdefghi", 3)) # un titre alleatoire
   livre = Livre( titre, p );   # creation d'un objet
   self.livres.append(livre);

#2.4)

b1 = Bibliotheque()
b1.ajouterLivres(15)
print b1
#2.5)
b1.reduirePrix(20)
b1.reduirePrix(25)
print "Bibliotheque apres les reductions: \n";
print b1

\end{lstlisting}

\end{solution}




\begin{solution}[ex:interface_simple_tkinter_1]
\begin{lstlisting}
from Tkinter import *
from math import *


def calculSphere(rayon) :
    """ Renvoie la surface """
    surface = 4 * pi * pow(rayon,2)
    return surface

def updateEntry() :
    r = rayon.get()
    if len(r) != 0 :
        rayonLu = float(r)
        resultat = calculSphere(rayonLu)
        surface.set(resultat)

fenetre = Tk()

# les variables dont on a besoin

rayon = StringVar()
surface = StringVar()
labelRayon = Label(fenetre,text = "Rayon de la sphere : ")
inputRayon = Entry(fenetre, textvariable = rayon)
labelSurface = Label(fenetre,text = "Surface : ")
labelCalculSurface = Label(fenetre, textvariable = surface)

validButton = Button(fenetre,text = "Valider", command = updateEntry)

quitButton = Button(fenetre,text = "Quitter", command = fenetre.destroy)

labelRayon.pack()
inputRayon.pack()
labelSurface.pack()
labelCalculSurface.pack()
validButton.pack()
quitButton.pack()
fenetre.mainloop()
\end{lstlisting}
\end{solution}




\begin{solution}[ex:fractions_2]
\begin{lstlisting}
#Algorithme d'Euclide.

def gcd(numerateur, denominateur):
    while denominateur != 0:
        reste = numerateur % denominateur
        numerateur = denominateur
        denominateur = reste
    #A la fin de l'algorithme, le numerateur est le gcd
    #Car numerateur vaut ici le denominateur du tour de boucle precedent
    return numerateur

#Notez qu'il y a plein d'autres moyens d'implementer l'algorithme d'Euclide
#Par exemple, par recursion :)

numerateur = input("Entrez le numerateur: ")
denominateur = input("Entrez le denominateur: ")

#Pour simplifier la fraction, on utilise simplement le gcd

gcd = gcd(numerateur, denominateur)
print numerateur/gcd, " / ", denominateur/gcd
\end{lstlisting}
\end{solution}



\begin{solution}[ex:fractions_tkinter_1]
\begin{lstlisting}[basicstyle=\scriptsize\ttfamily]
from Tkinter import *

def gcd(numerateur, denominateur):
    #Algorithme d'Euclide pour calculer le plus grand diviseur commun (gcd)
    while denominateur != 0:
        reste = numerateur % denominateur
        numerateur = denominateur
        denominateur = reste

    return numerateur

def simplifier():
    n = int(numerateur.get())
    d = int(denominateur.get())
    g = gcd(n, d)
    #Et on entre les valeurs de la fraction simplifiee dans le champ
    numerateur.set(n/g)
    denominateur.set(d/g)

fenetre = Tk()

etiquette = Label(fenetre, text="Fraction")
etiquette.pack()

#Premier champ, numerateur
numerateur = StringVar()
champ1 = Entry(fenetre, textvariable = numerateur, justify=CENTER)
champ1.pack()

#Barre de separation
separateur = Label(fenetre, text="------------")
separateur.pack()

#Deuxieme champ, denominateur
denominateur = StringVar()
champ2 = Entry(fenetre, textvariable = denominateur, justify=CENTER)
champ2.pack()

#Bouton simplifier
valider = Button(fenetre, text="Simplifier", command=simplifier)
valider.pack()

#Bouton quitter
quitter = Button(fenetre, text="Quitter", command=fenetre.destroy)
quitter.pack()

fenetre.mainloop()
\end{lstlisting}
\end{solution}



\begin{solution}[ex:addition_fractions_1]
\begin{lstlisting}[basicstyle=\scriptsize\ttfamily]
from Tkinter import *

def gcd(numerateur, denominateur):
    #Algorithme d'Euclide pour calculer le plus grand diviseur commun (gcd)
    while denominateur != 0:
        reste = numerateur % denominateur
        numerateur = denominateur
        denominateur = reste

    return numerateur

def addition():
    n1 = int(numerateur1.get())
    d1 = int(denominateur1.get())
    n2 = int(numerateur2.get())
    d2 = int(denominateur2.get())
    #Ici, on peut soit chercher le lcm (least common denominator)
    #Mais on peut aussi prendre n'importe quel denominateur commun
    #p.ex denominateur1 * denominateur2 et ensuite simplifier le resultat.
    #Prenons la deuxieme solution
    communDenom = d1 * d2
    addition = n1*(communDenom/d1) + n2*(communDenom/d2)
    g = gcd(addition, communDenom)
    #Et on entre les valeurs de la fraction simplifiee dans le champ resultat
    #via les textvariables.
    numerateur3.set(str(addition/g))
    denominateur3.set(str(communDenom/g))
    #On aurait aussi pu utiliser champ5.insert() p.ex
    #C'est moins propre car si on veut relancer les calculs il faut aussi
    #rajouter des champ5.delete() et compagnie (et retirer le DISABLED)


fenetre = Tk()

#PREMIERE COLONNE

etiquette = Label(fenetre, text="Fraction 1")
etiquette.grid(row=0, column=0, sticky=N)

#Premier champ, numerateur
numerateur1 = Entry(fenetre, justify=CENTER)
numerateur1.grid(row=1, column=0, sticky=N)

#Barre de separation
separateur = Label(fenetre, text="------------")
separateur.grid(row=2, column=0, sticky=N)

#Deuxieme champ, denominateur
denominateur1 = Entry(fenetre, justify=CENTER)
denominateur1.grid(row=3, column=0, sticky=N)

#DEUXIEME COLONNE (+)

etiquette = Label(fenetre, text="+")
etiquette.grid(row=2, column=1, sticky=N)

#TROISIEME COLONNE

etiquette = Label(fenetre, text="Fraction 2")
etiquette.grid(row=0, column=2, sticky=N)

#Premier champ, numerateur
numerateur2 = Entry(fenetre, justify=CENTER)
numerateur2.grid(row=1, column=2, sticky=N)

#Barre de separation
separateur = Label(fenetre, text="------------")
separateur.grid(row=2, column=2, sticky=N)

#Deuxieme champ, denominateur
denominateur2 = Entry(fenetre, justify=CENTER)
denominateur2.grid(row=3, column=2, sticky=N)

#Bouton quitter
quitter = Button(fenetre, text="Quitter", command=fenetre.destroy)
quitter.grid(row=4, column=2, sticky=N)

#QUATRIEME COLONNE (=)

etiquette = Label(fenetre, text="=")
etiquette.grid(row=2, column=3, sticky=N)

#CINQUIEME COLONNE

etiquette = Label(fenetre, text="Resultat")
etiquette.grid(row=0, column=4, sticky=N)

#Premier champ, numerateur
numerateur3 = StringVar()
champ5 = Entry(fenetre, textvariable = numerateur3, justify=CENTER, state=DISABLED)
champ5.grid(row=1, column=4, sticky=N)

#Barre de separation
separateur = Label(fenetre, text="------------")
separateur.grid(row=2, column=4, sticky=N)

#Deuxieme champ, denominateur
denominateur3 = StringVar()
champ6 = Entry(fenetre, textvariable = denominateur3, justify=CENTER, state=DISABLED)
champ6.grid(row=3, column=4, sticky=N)

#Bouton calcul
valider = Button(fenetre, text="Calculer", command=addition)
valider.grid(row=4, column=4, sticky=N)

fenetre.mainloop()
\end{lstlisting}
\end{solution}



\begin{solution}[ex:tkinter_widgets_1]
\begin{lstlisting}
from Tkinter import *

# the function that is called when an item in the listbox is selected
# the event is passed as an argument by Tkinter
def listbox_clicked(evt):
    w = evt.widget
    index = int(w.curselection()[0])
    value = w.get(index)
    text.set(text.get() + "\n" + "You selected item " + str(index) + ": " + value)

fenetre = Tk()

# create the Listbox
lb = Listbox(fenetre, height=4)
lb.pack(side=LEFT)
lb.insert(END,"first entry")
lb.insert(END,"second entry")
lb.insert(END,"third entry")
lb.insert(END,"fourth entry")

# associate the select event to the function
lb.bind('<<ListboxSelect>>', listbox_clicked)

# create the Message widget and the relative StringVar
text = StringVar()
message = Message(fenetre, width=500, textvariable=text)
message.pack(side=RIGHT)

fenetre.mainloop()
\end{lstlisting}
\end{solution}




\begin{solution}[ex:fractions_tkinter_2]

Voir Exercice \ref{ex:fractions_tkinter_3}.

%\begin{lstlisting}[basicstyle=\scriptsize\ttfamily]
%from Tkinter import *
%
%def gcd(numerateur, denominateur):
%    #Algorithme d'Euclide pour calculer le plus grand diviseur commun (gcd)
%    while denominateur != 0:
%        reste = numerateur % denominateur
%        numerateur = denominateur
%        denominateur = reste
%
%    return numerateur
%
%def calcul(e=None):
%    # le parametre e=None est necessaire pour l'ex 4.3
%
%    #Ceci est juste pour l'ex 4.3, pour eviter d'avoir des messages d'erreur
%    #car il tentera de faire les calculs avant que tous les champs soient remplis
%    if numerateur1.get() == "" or numerateur2.get() == "" or denominateur1.get()=="" or denominateur2.get() == "":
%        return
%
%    n1 = int(numerateur1.get())
%    d1 = int(denominateur1.get())
%    n2 = int(numerateur2.get())
%    d2 = int(denominateur2.get())
%
%    #Addition
%    if choix.get() == 1:
%        #Ici on prend simplement d1 * d2 comme denonminateur commun
%        denominateur = d1 * d2
%        numerateur = n1*d2 + n2*d1
%
%    #Soustraction
%    if choix.get() == 2:
%        denominateur = d1 * d2
%        numerateur = n1*d2 - n2*d1
%
%    #Multiplication
%    if choix.get() == 3:
%        numerateur = n1*n2
%        denominateur = d1*d2
%
%    #Division
%    if choix.get() == 4:
%        numerateur = n1*d2
%        denominateur = n2*d1
%
%    #Pour la simplification, on doit calculer le plus grand diviseur commun
%    g = gcd(numerateur, denominateur)
%
%    #Et on entre les valeurs de la fraction simplifiee dans le champ resultat
%    #via les textvariables.
%    numerateur3.set( str(numerateur/g) )
%    denominateur3.set( str(denominateur/g) )
%    #On aurait aussi pu utiliser champ5.insert() p.ex
%    #C'est moins propre car si on veut relancer les calculs il faut aussi
%    #rajouter des champ5.delete() et compagnie (et retirer le DISABLED).
%
%fenetre = Tk()
%
%#PREMIERE COLONNE
%
%etiquette = Label(fenetre, text="Fraction 1")
%etiquette.grid(row=0, column=0, sticky=N)
%
%#Premier champ, numerateur
%numerateur1 = Entry(fenetre, justify=CENTER)
%numerateur1.grid(row=1, column=0, sticky=N)
%
%#Barre de separation
%separateur = Label(fenetre, text="------------")
%separateur.grid(row=2, column=0, sticky=N)
%
%#Deuxieme champ, denominateur
%denominateur1 = Entry(fenetre, justify=CENTER)
%denominateur1.grid(row=3, column=0, sticky=N)
%
%#DEUXIEME COLONNE (radiobutton)
%
%etiquette = Label(fenetre, text="Operation")
%etiquette.grid(row=0, column=1, sticky=N)
%
%choix = IntVar()
%choix.set(1)
%
%#Ex 4.2: grace a command=calclul dans les radiobutton, le calcul est lance quand on clique sur le bouton
%r1 = Radiobutton(fenetre, text="Addition", variable=choix, value=1, command=calcul)
%r1.grid(row=1, column=1, sticky=W)
%r2 = Radiobutton(fenetre, text="Soustraction", variable=choix, value=2, command=calcul)
%r2.grid(row=2, column=1, sticky=W)
%r3 = Radiobutton(fenetre, text="Multiplication", variable=choix, value=3, command=calcul)
%r3.grid(row=3, column=1, sticky=W)
%r4 = Radiobutton(fenetre, text="Division", variable=choix, value=4, command=calcul)
%r4.grid(row=4, column=1, sticky=W)
%
%#TROISIEME COLONNE
%
%etiquette = Label(fenetre, text="Fraction 2")
%etiquette.grid(row=0, column=2, sticky=N)
%
%#Premier champ, numerateur
%numerateur2 = Entry(fenetre, justify=CENTER)
%numerateur2.grid(row=1, column=2, sticky=N)
%
%#Barre de separation
%separateur = Label(fenetre, text="------------")
%separateur.grid(row=2, column=2, sticky=N)
%
%#Deuxieme champ, denominateur
%denominateur2 = Entry(fenetre, justify=CENTER)
%denominateur2.grid(row=3, column=2, sticky=N)
%
%#Bouton quitter
%quitter = Button(fenetre, text="Quitter", command=fenetre.destroy)
%quitter.grid(row=4, column=2, sticky=N)
%
%#QUATRIEME COLONNE (=)
%
%etiquette = Label(fenetre, text="=")
%etiquette.grid(row=2, column=3, sticky=N)
%
%#CINQUIEME COLONNE
%
%etiquette = Label(fenetre, text="Resultat")
%etiquette.grid(row=0, column=4, sticky=N)
%
%#Premier champ, numerateur
%numerateur3 = StringVar()
%champ3n = Entry(fenetre, textvariable=numerateur3, justify=CENTER, state=DISABLED)
%champ3n.grid(row=1, column=4, sticky=N)
%
%#Barre de separation
%separateur = Label(fenetre, text="------------")
%separateur.grid(row=2, column=4, sticky=N)
%
%#Deuxieme champ, denominateur
%denominateur3 = StringVar()
%champ3d = Entry(fenetre, textvariable=denominateur3, justify=CENTER, state=DISABLED)
%champ3d.grid(row=3, column=4, sticky=N)
%
%#Bouton calcul
%valider = Button(fenetre, text="Calculer", command=calcul)
%valider.grid(row=4, column=4, sticky=N)
%
%#Ex 4.3 - Notez qu'il faut aussi mettre un parametre par defaut dans calcul()
%fenetre.bind("<Key>", calcul)
%
%fenetre.mainloop()
%
%\end{lstlisting}

\end{solution}



\begin{solution}[ex:fractions_tkinter_3]

\begin{lstlisting}[basicstyle=\scriptsize\ttfamily]
from Tkinter import *

def gcd(numerateur, denominateur):
    #Algorithme d'Euclide pour calculer le plus grand diviseur commun (gcd)
    while denominateur != 0:
        reste = numerateur % denominateur
        numerateur = denominateur
        denominateur = reste

    return numerateur

def calcul(e=None):
    # le parametre e=None est necessaire

    #Ceci est pour eviter d'avoir des messages d'erreur
    #car il tentera de faire les calculs avant que tous les champs soient remplis
    if numerateur1.get() == "" or numerateur2.get() == "" or denominateur1.get()=="" or denominateur2.get() == "":
        return

    n1 = int(numerateur1.get())
    d1 = int(denominateur1.get())
    n2 = int(numerateur2.get())
    d2 = int(denominateur2.get())

    #Addition
    if choix.get() == 1:
        operation = "+"
        #Ici on prend simplement d1 * d2 comme denonminateur commun
        denominateur = d1 * d2
        numerateur = n1*d2 + n2*d1

    #Soustraction
    if choix.get() == 2:
        operation = "-"
        denominateur = d1 * d2
        numerateur = n1*d2 - n2*d1

    #Multiplication
    if choix.get() == 3:
        operation = "*"
        numerateur = n1*n2
        denominateur = d1*d2

    #Division
    if choix.get() == 4:
        operation = "/"
        numerateur = n1*d2
        denominateur = n2*d1

    #Pour la simplification, on doit calculer le plus grand diviseur commun
    g = gcd(numerateur, denominateur)

    #Et on entre les valeurs de la fraction simplifiee dans le champ resultat
    #via les textvariables.
    numerateur3.set( str(numerateur/g) )
    denominateur3.set( str(denominateur/g) )

    ###Historique ###

    ligne = str(n1)+"/"+str(d1)+" "+operation+" "+str(n2)+"/"+str(d2)+" = "+numerateur3.get()+"/"+denominateur3.get()+"\n"

        #Eviter les doublons
    if len(listeHisto) == 0 or ligne!=listeHisto[-1]:
        listeHisto.append(ligne)
        #Le widget Message affiche des strings, on ne peut pas afficher la liste directement
        #Certains ont voulu faire une boucle pour le faire, mais il y a la fonction join
        historique["text"] = "Historique:\n" + "".join(listeHisto[-5:])
        #La fonction join permet de prendre les elements d'une liste et les concatener sous forme de chaine
        #De plus, on peut ajouter un caractere entre chaque element
        #La syntaxe de join est assez speciale, c'est  "".join(liste)
        #et dans les "" vous pouvez mettre un element a inserer entre chaque ligne
        #Mais ici on veut juste concatener donc on laisse vide


fenetre = Tk()

#PREMIERE COLONNE

etiquette = Label(fenetre, text="Fraction 1")
etiquette.grid(row=0, column=0, sticky=N)

#Premier champ, numerateur
numerateur1 = Entry(fenetre, justify=CENTER)
numerateur1.grid(row=1, column=0, sticky=N)

#Barre de separation
separateur = Label(fenetre, text="------------")
separateur.grid(row=2, column=0, sticky=N)

#Deuxieme champ, denominateur
denominateur1 = Entry(fenetre, justify=CENTER)
denominateur1.grid(row=3, column=0, sticky=N)

#DEUXIEME COLONNE (radiobutton)

etiquette = Label(fenetre, text="Operation")
etiquette.grid(row=0, column=1, sticky=N)

choix = IntVar()
choix.set(1)

#Ex 4.2: grace a command=calclul dans les radiobutton, le calcul est lance quand on clique sur le bouton
r1 = Radiobutton(fenetre, text="Addition", variable=choix, value=1, command=calcul)
r1.grid(row=1, column=1, sticky=W)
r2 = Radiobutton(fenetre, text="Soustraction", variable=choix, value=2, command=calcul)
r2.grid(row=2, column=1, sticky=W)
r3 = Radiobutton(fenetre, text="Multiplication", variable=choix, value=3, command=calcul)
r3.grid(row=3, column=1, sticky=W)
r4 = Radiobutton(fenetre, text="Division", variable=choix, value=4, command=calcul)
r4.grid(row=4, column=1, sticky=W)

#TROISIEME COLONNE

etiquette = Label(fenetre, text="Fraction 2")
etiquette.grid(row=0, column=2, sticky=N)

#Premier champ, numerateur
numerateur2 = Entry(fenetre, justify=CENTER)
numerateur2.grid(row=1, column=2, sticky=N)

#Barre de separation
separateur = Label(fenetre, text="------------")
separateur.grid(row=2, column=2, sticky=N)

#Deuxieme champ, denominateur
denominateur2 = Entry(fenetre, justify=CENTER)
denominateur2.grid(row=3, column=2, sticky=N)

#Bouton quitter
quitter = Button(fenetre, text="Quitter", command=fenetre.destroy)
quitter.grid(row=4, column=2, sticky=N)

#QUATRIEME COLONNE (=)

etiquette = Label(fenetre, text="=")
etiquette.grid(row=2, column=3, sticky=N)

#CINQUIEME COLONNE

etiquette = Label(fenetre, text="Resultat")
etiquette.grid(row=0, column=4, sticky=N)

#Premier champ, numerateur
numerateur3 = StringVar()
champ3n = Entry(fenetre, textvariable=numerateur3, justify=CENTER, state=DISABLED)
champ3n.grid(row=1, column=4, sticky=N)

#Barre de separation
separateur = Label(fenetre, text="------------")
separateur.grid(row=2, column=4, sticky=N)

#Deuxieme champ, denominateur
denominateur3 = StringVar()
champ3d = Entry(fenetre, textvariable=denominateur3, justify=CENTER, state=DISABLED)
champ3d.grid(row=3, column=4, sticky=N)

#Bouton calcul
valider = Button(fenetre, text="Calculer", command=calcul)
valider.grid(row=4, column=4, sticky=N)

#Ex 4.3 - Notez qu'il faut aussi mettre un parametre par defaut dans calcul()
fenetre.bind("<Key>", calcul)



#On initialise la liste de l'historique
listeHisto = []

#Widget Message pour afficher l'historique
#On modifie la fonction calcul du TP04 pour qu'elle fasse l'historique
historique = Message(fenetre, text="Historique:", width=500) #width donne la longueur max du message en pixels (pour 1 ligne)
historique.grid(row=5, column=0, columnspan=3, sticky=W) #columnspan permet au message de s'etendre sur 3 colonnes

fenetre.mainloop()
\end{lstlisting}

\end{solution}



\begin{solution}[ex:fractions_tkinter_4]

\begin{lstlisting}[basicstyle=\scriptsize\ttfamily]
from Tkinter import *

def gcd(numerateur, denominateur):
    #Algorithme d'Euclide pour calculer le plus grand diviseur commun (gcd)
    while denominateur != 0:
        reste = numerateur % denominateur
        numerateur = denominateur
        denominateur = reste

    return numerateur

def calcul(e=None):
    # le parametre e=None est necessaire pour l'ex 4.3

    #Ceci est juste pour l'ex 4.3, pour eviter d'avoir des messages d'erreur
    #car il tentera de faire les calculs avant que tous les champs soient remplis
    if numerateur1.get() == "" or numerateur2.get() == "" or denominateur1.get()=="" or denominateur2.get() == "":
        return

    n1 = int(numerateur1.get())
    d1 = int(denominateur1.get())
    n2 = int(numerateur2.get())
    d2 = int(denominateur2.get())

    #On cherche quelle valeur est selectionnee dans la listbox
    choix = str(listb.curselection()[0])

    #Addition
    if choix == "0":
        choix = "+"
        #Ici on prend simplement d1 * d2 comme denonminateur commun
        denominateur = d1 * d2
        numerateur = n1*d2 + n2*d1

    #Soustraction
    if choix == "1":
        choix = "-"
        denominateur = d1 * d2
        numerateur = n1*d2 - n2*d1

    #Multiplication
    if choix == "2":
        choix = "*"
        numerateur = n1*n2
        denominateur = d1*d2

    #Division
    if choix == "3":
        choix = "/"
        numerateur = n1*d2
        denominateur = n2*d1

    #Pour la simplification, on doit calculer le plus grand diviseur commun
    g = gcd(numerateur, denominateur)

    #Et on entre les valeurs de la fraction simplifiee dans le champ resultat
    #via les textvariables.
    numerateur3.set( str(numerateur/g) )
    denominateur3.set( str(denominateur/g) )

    ### Historique ###

    if histActive: #Ex 1.3
        #Ex 1.1
        #On commence par preparer la ligne a mettre dans l'historique
        ligne = str(n1)+"/"+str(d1)+" "+choix+" "+str(n2)+"/"+str(d2)+" = "+numerateur3.get()+"/"+denominateur3.get()+"\n"

        #Eviter les doublons
        if len(listeHisto) == 0 or ligne!=listeHisto[-1]:
            listeHisto.append(ligne)

            #Le widget Message affiche des strings, on ne peut pas afficher la liste directement
            #Certains ont voulu faire une boucle pour le faire, mais il y a la fonction join
            historique["text"] = "Historique:\n" + "".join(listeHisto[-5:]) #Ex 1.2

            #La fonction join permet de prendre les elements d'une liste et les concatener sous forme de chaine
            #De plus, on peut ajouter un caractere entre chaque element
            #La syntaxe de join est assez speciale, c'est  "".join(liste)
            #et dans les "" vous pouvez mettre un element a inserer entre chaque ligne
            #Mais ici on veut juste concatener donc on laisse vide.

            #Ex 1.4
            fichier.write(ligne)

def activerHist():  # fonction pour activer/desactiver l'historique
 global histActive
 #global permet a une variable locale de fonction de devenir globale
 #ici on doit faire ca car histActive est utilisee dans la fonction calcul
 histActive = not histActive;
 if (histActive):
  txt = "Desactiver historique" # il faut changer le texte du menu
  historique.config(fg="Black") # ainsi que la couleur du texte d'historique
 else:
  txt = "Activer historique"  # il faut changer le texte du menu
  historique.config(fg="gray")# ainsi que la couleur du texte d'historique

 actmenu.entryconfig(0, label=txt)

def ecraserHist():  # pour ecraser l'historique (seulement sur l'ecran)
 global listeHisto
 #ici aussi il faut global pour que l'affecation de la ligne suivante
 #concerne bien la variable listeHisto en dehors de la fonction
 listeHisto = [] # on ecrase la liste de l' historique
 historique["text"] = "Historique:\n" # et aussi le texte


fenetre = Tk()

#PREMIERE COLONNE

etiquette = Label(fenetre, text="Fraction 1")
etiquette.grid(row=0, column=0, sticky=N)

#Premier champ, numerateur
numerateur1 = Entry(fenetre, justify=CENTER)
numerateur1.grid(row=1, column=0, sticky=N)

#Barre de separation
separateur = Label(fenetre, text="------------")
separateur.grid(row=2, column=0, sticky=N)

#Deuxieme champ, denominateur
denominateur1 = Entry(fenetre, justify=CENTER)
denominateur1.grid(row=3, column=0, sticky=N)

#DEUXIEME COLONNE (listbox)

etiquette = Label(fenetre, text="Operation")
etiquette.grid(row=0, column=1, sticky=N)


#Listbox au lieu de RadioButton
#On veut changer le script un minimum, alors on garde la variable de choix.
#Mais on doit changer tous les choix.get() en choix tout court
listb = Listbox(fenetre, height=4, width=5, exportselection=0)
listb.grid(row=1, column=1, rowspan=4, sticky=N)

#On remplit la liste
for operation in ["+", "-", "*", "/"]:
    listb.insert(END, operation)

#Selection par defaut du + (evite message d'erreur moche au debut si on ne selectionne rien)
listb.select_set(0)

#Un clic sur l'element de la liste lance le calcul
listb.bind("<Double-1>", calcul)

#TROISIEME COLONNE

etiquette = Label(fenetre, text="Fraction 2")
etiquette.grid(row=0, column=2, sticky=N)

#Premier champ, numerateur
numerateur2 = Entry(fenetre, justify=CENTER)
numerateur2.grid(row=1, column=2, sticky=N)

#Barre de separation
separateur = Label(fenetre, text="------------")
separateur.grid(row=2, column=2, sticky=N)

#Deuxieme champ, denominateur
denominateur2 = Entry(fenetre, justify=CENTER)
denominateur2.grid(row=3, column=2, sticky=N)

#Bouton quitter
quitter = Button(fenetre, text="Quitter", command=fenetre.destroy)
quitter.grid(row=4, column=2, sticky=N)

#QUATRIEME COLONNE (=)

etiquette = Label(fenetre, text="=")
etiquette.grid(row=2, column=3, sticky=N)

#CINQUIEME COLONNE

etiquette = Label(fenetre, text="Resultat")
etiquette.grid(row=0, column=4, sticky=N)

#Premier champ, numerateur
numerateur3 = StringVar()
champ3n = Entry(fenetre, textvariable=numerateur3, justify=CENTER, state=DISABLED)
champ3n.grid(row=1, column=4, sticky=N)

#Barre de separation
separateur = Label(fenetre, text="------------")
separateur.grid(row=2, column=4, sticky=N)

#Deuxieme champ, denominateur
denominateur3 = StringVar()
champ3d = Entry(fenetre, textvariable=denominateur3, justify=CENTER, state=DISABLED)
champ3d.grid(row=3, column=4, sticky=N)

#Bouton calcul
valider = Button(fenetre, text="Calculer", command=calcul)
valider.grid(row=4, column=4, sticky=N)

#Ex 4.3 - Notez qu'il faut aussi mettre un parametre par defaut dans calcul()
fenetre.bind("<Key>", calcul)


#Notez que les values pour les radiobutton ont ete changee en +-*/
#Car dans la correction du ex. 98 il y avait un IntVar qui prenait 1234 en valeur
#Ce qui est moins opti pour le cas ou on doit faire un historique
#En effet, on pourra reprendre choix pour mettre le symbole de l'operation.

#On initialise la liste de l'historique
listeHisto = []

#Widget Message pour afficher l'historique
#On modifie la fonction calcul du TP04 pour qu'elle fasse l'historique
historique = Message(fenetre, text="Historique:", width=500) #width donne la longueur max du message en pixels (pour 1 ligne)
historique.grid(row=5, column=0, columnspan=3, sticky=W) #columnspan permet au message de s'etendre sur 3 colonnes


#NB: les fonctions de cet exercice sont definiies en debut de script

histActive = True

#Menu Principal
menubar = Menu(fenetre)

#Sous-Menu actions
actmenu = Menu(menubar, tearoff=0)
actmenu.add_command(label="Desactiver historique", command=activerHist)
actmenu.add_command(label="Ecraser historique", command=ecraserHist)
actmenu.add_command(label="Quitter", command=fenetre.destroy)

#On met le sous-menu dans le menu principal
menubar.add_cascade(label="Actions", menu=actmenu)

#Finalement, on attache le menu a la fenetre
fenetre.config(menu=menubar)



#On doit verifier si le fichier existe.
#On peut le faire avec os.path.exists()
import os
if os.path.exists("histoire.txt"):  # si le fichier existe deja  il faut l'ouvrir en lecture/ecriture
 modeDOuverture = "r+" #r+ permet a la fois de lire le fichier et d'ecrire
else:
 modeDOuverture = "w+" # il faut le creer d'abord

fichier = open("histoire.txt", modeDOuverture)
listeHisto = fichier.readlines()
historique["text"] = "Historique:\n" + "".join(listeHisto[-5:])

fenetre.mainloop()

#On ferme le fichier quand l'utilisateur quitte la fenetre
#NB: Avant que la fenetre ne soit fermee, le fichier ne sera pas fini.
fichier.close()

\end{lstlisting}


\end{solution}



\begin{solution}[ex:html_1]

\begin{lstlisting}[style=verbatim]
<html>
 <head>
  <meta http-equiv="Content-Type" content="text/html; charset=UTF-8" />
  <title>Ma page web</title>
 </head>
 <body>
  <h1>Mon exercice sur le HTML</h1>
   <hr />
   <h2>Exercice 1</h2>
    <p>A) Un paragraphe simple aligné à gauche</p>
    <p align="center">B) Un autre paragraphe de deux lignes centré:<br />
       avec un mot en <i>italique</i>, un autre mot en <b>gras</b> et le mot <font color="red">Attention</font> en rouge!
    </p>
    <p align="right">C) C'est le dernier paragraphe de la première section. <br />
       Il est aligné à droite et contient un lien vers le site de l'<a href="http://www.unil.ch">UNIL</a>.
    </p>
   <hr />
   <h2>Exercice 2</h2>
    <p align="justify">A) Vous
    pouvez intégrer les images dans votre page web grâce a la balise <b>&lt;img&gt;</b>.
       L'attribut <b>src</b> de la balise devra contenir l'URL d'une image (locale ou d'internet).
       Par exemple le logo de l'UNIL se trouve à l'adresse suivante: <br />
       http://www.unil.ch/modules/unil-core/html/banners/unil/v14logo_unil_75.png <br />
       <img src="http://www.unil.ch/modules/unil-core/html/banners/unil/v14logo_unil_75.png" /><br />
       Veuillez noter que le paragraphe est justifié.
    </p>
    <p>B) Un tableau simple de trois lignes et deux colonnes:
       <table border="2px solid" width="50%">
        <tr><td>cas 1.1</td><td>cas 1.2</td></tr>
        <tr bgcolor="gray"><td>cas 2.1</td><td>cas 2.2</td></tr>
        <tr><td>cas 3.1</td><td>cas 3.2</td></tr>
       </table>
    </p>
 </body>
</html>
\end{lstlisting}

\begin{center}
\includegraphics[width=0.8\textwidth]{graphics/tp12_tree_complete.pdf}
\end{center}

\end{solution}



\begin{solution}[ex:html_2]

\begin{lstlisting}
def enregistreEchequier(fNom, taille):
    code  = "<html><body>";         # nouvelle tableau dans un fichier html
    code += "<table width="+str(taille*50)+" height="+str(taille*50)+">";
    case  = " ";

    estNoir = False;            # couleur de la case au debut

    for y in range(taille):         # pour chaque ligne
        code += "<tr>\n";         # nouvelle ligne du tableau
        for x in range(taille):     # pour chaque case
            if (estNoir):       # on ajoute le code d'une case
                code += "<td width=50 align='center' bgcolor='black'>"+case+"</td>\n"
            else:
                code += "<td width=50 align='center' bgcolor='white'>"+case+"</td>\n"

            estNoir = not estNoir;

        estNoir = not estNoir;
        code += "</tr>\n"         # le code de la fin de ligne

    code += "</table>";         # le tableau se termine
    code += "</body>";          #
    code += "</html>";

    with open(fNom, "w+") as f:     # on ouvre le fichier pour ecriture
        f.write(code);          # on ecrit le code du tableau dans le fichier
    #on ferme le fichier



enregistreEchequier("test.html", 100);
\end{lstlisting}


\end{solution}



\newpage

\begin{solution}[ex:xml_livre]

\begin{lstlisting}[style=verbatim]
<?xml version="1.0" encoding="iso-8859-1"?>
<livre>
 <titre>Mon livre</titre>
 <auteurs>
  <auteur>
   <nom>Brillant</nom>
   <prenom>Alexandre</prenom>
  </auteur>
  <auteur>
   <nom>Briand</nom>
   <prenom>Aristide</prenom>
  </auteur>
 </auteurs>
 <chapitres>
  <chapitre>
   <titre>Chapitre 1</titre>
   <sections>
    <section>
     <titre>Section 1</titre>
     <paragraphes>
      <paragraphe>Premier paragraphe</paragraphe>
      <paragraphe>Deuxième paragraphe</paragraphe>
     </paragraphes>
    </section>
    <section>
     <titre>Section 2</titre>
     <paragraphes>
      <paragraphe>Premier paragraphe</paragraphe>
      <paragraphe>Deuxième paragraphe</paragraphe>
     </paragraphes>
    </section>
   </sections>
  </chapitre>
  <chapitre>
   <titre>Chapitre 2</titre>
   <sections>
    <section>
     <titre>Section 1</titre>
     <paragraphes>
      <paragraphe>Premier paragraphe</paragraphe>
      <paragraphe>Deuxième paragraphe</paragraphe>
     </paragraphes>
    </section>
    <section>
     <titre>Section 2</titre>
     <paragraphes>
      <paragraphe>Premier paragraphe</paragraphe>
      <paragraphe>Deuxième paragraphe</paragraphe>
     </paragraphes>
    </section>
   </sections>
  </chapitre>
 </chapitres>
</livre>
\end{lstlisting}

Nous avons fait le choix de créer des balises supplémentaires telles que auteurs, chapitres, sections, paragraphes pour éviter de mélanger des ensembles distincts, comme le titre. Cela présente l'avantage de créer des blocs homogènes (tels que les auteurs, les chapitres, les sections...).

\end{solution}



\begin{solution}[ex:xml_livre_2]

\begin{lstlisting}[style=verbatim]
<?xml version="1.0" encoding="iso-8859-1"?>
<livre titre="Mon livre">
 <auteurs>
  <auteur nom="Brillant" prenom="Alexandre"/>
  <auteur nom="Briand" prenom="Aristide"/>
 </auteurs>
 <chapitres>
  <chapitre titre="Chapitre 1">
   <section titre="Section 1">
    <paragraphe>Premier paragraphe</paragraphe>
    <paragraphe>Deuxième paragraphe</paragraphe>
   </section>
   <section titre="Section 2">
    <paragraphe>Premier paragraphe</paragraphe>
    <paragraphe>Deuxième paragraphe</paragraphe>
   </section>
  </chapitre>
  <chapitre titre="Chapitre 2">
   <section titre="Section 1">
    <paragraphe>Premier paragraphe</paragraphe>
    <paragraphe>Deuxième paragraphe</paragraphe>
   </section>
   <section titre="Section 2">
    <paragraphe>Premier paragraphe</paragraphe>
    <paragraphe>Deuxième paragraphe</paragraphe>
   </section>
  </chapitre>
 </chapitres>
</livre>
\end{lstlisting}
Comme l'élément titre disparaît au profit de l'attribut, nous pouvons alléger notre structure en éliminant les blocs superflus, comme sections ou paragraphes.

Parsing:
\begin{lstlisting}
# coding:utf-8

import xml.etree.ElementTree as ET

tree = ET.parse('livre_2.xml')
root = tree.getroot()

print "Auteurs"
auteurs = root.find("auteurs")
for auteur in auteurs:
    print auteur.attrib["nom"], auteur.attrib["prenom"]

print "Chapitres"
chapitres = root.find("chapitres")
for chapitre in chapitres:
    titre = chapitre.attrib["titre"]
    n_sections = len(chapitre.findall("section"))
    print "Le chapitre", titre, "possède", n_sections, "sections."
\end{lstlisting}


\end{solution}

\newpage


\begin{solution}[ex:regexp_0]

\begin{lstlisting}
import re

motif = r'(python)'
if re.search(motif, "j'aime python est cool"):
   print "match!"


motif = r'^([0-9]|-)+$'
if re.search(motif, "0041-342-42422-2"):
   print "match!"
\end{lstlisting}

\end{solution}



\begin{solution}[ex:regexp_1]

\noindent\begin{tabularx}{\textwidth}{|c|X|X|}
\hline
\# & Motif & Regexp \\
\hline
1 & Un mot qui contient la suite de lettres ``uni'' &  \verb@^\w*uni\w*$@ \\
\hline
2 & Un mot qui contient la lettre ``o'' ou deux lettres ``n'' &  \verb@^\w*(o|nn)\w*$@ \\
\hline
3 & Un mot qui commence par les lettres ``uni'' & \verb@^uni\w*$@ \\
\hline
4 & Un mot qui commence par au moins un chiffre &  \verb@^\d\w*$@ \\
\hline
5 & Un mot qui commence par la lettre ``E'' majuscule &  \verb@^E\w*$@ \\
\hline
6 & Un mot qui commence ou finit par au moins deux voyelles &  \verb@^[aeyuio]{2}\w*$|^\w*[aeyuio]{2}$@ \\
\hline
7 & Un mot d'exactement 3 caractères &  \verb@^\w\w\w$@ \\
\hline
8 & Un acronyme d'au moins 3 caractères constitué de lettres majuscules &  \verb@^[A-Z]{3,}$@ \\
\hline
9 & Un numéro de téléphone lausannois sous sa forme locale \texttt{021-nnn-nn-nn} & \verb@^021-\d{3}(-\d{2}){2}$@ \\
\hline
10 & Un chiffre romain compris entre 10 et 15 inclusivement &  \verb@^X(I{0,3}|I?V)$@ \\
\hline
11 & Une phrase affirmative de 12 mots séparés par des espaces &  \verb@^[^\s]+(\s+[^\s]+){11}\.$@ \\
\hline
12 & Une phrase qui se termine par un point d'exclamation\newline où les mots sont séparés par des espaces ou des virgules &  \verb@^[^\s]+(\s+[^\s]+)+\!$@ \\
\hline
13 & Deux mots séparés par un trait d'union & \verb@^[^\d\s]+-[^\d\s]+$@ \\
\hline
14 & Un mot précédé par un article définit (i.e. le, la, les et l') &  \verb@^((l'|(le|la|les)\s)[\w]+)$@ \\
\hline
\end{tabularx}
\

\end{solution}



\begin{solution}[ex:regexp_3]

L'expression rationnelle demandée est : \verb&^[A-Z][A-Za-z]*(\ [A-Za-z]+)*\.$&

Script d'exemple:
\begin{lstlisting}
import re
while True:
    s = raw_input("> ")
    if re.match("^[A-Z][A-Za-z]*(\ [A-Za-z]+)*\.",s):
        print "La phrase est correcte."
    else:
        print "La phrase n'est pas correcte."
\end{lstlisting}
\end{solution}



\begin{solution}[ex:regexp_4]

\begin{lstlisting}
import re
compteur = 0
ligne = 1
with open("charte_unil.txt") as fichier:
    for texte in fichier.readlines():
        n = len(re.findall("uni", texte, re.IGNORECASE))
        if n > 0:
            print "Le mot apparait dans la ligne", ligne
            compteur += n
        ligne += 1

print "Le mot apparait", compteur, "fois dans le fichier."
\end{lstlisting}

\end{solution}





\end{document}
