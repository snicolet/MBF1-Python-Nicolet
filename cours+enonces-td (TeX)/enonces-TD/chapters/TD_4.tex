\section{Chapitre 4: Boucles et tests conditionnels}

\setcounter{exercice}{0}

\bigskip

%%%%%%%%
%%%%%%%%
%%%%%%%%

\medskip
\begin{myboxi}[Rappel sur les boucles \inlinecode{for}]
La fonction \inlinecode{range(start,stop,[step])} produit un objet qui retourne une liste de nombres entiers telle que: la premi\`ere valeur est \inlinecode{start} (\inlinecode{start} vaut $0$ par d\'efaut); les valeurs suivantes sont incr\'ement\'ees par pas de \inlinecode{step} (\inlinecode{step} vaut $1$ par d\'efaut); et ce jusqu'\`a atteindre la valeur \inlinecode{stop} non incluse.

\smallskip

Une boucle \inlinecode{for} signifie ``pour \inlinecode{var} allant de \inlinecode{start} \`a \inlinecode{stop} par pas de \inlinecode{step}, effectuer les instructions...''. La syntaxe d'une boucle \inlinecode{for} utilise la fonction \inlinecode{range()} et se pr\'esente comme suit:
\begin{lstlisting}
for var in range(start,stop[,step]):
	instructions...
\end{lstlisting}
On peut forcer l'interruption d'une boucle \inlinecode{for} avec l'instruction \inlinecode{break}.

\smallskip

On rappelle \'egalement la syntaxe pour parcourir les cha\^ines de caract\`eres, les listes, les tuples et les dictionnaires \`a l'aide de boucles \inlinecode{for}.
\begin{lstlisting}
# parcours de chaines, listes ou tuples
for variable in chaine_liste_ou_tuple:
	instructions...
# parcours des cles d'un dico
for variable in dico.keys(): # (ou for variable in dico:)
	instructions...
# parcours des valeurs d'un dico
for variable in dico.values():
	instructions...
# parcours des couples "cle-valeur" d'un dico
for variable_cle, variable_valeur in dico.items():
	instructions...
\end{lstlisting}
\end{myboxi}


\medskip
\begin{myboxi}[Rappel sur les boucles \inlinecode{while}]
Une boucle \inlinecode{while} signifie ``tant que la condition est vraie, effectuer les instructions...''. La syntaxe d'une boucle \inlinecode{while} se pr\'esente comme suit:
\begin{lstlisting}
while condition:
	instructions...
\end{lstlisting}
On peut forcer l'interruption d'une boucle \inlinecode{while} avec l'instruction \inlinecode{break}.
\end{myboxi}


\medskip
\begin{myboxi}[Rappel sur les tests conditionnels \inlinecode{if... else...}]
On rappelle la syntaxe d'un test conditionnel \inlinecode{if... else...} (les blocs \inlinecode{elif} et \inlinecode{else} sont facultatifs):
\begin{lstlisting}
if condition:
	instructions...
elif:
	instructions...
...
else:
	instructions...
\end{lstlisting}
\end{myboxi}



\medskip
\begin{exercice}[Boucles ``for'']\label{TD3_ex1}
\

\begin{enumerate}
\item \'Ecrivez un script qui demande \`a l'utilisateur d'entrer un entier $N$ et affiche ensuite la table de multiplication de $N$.
\item Modifiez votre script de sorte que la table de multiplication soit affich\'ee sur une seule ligne. Il suffit d'ajouter une virgule apr\`es votre instruction \inlinecode{print}.
\item Modifiez votre script de telle mani\`ere qu'il affiche, line par ligne, les tables de multiplications de tous les entiers plus petit ou \'egaux \`a $N$. Utilisez une double boucle \inlinecode{for}.
\item \'Ecrivez un script qui demande \`a l'utilisateur d'entrer un entier $N$ et affiche ensuite un petit dessin comme ci-dessous de hauteur $N$.
\begin{lstlisting}[style = verbatim]
# exemple pour N = 4
*
**
***
****
\end{lstlisting}
\item Essayez de modifier votre script de telle sorte que le dessin affich\'e soit comme ci-dessous. D\'eterminez r\`egle qui donne le nombre d'espaces et d'\'etoiles de chaque ligne.
\begin{lstlisting}[style = verbatim]
# exemple pour N = 5
    *
   * *
  * * *
 * * * *
* * * * *
\end{lstlisting}
\end{enumerate}
\end{exercice}

\begin{filecontents*}{temp.tex}
\newpage
\begin{solution}[TD3_ex1]
\begin{lstlisting}
...
\end{lstlisting}
\end{solution}
\newpage
\end{filecontents*}
\appendsolution





\medskip
\begin{exercice}[Boucles ``for'']\label{TD3_ex2}
\

\noindent On consid\`ere les deux listes suivantes:
\begin{flushleft}
\inlinecode{jours = [31, 28, 31, 30, 31, 30, 31, 31, 30, 31, 30, 31]} \\
\inlinecode{mois = ["January", "February", "March", "April", "May", "June", "July", "August", "September", "October", "November", "December"]} 
\end{flushleft}
\begin{enumerate}
\item \'Ecrivez un script qui cr\'ee la liste de $12$ tuples suivante:
\begin{flushleft}
\inlinecode{mj = [("January", 31), ("February", 28), ("March", 31), ("April", 30), ("May", 31), ("June", 30), ("July", 31), ("August", 31), ("September", 30), ("October", 31), ("November", 30), ("December", 31)]} 
\end{flushleft}
\item \'Ecrivez un script qui cr\'ee une liste de $365$ cha\^ines de caract\`eres correspondant aux $365$ jours de l'ann\'ee, comme ci-dessous. Utilisez une double boucle \inlinecode{for} sur votre liste \inlinecode{mj}.
\begin{flushleft}
\inlinecode{annee = ["1 January", "2 January",..., "1 February", "2 February",...]} 
\end{flushleft}
\item Initialisez la liste suivante:
\begin{flushleft}
\inlinecode{jours_semaine = ["Monday", "Tuesday", "Wednesday", "Thursday", "Friday", "Saturday", "Sunday"]} \\
\end{flushleft}
Supposons que le premier janvier soit un lundi. \'Ecrivez un script qui cr\'ee une deuxi\`eme liste des $365$ jours de l'ann\'es comprenant les jours de la semaine en plus, comme ci-dessous. Effectuez une double boucle de taille $365$, utilisez les listes \inlinecode{annee} et \inlinecode{jours_semaine} et pensez \`a utiliser l'op\'erateur modulo sur les indices de la liste \inlinecode{jours_semaine}. 
\begin{flushleft}
\inlinecode{annee2 = ["Monday 1 January", "Tuesday 2 January",...]} 
\end{flushleft}
\item \`A partir de vos listes \inlinecode{annee} et \inlinecode{jours_semaine} , cr\'eez un dictionnaire dont les cl\'es sont les \'el\'ements de \inlinecode{annee} et les valeurs sont les jours de la semaine correspondants, comme ci dessous.
\begin{flushleft}
\inlinecode{dico_annee = ["1 January" : "Monday", "2 January" : "Tuesday",...]} 
\end{flushleft}
\item \`A quel jour de la semaine correspond au $28$ octobre?
\end{enumerate}
\end{exercice}

\begin{filecontents*}{temp.tex}
\newpage
\begin{solution}[TD3_ex2]
\begin{lstlisting}
...
\end{lstlisting}
\end{solution}
\newpage
\end{filecontents*}
\appendsolution



\medskip
\begin{exercice}[Boucles ``while''] \label{TD3ex3}
\

\noindent 
\begin{enumerate}
\item \'Ecrivez un programme qui demande \`a l'utilisateur d'entrer $3$ notes entre $0$ et $20$. Le programme affichera ensuite le minimum, le maximum et la moyenne des notes entr\'ees. Utilisez une boucle \inlinecode{while}. Les notes entr\'ees seront stock\'ees progressivement dans une liste. Utilisez un parcours de votre liste de notes pour calculer la moyenne de celles-ci.
\item Modifiez le programme de sorte que celui-ci commence par demander \`a l'utilisateur le nombre de notes $N$ qu'il d\'esire entrer, puis proc\`ede comme pr\'ec\'edemment.
\item Modifiez le programme de sorte que l'utilisateur puisse entrer autant de notes qu'il le d\'esire et termine sa saisie par une certaine instruction, comme ``fin'' par exemple.
\end{enumerate}


\end{exercice}


\begin{filecontents*}{temp.tex}
\begin{solution}[TD3ex3]
\begin{lstlisting}[style=verbatim]
...
\end{lstlisting}
\end{solution}
\newpage
\end{filecontents*}
\appendsolution





\medskip
\begin{exercice}[Boucles ``while''] \label{TD3ex4}
\

\noindent \'Ecrivez un programme qui choisit un nombre entier au hasard entre $1$ et $100$, puis demande \`a l'utilisateur de le deviner. Si l'utilisateur entre un nombre trop petit ou trop grand, le programme devra afficher ``plus haut'' ou ``plus bas'', respectivement. Le programme s'arr\^ete lorsque le nombre a \'et\'e trouv\'e. Pour g\'en\'erer un nombre entier au hasard, importez la librairie \inlinecode{random} et utilisez la fonction \inlinecode{randint()}.
\end{exercice}


\begin{filecontents*}{temp.tex}
\begin{solution}[TD3ex4]
\begin{lstlisting}[style=verbatim]
...
\end{lstlisting}
\end{solution}
\newpage
\end{filecontents*}
\appendsolution





%\begin{filecontents*}{temp.tex}
%
%\begin{solution}[ex:]
%
%\end{solution}
%
%
%\end{filecontents*}
%\appendsolution


