\section{Chapitre 7: Graphisme (MatPlotLib) - simulations de Monte-Carlo}

\setcounter{exercice}{0}

\bigskip

%%%%%%%%
%%%%%%%%
%%%%%%%%

\begin{exercice}[Tracer des courbes avec MatPlotLib] \label{ex:fichiers_1} \
\
\begin{enumerate}
\item Implémenter deux fonctions réelles $f$ et $g$ de votre choix.
\item Définir trois listes X, Y et Z, où la liste X contient les valeurs de $x$ dans $[-5,5]$ par pas de $0.05$, la liste Y les valeurs de $y=f(x)$ et Z les valeurs de $z=g(x)$.
\item Vérifier que les trois listes X, Y et Z ont la même longueur.
\item Tracer les courbes $Y=f(X)$ et $Z=g(X)$ sur le même graphique à l'aide de la librairie MatPlotLib.
\end{enumerate}
\end{exercice}

%%%%%%%%
%%%%%%%%
%%%%%%%%
\medskip



\begin{exercice}[Simulation de Monte-Carlo] \label{ex:fichiers_1} \
\

\noindent Le but de cet exercice est d'utiliser la méthode de Monte-Carlo pour obtenir une approximation statistique de l'aire de la surface à l'intérieur d'une courbe fermée. On obtiendra en particulier une valeur approchée de $\pi$.
\medskip
\begin{enumerate}

\item Utiliser la fonction \inlinecode{eval()} de Python pour évaluer une chaîne de caractères en remplaçant
les variables par leur valeur dans l'environnement actuel.
\begin{lstlisting}
x = 0.7
y = -0.8
formule = " x**2 + y**2 <= 1.0 "
test = eval(formule)
print(test)
\end{lstlisting}
\item On veut tirer $N$ point aléatoires dans le carré $[-1,1]\times[-1,1]$. A l'aide de la librairie NumPy, générer un couple $(X,Y)$, où $X$ et $Y$ sont deux listes de longueur $N$
de tirages de variables aléatoires uniformes réelles dans l'intervalle $[-1,1]$.
\item Si je tire N points aléatoires uniformément dans le carré $[-1,1]\times[-1,1]$, quelle est la probabilité que chaque point soit à l'intérieur du disque de centre $(0,0)$ et de rayon 1 ?
\item Implémenter la fonction \inlinecode{montecarlo(listeX, listeY, formule)} qui prend en argument une formule et les coordonnées de $N$ points aléatoires. La fonction tracera en rose le nuage des points qui valident la formule, et en vert le nuage des points qui invalident la formule.
\begin{myboxi}[Indication] Pour tracer un nuage de points avec MatPlotLib, utiliser \inlinecode{plt.scatter()}.
\end{myboxi}
\item Changer le titre du graphique tous les mille points pour afficher l'évolution de la valeur approchée de la surface à l'intérieur de la courbe (utiliser la fonction \inlinecode{plt.title()} de MatPlotLib).
\item Pour avoir une animation plus fluide, ralentir l'affichage en faisant une pause de 0.2 secondes tous les mille points affichés à l'aide de la fonction \inlinecode{plt.pause()}.
\item Afficher la simulation de Monte-Carlo avec 10000 points, et le calcul approché des aires,
pour les trois courbes suivantes :
$$ x^2 + y^2 < 1 $$
$$ \sqrt{|x|} + \sqrt{|y|} < 1 $$
$$ x^2 + (0.3 + 1.5y - |y|^{0.6})^2 < 1 $$
\end{enumerate}
\end{exercice}

\begin{filecontents*}{temp.tex}
\begin{solution}[ex:resultats_3]
\begin{lstlisting}
...
\end{lstlisting}
\end{solution}
\end{filecontents*}
\appendsolution





