\section{Chapitre 7: Librairie Pandas}

\setcounter{exercice}{0}

\bigskip


%%%%%%%%
%%%%%%%%
%%%%%%%%
\medskip
\begin{exercice}[Librairie Pandas: fichiers CSV et dataframe] 
\
\begin{enumerate}
\item Ouvrez le fichier \verb#diamonds.csv# dans un \'editeur de code et analysez sa structure.
Ecrivez une fonction Python qui ouvre le fichier \verb#diamonds.csv# et importe les donn\'ees
sous forme d'une matrice (dataframe) de la librairie Pandas. Extrayez la colonne des prix, et calculez ses
statistiques \'el\'ementaires (moyenne, \'ecart type, quantiles, etc.). Appliquez la fonction $f(x) = 1.3x + 5$
\`a la colonne des prix. Tracez le nuage de points des prix par rapport au nombre de carats.

\begin{myboxi}[Indications] Pour utiliser la librairie \inlinecode{pandas} il faut inclure la librairie en utilisant la commande \inlinecode{import pandas as pd}. Une documentation rapide de la librairie se trouve dans le r\'epertoire 
\inlinecode{http://cassio.free.fr/assas/info/MBF2/doc/} .\end{myboxi}


\item Triez les valeurs de la matrice par nombre de carats d\'ecroissants, puis modifiez votre script de
mani\`ere \`a pouvoir enregistrer les diamants dans des fichiers diff\'erents pour chaque
type de taille (par exemple \verb#ideal.csv# pour la taille "Ideal" et \verb#very-good.csv#
pour la taille "Very Good", etc.). Le format des fichiers enregistr\'es doit \^etre le m\^eme que celui de \verb#diamonds.csv#.

\end{enumerate}
\end{exercice}


%%%%%%%%
%%%%%%%%
%%%%%%%%
\medskip
\begin{exercice}[Utilisation de données méteo] 
\
\begin{enumerate}
\item Ouvrez le fichier \verb#AnomalieTempe_RCP2.6_mensuel.txt# dans un \'editeur de code et 
analysez sa structure. Essayez de lire le fichier avec la fonction  \inlinecode{pandas.read_csv()} :
quel(s) probl\`eme(s) rencontre-t-on ? 

\item Nettoyez les donn\'ees, renommez le fichier en 
\verb#anomalies.txt# et lisez les donn\'ees dans un dataframe correct.
Vous pouvez chercher sur le Web et lire la documentation de la fonction
 \inlinecode{pandas.read_csv()} pour connaitre ses options et l'adapter \`a vos donn\'ees.


\end{enumerate}

\end{exercice}



%\begin{filecontents*}{temp.tex}
%
%\begin{solution}[ex:]
%
%\end{solution}
%
%
%\end{filecontents*}
%\appendsolution


