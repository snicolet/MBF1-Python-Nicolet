\section{Chapitre 5: Fonctions }

\setcounter{exercice}{0}

\bigskip

%%%%%%%%
%%%%%%%%
%%%%%%%%

\medskip
\begin{myboxi}[Rappel sur les fonctions et les proc\'edures]
La syntaxe pour d\'efinir une fonction:
\begin{lstlisting}
def nom_fonction(arg_1,...,arg_N):
	instructions...
	return valeur   # dans le cas d'une procedure, simplement return
\end{lstlisting}
\end{myboxi}

\medskip
\begin{myboxi}[Paramètres par défaut dans les fonctions et les procédures]
La syntaxe pour d\'efinir une fonction avec des arguments qui poss\`edent des valeurs par d\'efaut est la suivante:
\begin{lstlisting}
def nom_fonction(arg_1 = val_1,...,arg_N = val_N):
	instructions...
	return valeur # dans le cas d'une procedure, simplement return
\end{lstlisting}
\end{myboxi}

\medskip
\begin{exercice}[Calcul d'intégrale par la méthode des rectangles]\label{TD5_ex1}
\
\begin{enumerate}
\item Montrez que l'équation $x^2 + y^2 = 1$ est l'équation du cercle de centre $(0,0)$ et de rayon 1.
En déduire l'équation développée du cercle de centre $(1,0)$ et de rayon 1 en faisant le changement de variable $x'=x-1$.
\item Soit $f$ la fonction $f(x) = \sqrt{2x - x^2}$. Quels sont l'ensemble de définition et le graphe de $f$ ? 
Vérifiez vos résultats en utilisant la calculatrice graphique du site \url{http://desmos.com}
\item Soit $g$ la fonction définie par $g(x) = f(|x|)$. Quel est l'ensemble de définition de $g$ ? Tracez son graphe
sur \url{http://desmos.com}
\item Expliquez pourquoi $$\int_{-2}^{2}g(x)dx = \pi$$
\item Programmez en Python les deux fonctions $f(x)$ et $g(x)$.
\item Définissez une fonction \inlinecode{integrale(phi, a, b, n)} qui renvoie une approximation de
l'intégrale de la fonction $\phi$ sur $[a,b]$ par la méthode des rectangles avec $n$ subdivisions.
On pourra consulter \url{https://fr.wikipedia.org/wiki/Somme_de_Riemann} et utiliser la méthode du point médian.
\item Appliquez la question précédente à la fonction $g$ sur $[-2,2]$ et en déduire une approximation numérique de $\pi$.
\end{enumerate}
\end{exercice}


\begin{filecontents*}{temp.tex}
\begin{solution}[TD3ex4]
\begin{lstlisting}[style=verbatim]
...
\end{lstlisting}
\end{solution}
\newpage
\end{filecontents*}
\appendsolution


\medskip
\begin{exercice}[fonctions, paramètres par défaut]\label{TD5_ex2}
\
\begin{enumerate}
\item D\'efinissez une fonction \inlinecode{surf_cercle()} qui calcule la surface d'un cercle dont le rayon est donn\'e en argument. Pour avoir acc\`es \`a la valeur de $\pi$, ajouter l'instruction \inlinecode{from math import pi}.
\item Modifiez votre fonction \inlinecode{surf_cercle} de sorte que votre argument poss\`ede une valeur par d\'efaut de $1$.
\item D\'efinissez une fonction \inlinecode{vol_boite()} qui renvoie le volume d'une bo\^ite parall\'el\'epip\'edique dont on fournit les trois dimensions en arguments.
\item Modifiez votre fonction de telle sorte que si un seul argument est fourni, la bo\^ite est consid\'er\'ee comme cubique (l'argument \'etant l'ar\^ete de ce cube); si deux arguments sont fournis, la bo\^ite est consid\'er\'ee comme un parall\'el\'epip\`ede \`a base carr\'ee (le premier argument est le c\^ot\'e du carr\'e, et le second la hauteur du parall\'el\'epip\`ede); si trois arguments sont fournis, la bo\^ite est consid\'er\'ee comme un parall\'el\'epip\`ede g\'en\'eral. Pour cela, donnez des valeurs par d\'efaut \`a vos arguments et utiliser un test conditionnel pour savoir si tel ou tel param\`etre poss\`ede telle ou telle valeur par d\'efaut...
\end{enumerate}
\end{exercice}

\medskip
\begin{exercice}[fonctions sur les chaines de caractères]\label{TD5_ex2}
\
\begin{enumerate}
\item D\'efinissez une fonction \inlinecode{remplacement()} qui prend trois arguments \inlinecode{c1}, \inlinecode{c2} et \inlinecode{ch} et qui remplace tous les caract\`eres \inlinecode{c1} par des caract\`eres \inlinecode{c2} dans la cha\^ine caract\`eres \inlinecode{ch}.
\item Modifiez votre fonction de telle sorte que si les arguments \inlinecode{c1}, \inlinecode{c2} et \inlinecode{ch} ne sont pas sp\'ecifi\'es, alors ils prendront les valeurs \inlinecode{" "}, \inlinecode{"*"} et \inlinecode{""}, respectivement.
\end{enumerate}
\end{exercice}

\begin{filecontents*}{temp.tex}
\newpage
\begin{solution}[TD5_ex2]
\begin{lstlisting}
...
\end{lstlisting}
\end{solution}
\newpage
\end{filecontents*}
\appendsolution



\medskip
\begin{exercice}[comptage de lexèmes]\label{TD4_ex2}
\

\begin{enumerate}
\item En vous inspirant de l'exercice $2$ du TD $2$, cr\'eez deux fonctions \inlinecode{nb_car(chaine)} et \inlinecode{nb_mots(chaine)} qui comptent le nombre de caract\`eres et le nombre de mots d'une cha\^ine de caract\`eres \inlinecode{chaine} donn\'ee en argument.
\item Comparez votre solution avec l'usage de la fonction \inlinecode{str.split()} disponible dans la librairie
standard de Python; dont la documentation est par exemple dans \url{https://docs.python.org/3/library/stdtypes.html}.
\end{enumerate}
\end{exercice}

\begin{filecontents*}{temp.tex}
\newpage
\begin{solution}[TD4_ex2]
\begin{lstlisting}
...
\end{lstlisting}
\end{solution}
\newpage
\end{filecontents*}
\appendsolution



%\begin{filecontents*}{temp.tex}
%
%\begin{solution}[ex:]
%
%\end{solution}
%
%
%\end{filecontents*}
%\appendsolution


