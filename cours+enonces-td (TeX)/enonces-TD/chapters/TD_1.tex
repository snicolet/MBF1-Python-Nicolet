\section{Chapitre 1: Variables, types de base, fonctions d'entr\'ee et de sortie}

\setcounter{exercice}{0}

\bigskip

%%%%%%%%
%%%%%%%%
%%%%%%%%



\begin{exercice}[les nombres: types int et float] \label{ex1a}
\

\noindent Initialisez trois variables \inlinecode{x}, \inlinecode{y} et \inlinecode{z} \`a des nombre entiers ou flottants, puis familiarisez-vous avec les op\'erateurs arithm\'etiques de base en les appliquant sur ces variables. On rappelle que:
\begin{itemize}
\item[$\bullet$] l'addition se note \inlinecode{+}
\item[$\bullet$] la soustraction se note \inlinecode{-}
\item[$\bullet$] la multiplication se note \inlinecode{*}
\item[$\bullet$] l'exponentiation se note \inlinecode{**}
\item[$\bullet$] le modulo se note \%
\item[$\bullet$] la division se note \inlinecode{/}
\end{itemize}
En Python 2 (contrairement \`a Python 3), la division par d\'efaut est la division enti\`ere: cela signifie que si les deux nombres sont des entiers, c'est la division enti\`ere qui est appliqu\'ee. Si au moins un des deux nombres est r\'eel, c'est la division r\'eelle qui est appliqu\'ee.
\end{exercice}


\begin{filecontents*}{temp.tex}
\begin{solution}[ex1a]
\begin{lstlisting}[style=verbatim]
...
\end{lstlisting}
\end{solution}
\newpage
\end{filecontents*}
\appendsolution



\medskip
\begin{exercice}[les cha\^ines de caract\`eres: type string] \label{ex1b}
\

\noindent Initialisez deux variables \inlinecode{s} et \inlinecode{t} \`a des cha\^ines de caract\`eres, puis familiarisez-vous avec les op\'erateurs de base sur les cha\^ines en les appliquant sur ces variables. On rappelle que:
\begin{itemize}
\item[$\bullet$] la concat\'enation de cha\^ines de caract\`eres s'obtient par l'op\'erateur \inlinecode{+}
\item[$\bullet$] la r\'ep\'etition de cha\^ines de caract\`eres s'obtient par l'op\'erateur \inlinecode{*}
\item[$\bullet$] l'extraction de l'\'el\'ement d'indice \inlinecode{n} d'une cha\^ine \inlinecode{s} s'obtient par la syntaxe \inlinecode{s[n]} (les indices commencent \`a $0$)
\item[$\bullet$] l'extraction de l'\'el\'ement d'indice \inlinecode{n}, mais en partant de la fin de la cha\^ine \inlinecode{s}, s'obtient par la syntaxe \inlinecode{s[-n]} (dans ce cas, les indices commencent \`a $-1$)
\item[$\bullet$] l'extraction de la sous-cha\^ine de \inlinecode{s} d'indices \inlinecode{m} compris jusqu'\`a l'indice \inlinecode{n} non compris s'obtient par la syntaxe \inlinecode{s[m:n]}
\item[$\bullet$] la longueur d'une cha\^ine \inlinecode{s} s'obtient par la syntaxe \inlinecode{len(s)}
\end{itemize}
Pour la syntaxe \inlinecode{s[m:n]}, si l'indice \inlinecode{m} n'est pas sp\'ecifi\'e, il vaut $0$ par d\'efaut; si l'indice \inlinecode{n} n'est pas sp\'ecifi\'e, il vaut \inlinecode{len(n)} par d\'efaut.
\end{exercice}


\begin{filecontents*}{temp.tex}
\begin{solution}[ex1b]
\begin{lstlisting}[style=verbatim]
...
\end{lstlisting}
\end{solution}
\newpage
\end{filecontents*}
\appendsolution




\medskip
\begin{exercice}[les Bool\'eens: type bool] \label{ex1c}
\

\noindent En Python, les valeurs bool\'eennes se disent \inlinecode{True} et \inlinecode{False}. Les op\'erateurs logiques sont les suivants:
\begin{itemize}
\item[$\bullet$] la n\'egation se note \inlinecode{not}
\item[$\bullet$] le ``et'' logique, appel\'e aussi disjonction, se note \inlinecode{and}
\item[$\bullet$] le ``ou'' logique, appel\'e aussi conjonction, se note \inlinecode{or}
\end{itemize}
Initialisez quelques variables aux valeurs Bool\'eennes \inlinecode{True} et \inlinecode{False} et testez les op\'erateurs logiques sur vos variables en tapant quelques expressions logiques de votre choix.

\smallskip

\noindent De plus, les diff\'erents op\'erateurs de comparaison entre nombres (ou autre types \'egalement) se note \inlinecode{<}, \inlinecode{<=}, \inlinecode{>}, \inlinecode{>=}, \inlinecode{==}, \inlinecode{!=}. Les r\'esultat des \'evaluations d'expressions contenant ces op\'erateurs sont des valeurs Bool\'eennes, \inlinecode{True} ou \inlinecode{False}. Par exemple, l'expression \inlinecode{2 < 3} est \'evalu\'ee en \inlinecode{True} et l'expression \inlinecode{2 == 3} est \'evalu\'ee en \inlinecode{False}. Tapez quelques expressions de votre choix contentant ces op\'erateurs de comparaison.

\begin{myboxi}[Point important] Fa\^ites attention de ne pas confondre le symbole d'\'egalit\'e ``\inlinecode{=}'', qui sert \`a instancier une variable (par exemple \inlinecode{x = 2}), de celui de double \'egalit\'e ``\inlinecode{==}'', qui sert \`a tester l'\'egalit\'e entre deux op\'erands (par exemple \inlinecode{2 == 3}). \end{myboxi}
\end{exercice}


\begin{filecontents*}{temp.tex}
\begin{solution}[ex1c]
\begin{lstlisting}[style=verbatim]
...
\end{lstlisting}
\end{solution}
\newpage
\end{filecontents*}
\appendsolution




\medskip
\begin{exercice}[Cha\^ines de caract\`eres] \label{ex2}
\
\begin{enumerate}
\item Initialisez deux variables appel\'ees \inlinecode{prenom} et \inlinecode{nom} contenant votre pr\'enom et votre nom, respectivement.
\item En utilisant l'op\'erateur de concat\'enation sur les cha\^ines de caract\`eres (\inlinecode{+}), initialisez une variable \inlinecode{nom_complet} contenant votre pr\'enom et votre nom s\'epar\'es par un espace.
\item En utilisant l'op\'erateur de multiplication sur les cha\^ines de caract\`eres (\inlinecode{*}) et la fonction \inlinecode{print}, imprimez vos nom et pr\'enom une centaine fois \`a l'\'ecran s\'epar\'es par ``\verb# * #'' (espace - \'etoile - espace).
\item En utilisant les op\'erateurs d'extraction de sous-cha\^ines, initialisez \`a partir de la variable \inlinecode{nom_complet} une autre variable \inlinecode{initiales} contenant vos initiales s\'epar\'ees par un espace.
\item Imprimez vos initiales une centaine fois \`a l'\'ecran s\'epar\'es par ``\verb# * #'' (espace - \'etoile - espace).
\end{enumerate}
\end{exercice}


\begin{filecontents*}{temp.tex}
\begin{solution}[ex2]
\begin{lstlisting}[style=verbatim]
...
\end{lstlisting}
\end{solution}
\newpage
\end{filecontents*}
\appendsolution



\medskip
\begin{exercice}[Cha\^ines de caract\`eres] \label{ex3}
\

\noindent En utilisant la fonction \inlinecode{input()}, \'ecrivez un script qui demande \`a l'utilisateur d'entrer son pr\'enom, puis d'entrer son nom, puis affiche automatiquement le pr\'enom et le nom, les initiales et la premi\`ere lette du nom de famille, comme illustr\'e ci-dessous:
\begin{lstlisting}[style=verbatim]
>>> Quel est votre prenom? Astor
>>> Quel est votre nom? Piazzolla
Astor Piazzolla
A. P.
Premiere lettre du nom de famille: P
\end{lstlisting}
\end{exercice}


\begin{filecontents*}{temp.tex}
\begin{solution}[ex3]
\begin{lstlisting}[style=verbatim]
...
\end{lstlisting}
\end{solution}
\newpage
\end{filecontents*}
\appendsolution



\medskip
\begin{exercice}[int et float] \label{ex4}
\

\noindent Cr\'eez un script permettant de convertir des temp\'eratures de degr\'es Celsius en degr\'es Fahrenheit et vice-versa. La formule de conversion est la suivante:
$$
F = \frac{9}{5} \cdot C + 32
$$
Votre script devra demander \`a l'utilisateur la temp\'erature et afficher les deux conversions comme suit:
\begin{lstlisting}[style=verbatim]
Quelle est la temperature? 20.2
20.2 C = 68.36 F
20.2 F = -6.555555555 C
\end{lstlisting}
\end{exercice}


\begin{filecontents*}{temp.tex}
\begin{solution}[ex4]
\begin{lstlisting}[style=verbatim]
...
\end{lstlisting}
\end{solution}
\newpage
\end{filecontents*}
\appendsolution




\medskip
\begin{exercice}[Bool\'eens] \label{ex5}
\

\noindent Cr\'eez un programme qui, premi\`erement, demande \`a l'utilisateur d'entrer des valeurs de v\'erit\'e (\inlinecode{True} ou \inlinecode{False}) pour trois variables propositionnelle $A$, $B$ et $C$, puis, deuxi\`emement, demande \`a l'utilisateur d'entrer ``en fran\c{c}ais'' une expression de logique Bool\'eenne sur ces trois variables propositionnelles, et enfin, affiche l'\'evaluation de cette expression en fonction des valeurs de v\'erit\'e de $A$, $B$ et $C$. Le d\'eroulement du programme devra ressembler \`a l'ex\'ecution ci-dessous.

\begin{lstlisting}[style=verbatim]
Entrer une valeur de verite pour A: True
Entrer une valeur de verite pour B: False
Entrer une valeur de verite pour C: True
Entrer une expression Boolenne (exprimee avec des "non", "ou", "et" et des parentheses): 
non (A et B) ou (C et non B)
L'evaluation de votre expression Booleenne est: True
\end{lstlisting}
\end{exercice}

\begin{myboxi}[Indication] Si \inlinecode{s} est une cha\^ine de caract\`eres, la fonction \inlinecode{s.replace(old, new)} retourne une autre cha\^ine de caract\`eres o\`u toutes les occurrences de la sous-cha\^ine \inlinecode{old} dans \inlinecode{s} sont remplac\'ees par la la sous-cha\^ine \inlinecode{new}. Ainsi, il vous faudra remplacer les sous-cha\^ines \inlinecode{"non"}, \inlinecode{"et"} et \inlinecode{"ou"} par \inlinecode{"not"}, \inlinecode{"and"} et \inlinecode{"or"}, puis \'evaluer votre expression Bool\'eenne \`a l'aide de la fonction \inlinecode{bool()} pour obtenir le r\'esultat d\'esir\'e.\end{myboxi}


\begin{filecontents*}{temp.tex}
\begin{solution}[ex5]
\begin{lstlisting}[style=verbatim]
...
\end{lstlisting}
\end{solution}
\newpage
\end{filecontents*}
\appendsolution







%\begin{filecontents*}{temp.tex}
%
%\begin{solution}[ex:]
%
%\end{solution}
%
%
%\end{filecontents*}
%\appendsolution


