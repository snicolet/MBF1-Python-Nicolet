\section{Chapitre 3: Expressions arithmétiques, entrées/sorties}

\setcounter{exercice}{0}

\bigskip

%%%%%%%%
%%%%%%%%
%%%%%%%%



\medskip
\begin{myboxi}[Le concile de Nicée]
La définition de la date de Pâques a été
 fixée en 325 lors du concile de Nicée :
P\^aques est le dimanche qui suit le quatorzième jour de la lune qui atteint cet âge au 21 mars ou immédiatement après.

Autrement dit, c'est le premier dimanche qui suit ou qui coïncide avec la première pleine lune apres le 21 mars (marquant le début du printemps).
\end{myboxi}


\medskip
\begin{myboxi}[Algorithme de Oudin pour calculer la date de Pâques]
Calculer la date de Pâques est loin d'être une chose facile. On connait plusieurs méthodes, 
dont la suivante due à Oudin qui a l'avantage de
demander peu d'opérations. Quoiqu'il existe une version généraliste de l'agorithme de Oudin
(sans limite de siècle), nous présentons ici une forme simplifiée uniquement valable pour le
calendrier grégorien, donc pour {\textbf{toute année postérieure à 1583}}.


\medskip
\medskip

Dans l'algorithme ci-dessous, chaque ligne introduit une nouvelle variable qui dépend 
des précédentes. Les divisions doivent toujours être entières. L'exemple en bleu indique les calculs pour l'année 2007.
\medskip

\begin{enumerate}[label=\arabic*.]
\item G représente l'écart d'or diminué de 1 : diviser l'année par 19, en prendre le reste ;  
\color{blue} (2007/19 = 105, or 105 × 19 = 1995 et il nous faut 2007, donc l'écart vaut G = 12) \color{black}
\item C et C4 permettent le suivi des années bissextiles : diviser l'année par 100 puis encore par 4 ;  
\color{blue} (2007/100 = C = 20 et 20/4 = C4 = 5) \color{black}
\item E : diviser 8×C +13 par 25 sans les décimales;  
\color{blue} (8 × 20 + 13 = 173/25 = E = 6) \color{black}
\item H qui dépend de l'épacte : diviser 19 × G + C - C4 - E + 15 par 30, en prendre le reste ; 
\color{blue} (on prend le reste d'une division selon le même principe que pour G : 252/30 = 8, or 8 × 30 = 240 et il nous faut 252, donc l'écart vaut H = 12) \color{black}
\item K : diviser H par 28 ; 
\color{blue} (12/28 = K = 0) \color{black}
\item P : diviser 29 par H + 1 ; 
\color{blue} (29/13 = P = 2) \color{black}
\item Q : diviser 21-G par 11; 
\color{blue} ( 21 - 12 = 9/11 = Q = 0) \color{black}
\item I représente le nombre de jours entre la pleine lune pascale et le 21 mars : (-1 + K × P × Q) × K + H ; 
\color{blue} ( 0 × 2  × 0 - 1 = -1  × -0 = 0 + 12 = I = 12) \color{black}
\item B : diviser l'année par 4 et enlever les décimales, y ajouter l'année; 
\color{blue} (2007/4 = 501 + 2007 = 2508) \color{black}
\item J1 : Additionner B + I + 2 + C4 et retrancher C; 
  \color{blue} (J1 =2507) \color{black}
\item J2 calcule le jour de la lune pascale (0=dimanche 1=lundi...6=samedi) : diviser J1 par 7 et en prendre le reste ; 
\color{blue} (on calcule toujours le reste d'une division selon le même principe qu'avec G et H, le résultat est J2 = 1) \color{black}
\item R = 28 + I - J2. C'est le résultat final, enfin !
\color{blue} (R = 39)  \color{black}
\item R représente la date de Pâques dans le mois de mars, s'il dépasse 31 cela signifie que l'on déborde sur avril (... 30 correspond au 30 mars, 31 au 31 mars, 32 au 1er avril, 33 au 2 avril, ...). Retrancher 31 le cas échéant pour obtenir la date d'avril. 
\color{blue} (Pâques 2007 tombe donc le 8 avril) \color{black}
\end{enumerate}
\end{myboxi}

\newpage

\medskip
\begin{exercice}[Date de Pâques]\label{TD3_ex1}
\

\begin{enumerate}
\item Calculez la date de Pâques pour l'année 2007 en suivant la méthode de Oudin.
\item Debugguez au fur et à mesure les valeurs des variables avec la fonction 
\inlinecode{print} pour vérifier votre implémentation de l'algorithme à l'aide de l'exemple bleu.
\
\begin{lstlisting}
 G = ...
 C = ...
 C4 = ...
 print( f"G = {G}" )
 print( f"C = {C}" )
 print( f"C4 = {C4}" )
\end{lstlisting}
\end{enumerate}

\end{exercice}

\medskip
\begin{exercice}[Utilisation de fonctions et de fichiers]\label{TD3_ex2}
\

\begin{enumerate}
\item Modifiez le script pour mettre le calcul de Pâques
dans une fonction \inlinecode{date_de_paques(N)}. Votre fonction doit renvoyer une
chaine de caractères, par exemple "\inlinecode{Dimanche 23 mars 2008}"
\item Réalisez un programme qui demande une année à l'utilisateur dans le programme
principal, puis affiche la date de Pâques pour cette année.
\item Quelle sera la prochaine année (à partir de 2025) où Pâques sera fêté un 1er avril ?
\item Réalisez un programme qui affiche toutes les années du XXIe siècle durant lesquelles
Pâques est fêté une 1er avril.
\item Déterminez les années du XXIe siècle pour lesquelles Pâques est fêté le plus tard.
\item \'Ecrivez toutes les dates de Pâques du XXIe siècle dans un fichier texte.
\
\begin{lstlisting}
 with open("toto.txt", "w") as f:
     f.write("I have written something in the file !!!")
\end{lstlisting}

\item Déterminez le jour de votre anniversaire (lundi, mardi, mercredi, etc) pour une année N
donnée (indication : Pâques est un dimanche).
\end{enumerate}

\end{exercice}




%\begin{filecontents*}{temp.tex}
%
%\begin{solution}[ex:]
%
%\end{solution}
%
%
%\end{filecontents*}
%\appendsolution


