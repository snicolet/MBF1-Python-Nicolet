\section{Chapitre 6: Programmation structurée}

\setcounter{exercice}{0}

\bigskip


%%%%%%%%
%%%%%%%%
%%%%%%%%




\medskip
\begin{exercice}[table de multiplication]\label{TD4_ex3}
\
\begin{enumerate}
\item D\'efinissez une fonction \inlinecode{TableMult(n, debut = 1, fin = 10)} qui imprime la table de multiplication de \inlinecode{n} entre les multiplicateurs \inlinecode{debut} et \inlinecode{fin}.
\item Am\'eliorez votre fonction de telle sorte que si l'argument \inlinecode{debut} et plus grand que \inlinecode{fin}, la table de multiplication soit affich\'ee correctement, dans l'ordre d\'ecroissant.
\item Cr\'eez une fonction similaire o\`u les param\`etres \inlinecode{n}\inlinecode{debut} et \inlinecode{fin} sont demand\'es \`a l'utilisateur plut\^ot que pass\'es en argument.
\end{enumerate}
\end{exercice}

\begin{filecontents*}{temp.tex}
\newpage
\begin{solution}[TD4_ex3]
\begin{lstlisting}
...
\end{lstlisting}
\end{solution}
\newpage
\end{filecontents*}
\appendsolution



\medskip
\begin{exercice}[tour de magie]\label{TD4_ex3}
\

\noindent Le but de cet exercice est d'implémenter un tour de magie sur un jeu de 52 cartes.
\medskip
\begin{enumerate}
\item Parmi les structures de données : entier, flottant, couple, chaine de caractères, liste et dictionnaire,
quelle est celle qui vous parait la plus adaptée pour représenter l'ensemble des couleurs d'un jeu de cartes 
dans l'ordre Trèfle, Carreau, Coeur, Pique ? Même question pour les hauteurs des cartes de l'as au roi.
\item Après avoir défini les ensembles de hauteurs et de couleurs en Python, définissez les deux fonctions 
\inlinecode{hauteur(carte)} et \inlinecode{couleur(carte)} qui renvoient respectivement l'index de la hauteur et de la couleur d'une carte dans vos ensembles. Faites attention que les index commencent à zéro en Python. Par exemple, on a:
\begin{lstlisting}[style = verbatim]
hauteur("valet de carreau") = 10
couleur("valet de carreau") = 1
\end{lstlisting}
\item Définissez la fonction \inlinecode{numero(carte)} qui renvoie le numéro d'une carte (un entier entre 0 et 51) dans l'ordre canonique : tous les Trèfles, tous les Carreaux, tous les Coeurs puis tous les Piques.
\begin{lstlisting}[style = verbatim]
numero("valet de carreau") = 23
\end{lstlisting}
\item Définir la bijection réciproque \inlinecode{carte(numero)} qui prend un numéro en argument et renvoie la carte correspondante. Testez votre implémentation en vérifiant que l'on a bien la relation de composition \inlinecode{n = numero(carte(n))} pour tout $n$.
\begin{lstlisting}[style = verbatim]
carte(23) = "le valet de carreau"
\end{lstlisting}
\item Programmez la fonction entree(prompt) qui demande dans la console une carte à l'utilisateur et renvoie la carte tapée par l'utilisateur, sous forme de chaîne de caractères. Votre fonction pourra être robuste et redemander la carte si l'ordinateur n'a pas compris.
\item Implémentez le tour de magie "The Fitch Cheney Five-Card Trick"
\begin{lstlisting}[style = verbatim]
saisir la premiere carte : le 10 de pique
saisir la deuxieme carte : le 4 de coeur
saisir la troisieme carte : le 10 de carreau
saisir la quatrieme carte : le roi de trefle

... hmm, je pense que la cinquieme carte est le 3 de pique
\end{lstlisting}
\end{enumerate}
\end{exercice}

\newpage


\medskip
\begin{exercice}[le jeu ``Motus'']\label{TD4_ex4}
\

\noindent Soit la liste de mots de cinq lettres suivantes:
\begin{flushleft}
\inlinecode{mots = ["arbre", "grave", "piece", "nuage", "crane", "sonne", "table", "herbe", "ecrou", "mulet"]} 
\end{flushleft}
Cr\'ez un programme qui impl\'emente le jeu ``Motus'' avec cette liste de mots. Plus pr\'ecis\'ement, au d\'epart, l'ordinateur choisit un mot au hasard dans cette liste (utilisez la fonction \inlinecode{randint()} de la librairie \inlinecode{random}). Ensuite, l'utilisateur dispose de $10$ essais pour deviner le mot. Pour cela, il proposera des lettres tour \`a tour. \`A chaque fois, l'ordinateur l'informera de si cette lettre est pr\'esente ou non dans le mot, et, si tel est le cas, mettra \`a jour les lettres d\'ej\`a devin\'ees. L'ex\'ecution de votre programme devra ressembler \`a quelque chose comme ci-dessous:
\begin{lstlisting}[style = verbatim]
Essai 6
Entrer une lettre: t

...et

Essai 7
Entrer une lettre: m

m..et
\end{lstlisting}
\begin{enumerate}
\item Commencez par d\'efinir une fonction \inlinecode{remplace(i, c, ch)} qui renvoie une cha\^ine dans laquelle le \inlinecode{i}-\`eme caract\`ere de la cha\^ine \inlinecode{ch} est remplac\'e par \inlinecode{c}.
\item Utilisez ensuite cette fonction dans votre programme. \`A chaque lettre propos\'ee par l'utilisateur, effectuez un parcours du mot \`a deviner, et, parall\`element, construisez un nouveau mot dans lequel les lettres non encore devin\'ees apparaissent comme des points alors que celles d\'ej\`a devin\'ees apparaissent clairement.
\item Modifiez votre programme de telle sorte que votre jeu soit impl\'ement\'e dans une proc\'edure \inlinecode{motus()} \`a un argument; cet argument devra correspondre au nombre d'essais maximum auquel l'utilisateur a le droit et aura une valeur par d\'efaut de $10$.
\end{enumerate}
\end{exercice}


\begin{filecontents*}{temp.tex}
\newpage
\begin{solution}[TD4_ex4]
\begin{lstlisting}
...
\end{lstlisting}
\end{solution}
\newpage
\end{filecontents*}
\appendsolution




\medskip
\begin{exercice}[le jeu ``Puissance $4$'' (long et plus difficile; tr\`es bon exercice r\'ecapitulatif)]\label{TD4_ex5}
\

\noindent Programmez un jeu ``Puissance $4$'' sur une grille de taille $N$. Le jeu devra avoir l'apparence comme montr\'e ci-dessous:
\begin{lstlisting}[style = verbatim]
. 	. 	. 	. 	. 	. 	. 	. 	

. 	. 	. 	. 	X 	. 	. 	. 	

. 	. 	. 	O 	. 	. 	. 	. 	

O 	. 	. 	. 	. 	. 	X 	. 	

. 	. 	. 	. 	. 	. 	. 	. 	

. 	. 	. 	O 	. 	. 	. 	. 	

. 	. 	. 	. 	. 	X 	. 	. 	

. 	. 	. 	. 	. 	. 	. 	. 	

Joueur 1: quelle position (ligne, colonne)?
\end{lstlisting}
\begin{enumerate}
\item Programmez une fonction \inlinecode{generer_grille(N = 4)} qui g\'en\`ere une grille d'une certaine taille \inlinecode{N} valant $4$ par d\'efaut. Chaque ligne de la grille est une liste de taille \inlinecode{N} dont les \'el\'ements sont des cha\^ines \inlinecode{"."}. La grille est alors repr\'esent\'ee comme la liste de ses lignes, comme illustr\'e ci-dessous.
\begin{lstlisting}[style = verbatim]
# exemple d'une grille de taille 3 x 3
# on a 3 listes de taille 3 remplies avec des "."
[[".", ".", "."], [".", ".", "."], [".", ".", "."]]
\end{lstlisting}
\item Programmez une fonction \inlinecode{placer_pion(joueur, grille, position)} qui permet de placer le pion du joueur \inlinecode{joueur} \`a la position \inlinecode{position} dans la grille \inlinecode{grille}. L'argument \inlinecode{joueur} vaut \inlinecode{1} ou \inlinecode{2}. S'il vaut, \inlinecode{1}, le pion plac\'e sera un rond (O majuscule) \inlinecode{"O"}; s'il vaut, \inlinecode{2}, le pion plac\'e sera une croix (X majuscule) \inlinecode{"X"}. L'argument \inlinecode{grille} est une liste de listes. L'argument \inlinecode{position} est une liste ou un tuple de la forme \inlinecode{(i,j)}.
\item Programmez une fonction \inlinecode{chercher_alignement(joueur, liste, n)} qui cherche si un alignement de \inlinecode{n} pions du joueur \inlinecode{joueur} existe dans la liste \inlinecode{liste} (cette liste repr\'esentera plus tard, une ligne, une colonne ou une diagonale). Suivant que l'argument \inlinecode{joueur} vaut \inlinecode{1} ou \inlinecode{2}, on cherchera un alignement de \inlinecode{n} \inlinecode{"O"} ou \inlinecode{n} \inlinecode{"X"}, respectivement.
\item Programmez une fonction \inlinecode{generer_lignes(grille)} qui retourne la liste de toutes les lignes de la grille \inlinecode{grille}. Remarquez que dans ce cas, il suffit de retourner la grille elle m\^eme, puisque celle-ci est donn\'ee comme la liste de ces lignes.
\item Programmez une fonction \inlinecode{generer_colonnes(grille)} qui retourne la liste de toutes les colonnes de la grille \inlinecode{grille}. L'id\'ee est de transposer la matrice \inlinecode{grille}...
\item Programmez une fonction \inlinecode{generer_diagonales1(grille)} qui retourne la liste de toutes les diagonales ``montantes'' de la grille \inlinecode{grille}. Pour cela, remarquez que chaque diagonale de ce type est constitu\'ee des \'el\'ements de la grille dont les indices donnent une m\^eme somme. Par exemple, la diagonale \inlinecode{[(3,1), (2,2), (1,3)]} est form\'ee de tous les \'el\'ements dont les indices ont une somme de $4$.
\item Programmez une fonction \inlinecode{generer_diagonales2(grille)} qui retourne la liste de toutes les diagonales ``descendantes'' de la grille \inlinecode{grille}. Pour cela, remarquez que chaque diagonale de ce type est constitu\'ee des \'el\'ements de la grille dont les indices donnent une m\^eme diff\'erence. Par exemple, la diagonale \inlinecode{[(1,3), (2,4), (3,5)]} d'une grille de taille $5$ est form\'ee de tous les \'el\'ements dont les indices ont une diff\'erence de $-2$.
\item Programmez une fonction \inlinecode{imprimer_grille(grille)} qui imprime la grille \inlinecode{grille} dans votre terminal de mani\`ere conviviale, comme illustr\'e ci-dessus. On rappelle que les cha\^ines \inlinecode{"\t"}, \inlinecode{"\n"} codent un espace de tabulation et un retour de ligne, respectivement.
\item En utilisant toutes les fonctions que vous avez programm\'ees jusque l\`a, d\'efinissez une fonction \inlinecode{jouer(n)} qui permet de jouer \`a Puissance $4$ sur une grille de taille \inlinecode{n}. Utilisez une grande boucle \inlinecode{while} qui demande \`a chaque joueur de placer un pion \`a tour de r\^ole, jusqu'\`a ce qu'un alignement de $4$ pions soit trouv\'e dans une ligne, une colonne ou une diagonale. Tout le programme peut se faire en $250$ lignes de code environ.
\end{enumerate}
\end{exercice}

\begin{filecontents*}{temp.tex}
\newpage
\begin{solution}[TD4_ex5]
\begin{lstlisting}
...
\end{lstlisting}
\end{solution}
\newpage
\end{filecontents*}
\appendsolution





%\begin{filecontents*}{temp.tex}
%
%\begin{solution}[ex:]
%
%\end{solution}
%
%
%\end{filecontents*}
%\appendsolution